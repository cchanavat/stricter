\documentclass[11pt,twoside]{article}

\usepackage{amsmath,amsthm,amsfonts,amssymb,mathtools,mathdots,scalerel,mathrsfs,stmaryrd,cmll}
\usepackage[utf8]{inputenc}
\usepackage[T1]{fontenc}
\usepackage{mlmodern, tikz-cd, quiver}
\usepackage[many]{tcolorbox}

\usepackage[shortlabels]{enumitem}
\usepackage[hidelinks]{hyperref}
\usepackage{tikz, fancyhdr, xparse, xcolor, tocloft}
\usepackage[british]{babel}
\usepackage[a4paper, top=4.5cm, bottom=4.5cm, left=4cm, right=4cm, asymmetric]{geometry}

\usepackage{crossreftools}
\usepackage[textwidth=3cm, textsize=small, colorinlistoftodos]{todonotes}

\usepackage{cctitle, titlesec}
\usepackage{etoolbox}

\usetikzlibrary{nfold}

\makeatletter
\patchcmd{\ttlh@hang}{\parindent\z@}{\parindent\z@\leavevmode}{}{}
\patchcmd{\ttlh@hang}{\noindent}{}{}{}
\makeatother

% general macros
\newcommand\eqdef{\coloneqq}
\newcommand\nbd{\nobreakdash-\hspace{0pt}}
\newcommand\idd[1]{\mathrm{id}_{#1}}
\newcommand\bigid[1]{\mathrm{Id}_{#1}}
\newcommand\invrs[1]{#1^{-1}}
\newcommand\after{\circ}

\newcommand\incl{\hookrightarrow}
\newcommand\surj{\twoheadrightarrow}
\newcommand\restr[2]{{#1}{\raisebox{0pt}{$|_{#2}$}}}

\newcommand\set[1]{\left\{ {#1} \right\}}
\newcommand\order[2]{#2^{(#1)}}


% category theory macros

\newcommand\opp[1]{#1^\mathrm{op}}

\newcommand\cat[1]{\mathbf{#1}}

\newcommand\fun[1]{\mathsf{#1}}
\DeclareMathOperator*{\colim}{colim}

\DeclareMathOperator*{\Ob}{Ob}

\def\raiseslice#1#2{\raisebox{-2pt}{$#1#2$}}
\newcommand{\slice}[2]{{#1{/}\mathpalette\raiseslice{#2}}}
% \newcommand\slice[2]{{#1}/\raisebox{-2pt}{$#2$}}

% special categories

\newcommand\rdcpx{\cat{RDCpx}}
\newcommand\rdcpxmap{\rdcpx_{\downarrow}}
\newcommand\rdcpxcomap{\rdcpx_{\uparrow}}
\newcommand\rdcpxcart{\rdcpx_{\mathsf{cart}}}

\newcommand{\atom}{{\scalebox{1.3}{\( \odot \)}}}
\newcommand{\smallatom}{\odot}
\newcommand{\Set}{\cat{Set}}
\newcommand{\Cat}{\cat{Cat}}
\newcommand{\Pos}{\cat{Pos}}
\newcommand{\dgmSet}{\atom\Set}
\newcommand{\sSet}{\cat{sSet}}

\newcommand{\pt}{\mathbf{1}}
\newcommand{\init}{\varnothing}

\newcommand{\bpt}[1]{#1_{**}}
% oriented poset macros
\DeclareMathOperator{\clos}{cl}
\newcommand\clset[1]{\mathrm{cl}\set{#1}}
\newcommand\gr[2]{#2_{#1}}
\newcommand\maxel[1]{\mathscr{M}\!\mathit{ax}\,#1}
\newcommand\bd[2]{\partial_{#1}^{#2}}

\newcommand\faces[2]{\Delta_{#1}^{#2}}
\newcommand\cofaces[2]{\nabla_{#1}^{#2}}
\newcommand\inter[1]{\mathrm{int}\,#1}

\newcommand\cp[1]{\,{\scriptstyle\#}_{#1}\,}
\newcommand\cpsub[1]{\triangleright_{#1}}
\newcommand\subcp[1]{\prescript{}{#1\!}{\triangleleft}\;}
\newcommand\gencp[1]{\,{\scriptstyle\widehat{\#}}_{#1}\,}

\newcommand\subs[3]{#1[#2/#3]}
\newcommand\compos[1]{\langle#1\rangle}

\newcommand\celto{\Rightarrow}
\newcommand\relcelto[1]{\Rightarrow_{\kern-3pt #1}}

\newcommand\submol{\sqsubseteq}

\newcommand\augm[1]{{{#1}_\bot}}

\newcommand\dual[2]{\fun{D}_{#1}{#2}}
\newcommand\join{\,{\star}\,}
\newcommand\gray{\otimes}
\newcommand\sus[1]{\fun{S}{#1}}

\newcommand{\tp}{\otimes}

\DeclareMathOperator{\lydim}{lydim}
%%%%%%%%%%% 


% diagrammatic set stuff
\DeclareMathOperator{\Pd}{Pd}
\DeclareMathOperator{\Rd}{Rd}
\DeclareMathOperator{\Eqv}{Eqv}
\DeclareMathOperator{\cell}{cell}
\DeclareMathOperator{\eqv}{eqv}
\DeclareMathOperator{\Dgn}{Dgn}
\DeclareMathOperator{\dgn}{dgn}
\DeclareMathOperator{\Nd}{Nd}
\DeclareMathOperator{\nd}{nd}

\newcommand\rev[1]{{#1}^\dag}

% unitors 

\newcommand{\un}{\varepsilon}
\newcommand{\lun}[1]{\lambda_{#1}}
\newcommand{\run}[1]{\rho_{#1}}
\newcommand{\hinv}[1]{\xi_{#1}}

\newcommand{\lcyl}[1]{\mathrm{L}_{#1}}
\newcommand{\rcyl}[1]{\mathrm{R}_{#1}}

\newcommand{\hcyl}[1]{\Xi_{#1}}

% special objects


% collapsible

\newcommand{\coll}[2]{{#2} / {#1}}
\newcommand{\zcoll}{\bullet}
\newcommand{\icoll}[1]{(#1)}
\newcommand{\mapcoll}{p}
% maps induced
\newcommand{\mapel}[1]{\iota_{#1}}
\newcommand{\imel}[2]{{#1}_{#2}}

% localisation
\newcommand{\loc}[2]{#1[#2^{-1}]}
\newcommand{\preloc}[2]{#1\set{{#2}^{-1}}}
\newcommand{\selfloc}[1]{\widetilde{#1}}
\newcommand{\locarr}{\tilde I}
\DeclareMathOperator{\Loc}{\fun{Loc}}
\newcommand{\revarr}{\tilde I}
\newcommand{\selflocm}[1]{\selfloc{#1}_{\m{}}}
\newcommand{\markarr}{\markmol{\arr}}

\newcommand{\td}[1]{\tilde{#1}}
% other
\renewcommand{\a}{\alpha}

%% coinductive classes
\newcommand{\B}{\mathcal{B}}
\DeclareMathOperator{\Linv}{\mathcal{L}}
\DeclareMathOperator{\Rinv}{\mathcal{R}}
\DeclareMathOperator{\Pst}{Pst}

% contexts
\renewcommand{\th}{\vartheta}

\newcommand{\F}{\fun{F}}
\newcommand{\G}{\fun{G}}

%% marked stuff
\newcommand{\m}[1]{{\mathsf{m}#1}}
\newcommand{\mrk}[1]{{#1}_{\m{}}}
\newcommand{\markmol}[1]{{#1}_{\m{}}}

\newcommand{\mdgmSet}{\atom\Set^{\m{}}}
\newcommand{\minmark}[1]{{#1}^{\flat}}
\newcommand{\maxmark}[1]{{#1}^{\sharp}}
\newcommand{\natmark}[1]{{#1}^{\natural}}

\newcommand{\minbdmap}[1]{\minmark{\partial}_{#1}}
\newcommand{\markbdmap}[1]{\partial^{\m{}}_{#1}}
\newcommand{\maxmarkabove}[2]{{#2}^{\sharp_{#1}}}

% model structures
\newcommand{\pp}[1]{\mathbin{\hat{#1}}}
\DeclareMathOperator{\an}{an}
\newcommand{\cof}[1]{{#1}\text{\nbd}\mathsf{cof}}
\newcommand{\pizero}{\pi_0}

%% other
\newcommand\cls[1]{\mathscr{#1}}

% anodyne sets
\newcommand{\Jn}[1]{J_{#1}}
\newcommand{\Jhorn}{J_{\mathsf{horn}}}
\newcommand{\Jcoind}{J_{\mathsf{coind}}}
\newcommand{\Jloc}{J_{\mathsf{loc}}}
\newcommand{\Jcomp}{J_{\mathsf{comp}}}

% simplicial notations

\newcommand{\simplex}[1]{\Delta[#1]}
\newcommand{\osimplex}[1]{\vec\Delta[#1]}
\newcommand{\simplexcat}{\Delta}
\newcommand{\isSet}{i_{\simplexcat}}
\newcommand{\rsSet}[1]{{#1}_{\simplexcat}}
\newcommand{\msSet}{\sSet^{\m{}}}

\newcommand{\thick}[1]{\mathfrak{th}_{#1}}
\newcommand{\rthick}[1]{\mathfrak{th}^{\mathsf{r}}_{#1}}
\newcommand{\rsnerve}{N^{\fun{rs}}}
\newcommand{\natnerve}{N^\natural}

\newcommand{\spine}[1]{\mathrm{Sp}[{#1}]}
\newcommand{\simplexorder}[1]{[#1]}
\newcommand{\bdsimplex}[1]{\partial[#1]}
\newcommand{\marksimplex}[1]{\simplex{#1}_{\m{}}}
\newcommand{\hornsimplex}[2]{\Lambda^{#1}[#2]}
\newcommand{\stratsimplex}[2]{\Delta^{#1}[#2]}

% omega cat notations

\newcommand{\omegaCat}{{\omega\Cat}}
\newcommand{\nCat}[1]{{#1\Cat}}
\newcommand{\somegaCat}{\omegaCat^>}
\newcommand{\snCat}[1]{\nCat{#1}^>}
\newcommand{\Comp}{\omega\cat{Comp}}
\newcommand{\nComp}[1]{{#1\cat{Comp}}}
\newcommand{\comp}[1]{*_{#1}}
\newcommand{\omegaReg}{\omega\cat{Reg}}
\newcommand{\omegaMol}{\omega\cat{Mol}}
\newcommand{\omegaRegSet}{[\opp{\omegaReg}, \Set]^{\mathsf{rw, pst}}}
\newcommand{\rc}{\fun{r}}
\newcommand{\rcs}{\fun{r}^{>}}

\newcommand{\pcell}[1]{\langle#1\rangle}
\let\amalg\relax
\DeclareMathOperator*{\amalg}{\mathrm{amalg}}

\newcommand{\trunc}[1]{\tau_{#1}}
\newcommand{\skel}[1]{\sigma_{#1} }
\newcommand\molec{\mathit{Mol}}
\newcommand\Atom{\mathit{Atom}}
\newcommand\molecin[1]{\slice{\molec}{#1}}
% \newcommand\molecinsmall[1]{\slice{\molec}{\rel#1}}
\newcommand\atomin[1]{\slice{\Atom}{#1}}

\newcommand{\globe}[1]{O^{#1}}
\newcommand{\dglobe}[1]{\vec{O}^{#1}}
\newcommand\arr{\dglobe{1}}
\newcommand{\rglobe}[1]{\selfloc{O}^{#1}}

\newcommand{\gcelto}{\to}

\DeclareMathOperator{\Dgm}{Dgm}

\newcommand\sd{\looparrowright}

\newcommand{\N}[1]{N_{#1}}


\newcommand{\eps}{\epsilon}
% folk model structure

\newcommand{\Icof}{\mathcal{I}}
\newcommand{\sJcof}{\mathcal{J}^>}
\newcommand{\Jcof}{\mathcal{J}}
\newcommand{\sfolkcof}{\mathrm{Cof}^{>}}
\newcommand{\wE}{\tilde E}
\newcommand{\swE}{\tilde E^>}
\newcommand{\rGamma}{\tilde\Gamma}
\newcommand{\rpi}{\tilde\pi}

% gray product

\newcommand{\homlax}{\underline{\fun{Hom}}_{\mathrm{lax}}}
\newcommand{\homcolax}{\underline{\fun{Hom}}_{\mathrm{colax}}}

% model structures

\DeclareMathOperator{\rlp}{r}
\DeclareMathOperator{\llp}{l}

% string

\newcommand{\s}{\mathsf{t}}

% adjunction

\newcommand{\unit}{\eta}
\newcommand{\counit}{\epsilon}

\makeatletter
\newcommand{\oset}[3][0ex]{%
  \mathrel{\mathop{#3}\limits^{
    \vbox to#1{\kern-1.5\ex@
    \hbox{$\scriptstyle#2$}\vss}}}}
\makeatother
\newcommand\qeq{\oset{?}{=}}

% other

\newcommand*\cocolon{%
        \nobreak
        \mskip6mu plus1mu
        \mathpunct{}%
        \nonscript
        \mkern-\thinmuskip
        {:}%
        \mskip2mu
        \relax
}



% notes
\newcommand{\cccom}[1]{\todo[inline, color=magenta!30, linecolor=magenta!30!black, size=\normalsize]{#1}}
\newcommand{\ccnote}[1]{\todo[noline, color=magenta!30, linecolor=magenta!30!black, size=\tiny]{#1}}

\newtheoremstyle{ittheorem}
  {\topsep}   % ABOVESPACE
  {\topsep}   % BELOWSPACE
  {\itshape}  % BODYFONT
  {0pt}       % INDENT (empty value is the same as 0pt)
  {\bfseries} % HEADFONT
  { ---}         % HEADPUNCT
  {5pt plus 1pt minus 1pt} % HEADSPACE
  {}          % CUSTOM-HEAD-SPEC

\newtheoremstyle{itdfn}
  {\topsep}
  {\topsep}
  {}
  {0pt}
  {\bfseries}
  {}
  {5pt plus 1pt minus 1pt}
  {\thmnumber{#2}{\thmnote{\normalfont\ \ %
{\sffamily(#3)}.}}}

\newtheoremstyle{itrmk}
  {0.5\topsep}
  {0.5\topsep}
  {\normalfont}
  {0pt}
  {\sffamily \itshape}
  { --- }
  {5pt plus 1pt minus 1pt}
  {}
  
% \makeatletter
%   \renewcommand\@upn{\textit}
% \makeatother

    
\theoremstyle{ittheorem}
\newtheorem{thm}{Theorem}[section]
\newtheorem*{thm*}{Theorem}
\newtheorem{prop}[thm]{Proposition}
\newtheorem*{prop*}{Proposition}
\newtheorem{cor}[thm]{Corollary}
\newtheorem{lem}[thm]{Lemma}
\newtheorem{conj}[thm]{Conjecture}
\newtheorem*{conj*}{Conjecture}
\theoremstyle{itdfn}
\newtheorem{dfn}[thm]{}
\theoremstyle{itrmk}
\newtheorem{rmk}[thm]{Remark}
\newtheorem{comm}[thm]{Comment}
\newtheorem{exm}[thm]{Example}


\setlength\parindent{1em}

\relpenalty=10000
\binoppenalty=10000

\setlist{leftmargin=20pt,itemsep=0pt,parsep=0pt,topsep=1ex}


\titleformat{\section}
 {\Large\bfseries}{\thesection}{1em}{}

\titleformat{\subsection}
 {\normalsize\bfseries}{\thesubsection.}{1em}{}



\showboxbreadth=50 
\showboxdepth=50


\newcommand\runtitle{stricter $\omega$-categories}
\newcommand\runauthor{chanavat}

\title{Homotopy theory of stricter $\omega$-categories}

\author{Cl\'emence Chanavat}

\institution{Tallinn University of Technology}

\begin{document}

\maketitle
\begin{center}
	\begin{minipage}[t]{.95\textwidth}
		\small\textsc{Abstract.}
		We make strict \( n \)\nbd categories even stricter by requiring they satisfy higher exchange laws governed by Hadzihasanovic's theory of regular directed complexes. 
		We study the first properties of stricter \( n \)\nbd categories, in particular, we define the Gray product, and prove that the definition is stable under suspension, which is non-trivial. 
		After reviewing and briefly expanding the theory diagrammatic sets and their model structure for diagrammatic \( (\infty, n) \)\nbd categories, we construct a folk model structure on stricter \( n \)\nbd category, show that the walking equivalence in stricter \( n \)\nbd categories coincides with the stricter polygraph generated by the walking equivalence in diagrammatic sets, and finally, that the folk model structure on stricter \( n \)\nbd categories is right transferred from the diagrammatic model structure along a diagrammatic nerve construction.
	\end{minipage}
	
	\vspace{20pt}

	\begin{minipage}[t]{0.95\textwidth}
		\setcounter{tocdepth}{2}
		\tableofcontents
	\end{minipage}
\end{center}

\makeaftertitle


\section*{Introduction}

\cccom{TODO}
\begin{itemize}
    \item review combinatorics of regular directed complexes, 
    \item functor \( \molecin{U} \)
    \item suspension, gray product of regular directed complexes
    \item dual
\end{itemize}

\section{Stricter \texorpdfstring{$n$}{n}-categories}

\subsection{Definitions and properties}

\begin{dfn} [Reflexive \( \omega \)\nbd graph]
    A \emph{reflexive \( \omega \)\nbd graph} is a set, whose element are called the \emph{globular cells}, \( C \) together with, for each \( k \geq 0 \), operators
    \begin{equation*}
        \bd{k}{-}, \bd{k}{+} \colon C \to C,
    \end{equation*}
    called the \emph{input and output \( k \)\nbd boundary}, respectively, satisfying the following axioms.
    \begin{enumerate}
        \item for all \( c \in C \), there exists \( k \geq 0 \) such that \( \bd{k}{-} c = c = \bd{k}{+} c \); the \emph{dimension} of \( c \), written \( \dim c \), is the minimum of all such values of \( k \);
        \item for all \( c \in C \), all \( k, n \geq 0 \) and all \( \a, \beta \in \set{-, +} \),
        \begin{equation*}
            \bd{k}{\a}(\bd{n}{\beta} c) = 
            \begin{cases}
                \bd{k}{\a} c & k < n,
                \bd{n}{\beta}, k \ge n.
            \end{cases}
        \end{equation*}
    \end{enumerate}
    A \emph{morphism of reflexive \( \omega \)\nbd graph} is a function of the underlying set commuting with the boundary operators.
\end{dfn}

\noindent If \( C \) is a reflexive \( \omega \)\nbd graph, the set of \emph{\( k \)\nbd composable pairs of globular cells} is the set 
\begin{equation*}
    C \times_k C \eqdef \set{(c, d) \in C \times C \mid \bd{k}{+} c = \bd{k}{-} d}.
\end{equation*}
Given a globular cell \( c \) and \( \a \in \set{-, +} \), we write \( \bd{}{\a} c \) in place of \( \bd{\dim c - 1}{\a} c \), and \( c \colon u^- \gcelto u^+ \) to signify that \( \bd{}{\a} c = u^\a \), and call \( u^- \gcelto u^+ \) the \emph{type of \( c \)}. 
Two globular cells of the same type are said to be \emph{parallel}.
We say that a globular cell \( c \) is an \emph{object} if \( \dim c = 0 \). 

\begin{dfn} [Composition structure]
    A composition structure is a reflexive \( \omega \)\nbd graph \( C \) together with, for all \( k \geq 0 \), an operation
    \begin{equation*}
        - \comp{k} - \colon C \times_k C \to C,
    \end{equation*}
    call the \emph{\( k \)\nbd composition}.
    If \( C, D \) are composition structures, a \emph{strict functor} \( f \colon C \to D \) is a morphism of the underlying reflexive graph respecting the \( k \)\nbd composition.
    We denote \( \Comp \) the category of composition structures and strict functors.
\end{dfn}

\begin{dfn} [Basis for composition structure]
    Let \( C \) be a composition structure, and \( \cls{S} \) be a subset of the globular cells of \( C \).
    We say that \( \cls{S} \) is a \emph{generating set for \( C \)} if the closure of \( \cls{S} \) under \( - \comp{k} - \) is equal to \( C \).
    We say that a generating set is a \emph{basis for \( C \)} if for any other generating set \( \cls{T} \subseteq \cls{S} \), then \( \cls{T} = \cls{S} \).
\end{dfn}

\begin{lem}\label{lem:strict_functor_determined_by_basis}
    Let \( f, g \colon C \to D \) be strict functors and let \( \cls{S} \) be a generating set for \( C \) such that for all \( c \in \cls{C} \), \( f(c) = g(c) \).
    Then \( f = g \).
\end{lem}
\begin{proof}
    See \cite[Lemma 5.1.23]{hadzihasanovic2024combinatorics}.
\end{proof}

\noindent Recall\ccnote{cannot use Nd here} from \cite[Section 5.2]{hadzihasanovic2024combinatorics} that given a regular directed complex \( P \), the set \( \molecin{P} \eqdef \Nd(P) \), together with the boundary operators \( \bd{k}{\a} \) and the pasting operations \( - \cp{k} - \) is a composition structure with basis \( \atomin{P} \eqdef \nd(P) \).
Furthermore, by \cite[Theorem 6.2.32]{hadzihasanovic2024combinatorics}, this assignment extends to a functor
\begin{equation*}
    \molecin{-} \colon \rdcpx \to \Comp.
\end{equation*}
We recall as well that \( \molecin{-} \) is also functorial with respect to subdivisions \( s \colon P \sd Q \) of regular directed complexes.

\begin{dfn} [Globe]
    Let \( n \geq 0 \).
    The \emph{\( n \)\nbd globe} is the composition structure defined by \( \globe{n} \eqdef \molecin{\dglobe{n}} \), and its \emph{boundary} is the composition structure given by \( \bd{}{}\globe{n} \eqdef \molecin{\bd{}{}\dglobe{n}} \).
\end{dfn}

\begin{rmk}
    As usual, a strict functor \( u \colon O^n \to C \) classifies a globular cell of \( C \), and a strict functor \( \bd{}{} O^n \to C \) classifies two parallel cells \( u, v \colon O^{n - 1} \to C \).
\end{rmk}

\begin{dfn} [Diagram in a composition structure]
    Let \( C \) be a composition structure and \( P \) be a regular directed complex.
    A \emph{diagram of shape \( P \) in \( C \)} is a strict functor \( \F \colon \molecin{P} \to C \).
    If \( P = U \) is a molecule, we speak of \emph{pasting diagram}, and in that case, for \( k \geq 0 \) and \( \a \in \set{-, +} \), we write \( \bd{k}{\a} F \) for the restriction of \( \F \) along the strict functor \( \molecin{\bd{k}{\a} U} \to \molecin{U} \).
    We write \( \Dgm(C) \) for the collection of all pasting diagrams of \( C \).
    % Finally, if \( U \) is an atom, we speak of \emph{atomic diagram} in \( C \), and write \( \dgm(C) \) for the collection of all atomic diagram of \( C \).
\end{dfn}

\begin{dfn} [Principal cell]
    Let \( \F \colon \molecin{U} \to C \) be a pasting diagram.
    The \emph{principal cell of \( \F \)} is the cell \( \pcell{\F} \eqdef \F(\idd{U}) \).
\end{dfn}

% \noindent For each pair of molecules \( U, V \) and \( k \geq 0 \) such that \( U \cp{k} V \) is defined, there is a canonical strict functor \( s^k_{UV} \)
% \begin{center}
%     \begin{tikzcd}
%         {\molecin{\bd{k}{+}U}} & {\molecin{V}} \\
%         {\molecin{U}} & {\molecin{U}\cup\molecin{V}} \\
%         && {\molecin{U \cp{k} V}}.
%         \arrow[""{name=0, anchor=center, inner sep=0}, from=1-1, to=1-2]
%         \arrow[from=1-1, to=2-1]
%         \arrow[from=1-2, to=2-2]
%         \arrow[curve={height=-18pt}, from=1-2, to=3-3]
%         \arrow[from=2-1, to=2-2]
%         \arrow[curve={height=12pt}, from=2-1, to=3-3]
%         \arrow["{s^k_{UV}}"{description}, from=2-2, to=3-3]
%         \arrow["\lrcorner"{anchor=center, pos=0.125, rotate=180}, draw=none, from=2-2, to=0]
%     \end{tikzcd}
% \end{center}
\noindent For each regular directed complex \( P \), there is a canonical strict functor
\begin{equation*}
    s_P \colon \colim_{x \in P} \molecin{\imel{P}{x}} \to \molecin{P}
\end{equation*}
We let \( S \) be the set of all the strict functors \( s_U \) for \( U \) a molecule.

\begin{dfn}[Stricter \( \omega \)\nbd category]
    A \emph{stricter \( \omega \)\nbd category} is a composition structure \( C \) which is local with respect to \( S \).
    We let \( \somegaCat \) be the full subcategory of \( \omegaCat \) on stricter \( \omega \)\nbd categories.
\end{dfn}

\noindent Since \( \Comp \) is locally presentable, as a category of models of a limit sketch, and \( S \) is a small set, the full subcategory inclusion \( \iota \colon \somegaCat \incl \Comp \) is reflective \cite{freyd1972continuous}, and we denote by 
\begin{equation*}
    \rc \colon \Comp \to \somegaCat
\end{equation*}
the left adjoint of \( \iota \).

\begin{dfn} [Matching family and amalgamation]
    Let \( P \) be a regular directed complex, and \( C \) be a composition structures.
    A \( P \)\nbd matching family in \( C \) is a cone 
    \begin{equation*}
        \set{\F_x \colon \molecin{\imel{P}{x}} \to C}_{x \in P}
    \end{equation*}
    under the \( P \)\nbd shaped diagram \( x \mapsto \molecin{\imel{P}{x}} \).    
    An \emph{amalgamation} of this matching family is a strict functor 
    \begin{equation*}
        \amalg_{x\in P} \F_x \colon \molecin{P} \to C
    \end{equation*}
    such that for all \( x \in P \), \( (\amalg_y \F_y) \after \molecin{\mapel{x}} = \F_x \).
\end{dfn}

\begin{rmk}\label{rmk:data_matching family}
    The data of a \( P \)\nbd matching family 
    \begin{equation*}
        \set{\F_x \colon \molecin{\imel{P}{x}} \to C}_{x \in P}
    \end{equation*}
    in \( C \) is given by an element \( c_x \in C \) for each \( x \in P \).
    Indeed, given a matching family \( \set{\F_x}_{x \in P} \), define \( c_x \eqdef \pcell{\F_x} \).
    By functoriality, if \( x \le y \), then \( c_x = \F_y(\clset{x} \incl \clset{y}) \), thus by Lemma \ref{lem:strict_functor_determined_by_basis}, this data entirely determines the matching family \( \set{\F_x}_{x \in P} \).
    Of course, not all data of this type gives rise to a matching family. 
\end{rmk}

\begin{rmk}
    Thus, a composition structure \( C \) is a stricter \( \omega \)\nbd category if for all molecules \( U \), all \( U \)\nbd matching families in \( C \) have a unique amalgamation.
\end{rmk}

\begin{lem}\label{lem:at_most_one_lift}
    Let \( C \) be a composition structure, \( P \) be a regular directed complex, and \( \set{\F_x}_{x \in P} \) be a matching family. 
    Then \( \set{\F_x}_{x \in P} \) has at most one amalgamation.
\end{lem}
\begin{proof}
    Immediate by Lemma \ref{lem:strict_functor_determined_by_basis}.
\end{proof}

\begin{comm} \label{comm:well_defined_amalgamation}
    Given a \( P \)\nbd matching family \( \set{\F_x} \) in \( C \), we thus have a candidate amalgamation \( \F \colon \molecin{P} \to C \) defined by sending an element \( \mapel{x} \in \atomin{P} \) to \( \pcell{\F_x} \).
    Then, using induction on submolecules (a variant of \cite[Comment 4.1.7]{hadzihasanovic2024combinatorics}), to show that \( \F \) is a well defined strict functor, we may take an arbitrary element \( w \colon W \to P \) in \( \molecin{P} \), and prove that \( \F \after \molecin{w} \) is well defined under the hypothesis that for all proper subdiagrams \( w' \) of \( w \), \( \F \after \molecin{w'} \) is well defined. 
    The base case on subdiagrams \( w' \) of \( w \) of dimension \( 0 \) is always true in that case.
    Here, ``well-defined'' means that the value of \( \F(w) \) is independent of the chosen decomposition \( w = w_1 \cp{k} w_2 \).
\end{comm}

\begin{lem} \label{lem:stricter_iff_local_wrt_pasting}
    Let \( C \) be composition structure.
    The following are equivalent.
    \begin{enumerate}
        \item \( C \) is a stricter \( \omega \)\nbd category;
        \item for all regular directed complexes \( P \), \( C \) is local with respect to \( s_P \);
        \item for all pairs of molecules \( U, V \) and \( k \geq 0 \) such that \( U \cp{k} V \) is defined, each lifting problem
        \begin{center}
            \begin{tikzcd}
                {\molecin{U} \cup \molecin{V}} & C \\
                {\molecin{(U \cp{k} V)}}
                \arrow[from=1-1, to=1-2]
                \arrow[from=1-1, to=2-1]
            \end{tikzcd}
        \end{center}
        has a (necessarily unique) solution.
    \end{enumerate}
\end{lem}
\begin{proof}
    Suppose that \( C \) is stricter, consider a regular directed complex \( P \) and a \( P \)\nbd matching family \( \set{\F_x \colon \molecin{\imel{P}{x}} \to C}_{x \in P} \).
    Then for all element \( w \colon W \to P \), \( \set{\F_x}_{x \in W} \) defines \( W \)\nbd matching family in \( C \), whose amalgamation is \( \F \after \molecin{w} \).
    This shows that \( \amalg \F_x \) is a well defined strict functor.
    Conversely, if \( C \) is local with respect to all the strict functors \( s_P \) where \( P \) is a regular directed complex, then it is also the case for all the functor \( s_U \) where \( U \) is a molecule.
    This shows the first two conditions are equivalent.
    Finally, the last condition is clearly necessary, since any functor \( \molecin{U} \cup \molecin{V} \to C \) defines in particular a \( (U \cp{k} V) \)\nbd matching family in \( C \).
    Conversely, we show it is sufficient.
    Let \( P \) be a regular directed complex.
    We show that \( C \) is local with respect to \( s_p \).
    Let \( \set{\F_x} \) be a \( P \)\nbd matching family in \( C \).
    We show that the candidate amalgamation \( \F \colon \molecin{P} \to C \) is a strict functor as per Comment \ref{comm:well_defined_amalgamation}.
    Let \( w \colon W \to P \) in \( \molecin{P} \), and suppose that \( \F \after \molecin{w'} \) is well defined for all proper submolecules \( w' \) of \( w \).
    Then either \( w \) is in \( \atomin{P} \), in which cases we are done since \( \F \after \molecin{w} = \F_x \) for some \( x \in P \), or \( w = w_1 \cp{k} w_2 \) for some decomposition \( W = W_1 \cp{k} W_2 \).
    Then, by inductive hypothesis, we have a strict functor \( (\F \after \molecin{w_1}, \F \after \molecin{w_2}) \colon \molecin{W_1} \cup \molecin{W_2} \to C \).
    By hypothesis, this extends to a strict functor \( \F' \colon \molecin{(W_1 \cp{k} W_2)} \to C \), which is equal to \( \F \after \molecin{w} \) by Lemma \ref{lem:strict_functor_determined_by_basis}.
    This shows that \( \F \after \molecin{w} \) is well defined and concludes the proof.
\end{proof}

\begin{prop} \label{prop:regular_directed_complex_stricter}
    Let \( P \) be a regular directed complex.
    Then \( \molecin{P} \) is a stricter \( \omega \)\nbd category.
\end{prop}
\begin{proof}
    Let \( Q \) be a regular directed complex, and consider a \( Q \)\nbd matching family \( \set{\F_x \colon \molecin{\imel{Q}{x}} \to \molecin{P}} \) in \( \molecin{P} \).
    We want to show that the candidate amalgamation \( \F \colon \molecin{Q} \to \molecin{P} \) is well defined, but for each \( w \colon W \to Q \) in \( \F(w) \) is given by the canonical morphism \( \colim_{x \in W} \pcell{\F_{w(x)}} \to P \), which is independent of the chosen decomposition of \( w \).
    This concludes the proof.
\end{proof}

\noindent Therefore, the functor \( \molecin{-} \colon \rdcpx \to \Comp \) factors through the subcategory \( \somegaCat \).

\begin{cor} \label{cor:regular_directed_complex_colimit_of_itself}
    Let \( P \) be a regular directed complex.
    Then in \( \somegaCat \),
    \begin{equation*}
        s_p \colon \colim_{x \in P} \molecin{\imel{P}{x}} \cong \molecin{P}.
    \end{equation*}
\end{cor}
\begin{proof}
    By the Yoneda Lemma and Lemma \ref{lem:stricter_iff_local_wrt_pasting}, \( \rc(s_P) \) is an isomorphisms in \( \somegaCat \).
    Since \( \rc \) is left adjoint, we conclude by Proposition \ref{prop:regular_directed_complex_stricter}.
\end{proof}

\begin{cor} \label{cor:molecin_preserves_pushout_inclusions}
    The functor \( \molecin{-} \colon \rdcpx \to \somegaCat \) preserves all pushouts of inclusions. 
\end{cor}

\begin{dfn} [Pasting in a stricter \( \omega \)\nbd category]
    Let \( C \) be a stricter \( \omega \)\nbd category, consider two pasting diagrams \( \F \colon \molecin{U} \to C \), \( \G \colon \molecin{V} \to C \), and \( k \geq 0 \) such that \( \bd{k}{+} \F = \bd{k}{-} \G \).
    Then we write \( \F \cp{k} \G \colon \molecin{(U \cp{k} V)} \to C \) for the strict functor determined by the universal properties of pushout given by Corollary \ref{cor:molecin_preserves_pushout_inclusions}.
    More generally, if a generalised pasting \( U \gencp{k} V \) given by a span \( (i \colon U \cap V \incl U, j \colon U \cap V \incl V) \) is defined and such that \( \F \after \molecin{i} = \G \after \molecin{j} \), we write \( \F \gencp{k} \G \) for the strict functor determined by universal property of the pushout.
    If the generalised pasting is given by a pasting at a submolecules \( U \cpsub{\iota} V \) or \( V \subcp{\iota} V \), we write accordingly \( \F \cpsub{\iota} \G \) and \( \G \subcp{\iota} \F \).
\end{dfn}

\begin{dfn} [Stricter complex]
    We say that a stricter \( \omega \)\nbd category is a \emph{stricter regular complex} if it is of the form \( \molecin{P} \) for some regular directed complex \( P \), and let \( \omegaReg \) the full subcategory of \( \somegaCat \) on stricter regular complexes, and \( \omegaMol \) its full subcategory on stricter regular complexes \( \molecin{U} \) where \( U \) is a molecule.
\end{dfn}

\begin{dfn}
    Let \( X \colon \opp{\omegaMol} \to \Set \) be an \( S \)\nbd local presheaf.\ccnote{define local presheaf}
    The \emph{stricter \( \omega \)\nbd category associated to \( X \)} is the composition structure 
    \begin{equation*}
        C \eqdef \coprod_{n \geq 0} X(\dglobe{n}),
    \end{equation*}
    whose boundary operators are induced by the strict functor
    \begin{equation*}
        \bd{k}{\a} \colon \molecin{\dglobe{k}} \to \molecin{\dglobe{n}}, 
    \end{equation*}
    and \( k \)\nbd composition operation is induced by the strict functor
    \begin{equation*}
        \molec{\dglobe{n}} \to \molecin{(\dglobe{n} \cp{k} \dglobe{n})}.
    \end{equation*}
\end{dfn}

\begin{rmk}\label{rmk:local_presheaf_defines_stricter}
    The composition structure \( C \) is indeed a stricter \( \omega \)\nbd category, by definition of the locality with respect to \( S \).
    Furthermore, any natural transformation \( f \colon X \to Y \) between \( S \)\nbd local presheaves induces a strict functor between the associated stricter \( \omega \)\nbd categories.
    This defines a functor
    \begin{equation*}
        [\opp{\omegaMol}, \Set] \to \somegaCat.
    \end{equation*}
\end{rmk}

\begin{prop} \label{prop:stricter_cat_are_local_presheaves}
    The category of \( S \)\nbd local presheaves over \( \omegaMol \) is equivalent to the category of stricter \( \omega \)\nbd categories.
\end{prop}
\begin{proof}
    We construct an inverse up to natural isomorphism to the functor of Remark \ref{rmk:local_presheaf_defines_stricter}.
    Let \( C \) be a stricter \( \omega \)\nbd category.
    Then the presheaf defined by \(\molecin{U} \mapsto \somegaCat(\molecin{U}, C) \) is \( S \)\nbd local by definition. 
    One checks directly that this functor is the desired inverse up to natural isomorphism.
\end{proof}

\begin{cor} \label{cor:diagrams_are_dense} 
    Let \( C \) be a stricter \( \omega \)\nbd category.
    Then the canonical strict functor
    \begin{equation*}
        \phi \colon \colim_{u \in \Dgm(C)} \molecin{U} \to C,
    \end{equation*}
    is an isomorphism.
    That is, the category \( \omegaMol \) is dense in \( \somegaCat \).
\end{cor}
\begin{proof}
    Follows from the density formula for presheaves and Proposition \ref{prop:stricter_cat_are_local_presheaves}.
\end{proof}

\begin{dfn} 
    Let \( C \) be a composition structure, and \( a, b \) be \( 0 \)\nbd dimensional globular cells.
    We define the composition structure
    \begin{equation*}
        C(a, b) \eqdef \set{u \in C \mid \bd{0}{-} u = a, \bd{0}{+} u = b},    
    \end{equation*}
    whose boundary operators and \( k \)\nbd composition are induced by the one of \( C \) shifted by \( 1 \).
\end{dfn}

\begin{lem} \label{lem:hom_of_stricter_is_stricter}
    Let \( C \) be a stricter \( \omega \)\nbd category, and \( a, b \) be two \( 0 \)\nbd dimensional globular cells of \( C \).
    Then \( C(a, b) \) is a stricter \( \omega \)\nbd category.
\end{lem}
\begin{proof}
    Let \( U \) be a molecule.
    Any \( U \)\nbd matching family 
    \begin{equation*}
        \set{\F_x \colon \molecin{\imel{U}{x}} \to \C(a, b)}_{x \in U}
    \end{equation*}
    determines a \( \sus{U} \)\nbd matching family 
    by letting \( \F_{\sus{x}} \eqdef \F_x \) for \( x \in U \), and \( \F_{\bot^-}, \F_{\bot^+} \) be the strict functors from \( \molecin{\pt} \) classifying \( a \) and \( b \) respectively. 
    This matching family has an amalgamation \( \sus{\F} \colon \molecin{\sus{U}} \to C \), showing that the amalgamation of \( \set{\F_x}_{x \in U} \) is well defined.
\end{proof}

\begin{comm}
    Thus, any stricter \( \omega \)\nbd category can canonically be seen as a category enriched in stricter \( \omega \)\nbd categories.
    The converse is not true. \ccnote{add a ref here to the counter example of a 4-stricter which is not strict, but is strict enriched in stricter}
    We do not know any condition for a category enriched in a stricter \( \omega \)\nbd category to be itself a stricter \( \omega \)\nbd category.
    We know however from Proposition \ref{prop:stricter_are_strict} that it is at least a strict \( \omega \)\nbd category.
\end{comm}

\begin{dfn} [Stricter \( n \)\nbd category]
    Let \( n \geq 0 \).
    A \emph{\( n \)\nbd composition structure} is a composition structure \( C \) such that for all globular cells \( c \in C \), we have \( \dim c \le n \).
    If \( C \) was a stricter \( \omega \)\nbd category, we speak of \emph{stricter \( n \)\nbd category}.
    We denote by \( \nComp{n} \) and \( \snCat{n} \) the full subcategories of \( \Comp \) and \( \somegaCat \) on \( n \)\nbd composition structures and stricter \( n \)\nbd categories, respectively. 
\end{dfn}

\begin{dfn} 
    The inclusion \( \iota_n \colon \nComp{n} \incl \Comp \) has a right adjoint \( \skel{n} \) defined by
    \begin{equation*}
        \skel{n}(C) \eqdef \set{c \in \C \mid \dim c \le n},
    \end{equation*}
    and a left adjoint \( \trunc{n} \) defined by
    \begin{equation*}
        \trunc{n}(C) \eqdef \skel{n - 1}(C) \cup \set{[c] \mid c \in C, \dim c = n},
    \end{equation*}
    where \( [-] \) denote the equivalence class on the globular cells of \( C \) of dimension \( n \) generated by \( \bd{}{-} d \sim \bd{}{+} d \) for all globular cells \( d \) of dimension \( n + 1 \). 
    By convention, \( \skel{-1}(C) = \emptyset \).
\end{dfn}

\begin{rmk}
    By \cite[Proposition 5.2.14]{hadzihasanovic2024combinatorics}, if \( P \) is a regular directed complex, \( \skel{n} \molecin{P} \) is naturally isomorphic to \( \molecin{(\skel{n}P)} \).
\end{rmk}

\begin{lem} \label{lem:stricter_n_iff_local_with_dim_le_n}
    Let \( C \) be an \( n \)\nbd composition structure.
    The following are equivalent.
    \begin{enumerate}
        \item \( C \) is a stricter \( n \)\nbd category;
        \item for all regular directed complex with \( \dim P \le n \), \( C \) is local with respect to \( s_P \);
        \item for all pairs of molecules \( U, V \) with \( \dim U, \dim V \le n \) and \( k \geq 0 \) such that \( U \cp{k} V \) is defined, each lifting problem
            \begin{center}
                \begin{tikzcd}
                    {\molecin{U} \cup \molecin{V}} & C \\
                    {\molecin{(U \cp{k} V)}}
                    \arrow[from=1-1, to=1-2]
                    \arrow[from=1-1, to=2-1]
                \end{tikzcd}
            \end{center}
            has a (necessarily unique) solution.
    \end{enumerate}
\end{lem}
\begin{proof}
    Suppose \( C \) is a stricter \( n \)\nbd category, then by Lemma \ref{lem:stricter_iff_local_wrt_pasting}, the last two conditions holds.
    Now suppose the second condition holds, consider a regular directed complex \( P \) and a \( P \)\nbd matching family \( \set{\F_x \colon \molecin{\imel{P}{x}} \to C} \) with candidate amalgamation \( \F \).
    Restricting this matching family to \( x \in \gr{\le n}{P} \) and using the assumption, we have a an amalgamation 
    \begin{equation*}
        \gr{\le n}{\F} \colon \amalg_{x \in \gr{\le n}{P}} \F_x \colon \gr{\le n}{P} \to C.
    \end{equation*}
    Let \( x \in P \) with \( \dim x > n \).
    Then \( \dim \pcell{\F_x} \le n \), hence for any \( \a \in \set{-, +} \), \( \pcell{\bd{n}{\a} \F_x} = \pcell{\F(\bd{n}{\a} x \to P)} \).
    Using this fact, an induction on the submolecules of any \( w \colon W \to P \) shows that \( \F(w) = \gr{\le n}{\F}(\bd{n}{\a} w) \), proving that \( \F \) is well defined.
    This shows that \( C \) is stricter.
    For the last condition, one reason as in the proof of Lemma \ref{lem:stricter_iff_local_wrt_pasting}.
\end{proof}

\begin{lem} \label{lem:truncation_stricter_are_stricter}
    Let \( n \geq 0 \), and \( C \) be a stricter \( \omega \)\nbd category.
    Then \( \skel{n}(C) \) and \( \trunc{n}(C) \) are stricter \( n \)\nbd categories.
\end{lem}
\begin{proof}
    We use the second point of Lemma \ref{lem:stricter_n_iff_local_with_dim_le_n}.
    Let \( P \) be a regular directed complex with \( \dim P \le n \), and \( \set{\F_x \colon \molecin{\imel{P}{x}} \to \skel{n}(C)}_{x \in P} \) be a \( P \)\nbd matching family in \( \skel{n}(C) \).
    Then post-composing each \( \F_x \) by the unit \( \skel{n}(C) \to C \), and using the fact that \( C \) is stricter, the amalgamation defines a strict functor \( \F \colon \molecin{P} \to C \).
    Then, since \( \skel{n}(P) = P \), \( \skel{n}(\F) \) is the desired amalgamation of the matching family.

    Now consider a matching family \( \set{\G_x \colon \molecin{\imel{P}{x}} \to \trunc{n}(C)} \).
    By definition of \( \trunc{n}(C) \), and since \( \skel{n - 1}(C) \subseteq \trunc{n}(C) \) is stricter, we have the amalgamation
    \begin{equation*}
        \gr{< n }{\G} \eqdef \amalg_{x \in \gr{< n}{P}} \G_x \colon \molecin{\gr{< n}{P}} \to \trunc{n}(C).
    \end{equation*}
    For each \( x \in P \) of dimension \( n \) such that, choose a representative \( u_x \colon W_x \to C \) of the cell \( \pcell{\G_x} \) in \( \trunc{n}(C) \) (if \( \dim \pcell{\G_x} \le n \), then we mean that we let \( u_x \eqdef \pcell{\G_x} \)).
    We claim that \( \set{u_x}_{x \in P} \) give rise to a matching family \( \set{\G'_x \colon \molecin{\imel{P}{x}} \to C}_{x \in P} \), as per Remark \ref{rmk:data_matching family}.
    This is clear for all \( x \in \gr{< n}{P} \), since this is the data associated to the matching family \( \set{\G_x}_{x \in \gr{< n}{P}} \). 
    Then, if \( \dim x = n \), then \( \G'_x \colon \molecin{\imel{P}{x}} \to C \) is a well defined strict functor.
    Indeed, \( \pcell{\G'_x} = u_x \), and for all \( k < n \) and \( \a \in \set{-, +} \) we have \( \bd{k}{\a} u_x = \gr{< n}{G}(\bd{k}{\a} x \incl P ) \).
    since \( \bd{k}{\a} \pcell{\G_x} \) is independent of the chosen representative of \( \pcell{\G_x} \).
    Since \( C \) is stricter, this defines a strict functor \( \G' \colon \molecin{P} \to C \). 
    Post-composing \( \G' \) with the counit \( \varepsilon_C \colon C \to \trunc{n}(C) \), we obtain the strict functor \( \varepsilon_C \after \G' \), which is the candidate amalgamation \( \G \) by Lemma \ref{lem:strict_functor_determined_by_basis}.
    This shows that \( \trunc{n}(C) \) is stricter, and concludes the proof.
\end{proof}

\begin{dfn}[\( n \)\nbd skeletong and \( n \)\nbd truncation.]
    Let \( n \geq 0 \) and \( C \) be a stricter \( n \)\nbd category.
    The \emph{\( n \)\nbd skeleton} of \( C \) is the stricter \( n \)\nbd category \( \skel{n}(C) \).
    The \emph{\( n \)\nbd truncation} of \( C \) is the stricter \( n \)\nbd category \( \trunc{n}(C) \).
\end{dfn}

\noindent Thus, the adjoint triple \( \trunc{n} \dashv \iota_n \dashv \skel{n} \) restricts the adjoint triple
\begin{center}
    \begin{tikzcd}
        {\snCat{n}} && \somegaCat.
        \arrow["{\iota_n}"{description}, from=1-1, to=1-3]
        \arrow["{\skel{n}}", shift left=2, curve={height=-12pt}, from=1-3, to=1-1]
        \arrow["{\trunc{n}}"', shift right=2, curve={height=12pt}, from=1-3, to=1-1]
    \end{tikzcd}
\end{center}
Notice that given an stricter \( \omega \)\nbd category \( C \), we have a chain of inclusions 
\begin{equation*}
    \skel{-1} C \incl \skel{0} C \incl \skel{1} C \incl \ldots \incl \skel{n} C \incl \ldots
\end{equation*}
whose colimit \( \somegaCat \) is \( C \).


\subsection{Gray product of stricter \texorpdfstring{$\omega$}{}-categories}

Recall that if \( P, Q \) are regular directed complexes, the basis \( \atomin{(P \gray Q)} \) of the stricter regular complex \( \molecin{P \gray Q} \) is given exactly by the morphisms \( u \gray v \) for \( u \in \atomin{P} \) and \( v \in \atomin{Q} \).  

\begin{dfn} [Gray product stricter complexes] \label{dfn:gray_product_of_stricter_regular_complexes}
    Let \( P, Q \) be regular directed complexes.
    The \emph{Gray product} of \( \molecin{P} \) and \( \molecin{Q} \) is the stricter regular complex
    \begin{equation*}
        \molecin{P} \gray \molecin{Q} \eqdef \molecin{(P \gray Q)}.
    \end{equation*}
    If \( \F \colon \molecin{P} \to \molecin{P'} \) and \( \G \colon \molecin{Q} \to \molecin{P'} \) are two strict functors, we let 
    \begin{equation*}
        \F \gray \G \colon \molecin{P} \gray \molecin{Q} \to  \molecin{P'} \gray \molecin{Q'}
    \end{equation*}
    by sending a basis element \( u \gray v \) to \( \F(u) \gray \G(v) \).
\end{dfn}

\begin{prop} \label{prop:gray_stricter_regular_complex_monoidal}
    The Gray product determines a monoidal structure on the category \( \omegaReg \), whose monoidal unit is the terminal stricter \( \omega \)\nbd category \( \globe{0} \).
\end{prop}
\begin{proof}
    Let \( \F \colon \molecin{P} \to \molecin{P'} \) and \( \G \colon \molecin{Q} \to \molecin{Q'} \) be two strict functors, and \( w \colon W \to P \gray Q \) be a morphism.
    We show that \( \fun{H} \eqdef (\F \gray \G) \after \molecin{w} \) is well defined by induction on \( \dim w \), then on the subdiagrams \( w' \) of \( w \).
    The base cases where \( \dim w = 0 \) and \( w' \) is a point are clear.
    Suppose inductively that \( \dim w > 0 \) and that \( (\F \gray \G) \after \molecin{w'} \) is well defined for all proper subdiagrams \( w' \) of \( w \).
    Consider first the case where that \( W \) is an atom.
    Then by definition, \( \pcell{\F \gray \G} = \pcell{\F} \gray \pcell{\G} \).
    Then, for \( k \geq 0 \) and \( \a \in \set{-, +} \), we have
    \begin{align*}
        \bd{k}{\a} \pcell{\F \gray \G} &= \bd{k}{\a} (\pcell{\F} \gray \pcell{\G}) \\
                                       &= \bigcup_{i = 1}^k \pcell{\bd{i}{\a} \F} \gray \pcell{\bd{k - i}{(-)^i\a} \G} \\
                                       &= \pcell{\bigcup_{i = 1}^k \bd{i}{\a} \F \gray \bd{k - i}{(-)^i\a \G}} \\
                                       &= \pcell{\bd{k}{\a} (\F \gray \G)},
    \end{align*}
    where we used the inductive hypothesis in the case \( k < \dim W \).
    Then, consider an element \( u \in \molecin{W} \) which is not the principal cell. 
    Necessarily \( \dim u < \dim w \), thus
    \begin{align*}
        \bd{k}{\a} \fun{H}(u) &= \bd{k}{\a} \pcell{(\F \gray \G) \after \molecin{(w \after u)}} \\
                                                         &= \pcell{\bd{k}{\a} ((\F \gray \G) \after \molecin{(w \after u)})} \\
                                                         &= \fun{H}(\bd{k}{\a} u).
    \end{align*}
    This shows that \( (\F \gray \G) \after w \) is a morphism of the underlying reflexive globular graph. 
    Now let \( u, u' \in \molecin{W} \) and \( 0 \le k \le \dim u, \dim u' \) such that \( u \cp{k} u' \) is defined.
    Since \( \atomin{W} \) is a basis for \( \molecin{W} \), we get that \( \dim (u \cp{k} u') < n \), and thus by inductive hypothesis, 
    \( (\F \gray \G)  \after \molecin{w \after (u \cp{k} u')} \) is well defined.
    Then,
    \begin{align*}
        ((\F \gray \G)  \after \molecin{w})(u \cp{k} u') &= \pcell{\fun{H} \after \molecin{(u \cp{k} u')}} \\
                                                         &= \pcell{\fun{H} \after \molecin{u}} \cp{k} \pcell{\fun{H} \after \molecin{u'}} \\
                                                         &= \fun{H}(u) \cp{k} \fun{H}(u').
    \end{align*}
    This shows that \( \fun{H} \) is a strict functor.
    Finally, suppose \( w \) is not an atom.
    Given \( x \in W \), we have \( (a, b) \in P \gray Q \) such that \( w(x) = (a, b) \).
    By inductive hypothesis on the subdiagrams, 
    \begin{equation*}
        \fun{H}_x \eqdef (\F \after \molecin{\mapel{a}}) \gray (\G \after \molecin{\mapel{b}}) \colon \molecin{(\imel{P}{a} \gray \imel{Q}{b})} \to \molecin{(P' \gray Q')}
    \end{equation*}
    is a strict functor, and the collection \( \set{\fun{H}_x}_{x \in W} \) is a \( W \)\nbd matching family in \( \molecin{(P' \gray Q')} \), whose amalgamation is \( \fun{H} \), which is therefore a well defined strict functor.
    This shows that \( \F \gray \G \) is a strict functor.
    Since for all regular directed complexes \( P \), \( \pt \gray P = P = P \gray \pt \), we deduce that the monoidal unit is \( \molecin{\pt} = \globe{0} \).
    Functoriality of \( - \gray - \) is straightforwards.
    This concludes the proof.
\end{proof}

\begin{rmk}
    If \( \F \colon \molecin{P} \to \molecin{P'} \) and \( \G \colon \molecin{Q} \to \molecin{P'} \) are strict functors, \( u \in \molecin{P} \), and \( v \in \molecin{Q} \), then 
    \begin{equation*}
       (\F \gray \G)(u \gray v) = \F(u) \gray \G(v). 
    \end{equation*}
\end{rmk}

\begin{rmk}
    Since the point is a molecule, and the Gray product of two molecules is a molecule, the category \( \omegaMol \) inherit from \( \omegaReg \) of the monoidal structure given by the Gray product. 
\end{rmk}

\begin{dfn} 
    Let \( P \) be a regular directed complex and \( C \) be a stricter \( \omega \)\nbd category.
    By \cite[Lemma 7.2.8]{hadzihasanovic2024combinatorics}, the presheaf on \( \omegaMol \) given by
    \begin{equation*}
        \molecin{U} \mapsto \somegaCat(\molecin{(P \gray U)}, C) 
    \end{equation*}
    is \( S \)\nbd local.
    By Proposition \ref{prop:stricter_cat_are_local_presheaves}, this defines the stricter \( \omega \)\nbd category 
    \begin{equation*}
        \homlax(\molecin{P}, C),
    \end{equation*}
    whose \( n \)\nbd cells are strict functors \( \F \colon \molecin{(P \gray \dglobe{n})} \to C \).
    Dually, we define the stricter \( \omega \)\nbd category
    \begin{equation*}
        \homcolax(\molecin{P}, C),
    \end{equation*}
    whose \( n \)\nbd cells are strict functors \( \F \colon \molecin{(\dglobe{n} \gray P)} \to C \).
\end{dfn}

\begin{lem}
    Let \( U, V \) be molecules, \( C \) be a stricter \( \omega \)\nbd category.
    Then there are bijections
    \begin{align*}
        \somegaCat(\molecin{(U \gray V)}, C) &\cong \somegaCat(\molecin{U}, \homcolax(\molecin{V}, C)) \\
                                             &\cong \somegaCat(\molecin{V}, \homlax(\molecin{U}, C)),
    \end{align*}
    natural in \( \molecin{U}, \molecin{V} \) and \( C \).
\end{lem}
\begin{proof}
    Follows directly by Proposition \ref{prop:stricter_cat_are_local_presheaves} and the Yoneda Lemma. 
\end{proof}

Applying \cite[Th\'eor\`eme 5.3]{ara2020joint} together with the previous result, and Corollary \ref{cor:diagrams_are_dense}, we get the following definition.
\begin{dfn} [Gray product of stricter \( \omega \)\nbd categories]
    Let \( C, D \) be stricter \( \omega \)\nbd categories.
    The Gray product of \( C \) and \( D \) is the stricter \( \omega \)\nbd category
    \begin{equation*}
        \colim_{\substack{u \in \Dgm(C),\\ v \in \Dgm(D)}} \molecin{(U \gray V)}.
    \end{equation*}
    This determines a biclosed monoidal structure whose monoidal unit is the terminal stricter \( \omega \)\nbd category \( \globe{0} \) and such that the inclusion
    \begin{equation*}
        \omegaMol \to \somegaCat
    \end{equation*}
    is strong monoidal.
\end{dfn}

\begin{rmk}
    Let \( P, Q \) be regular directed complexes.
    Since \( - \gray - \) is biclosed and using Corollary \ref{cor:regular_directed_complex_colimit_of_itself}, we get
    \begin{align*}
        \molecin{P} \gray \molecin{Q} &\cong \colim_{x \in P, y \in Q} \molecin{\imel{P}{x}} \gray \molecin{\imel{Q}{y}} \\
                                      &\cong \colim_{(x, y) \in P \gray Q} \molecin{(\imel{P}{x} \gray \imel{Q}{y})} \\
                                      &= \molecin{(P \gray Q)}.
    \end{align*}
    Thus the notation is consistent with Definition \ref{dfn:gray_product_of_stricter_regular_complexes}, so that the inclusion
    \begin{equation*}
        \omegaReg \incl \somegaCat
    \end{equation*}
    is also strong monoidal.
\end{rmk}

\subsection{Strict and stricter categories}

The goal of this section is to clarify the relationship between strict and stricter categories.

\begin{prop} \label{prop:stricter_are_strict}
    Let \( C \) be a stricter \( \omega \)\nbd category.
    Then \( C \) is a strict \( \omega \)\nbd category.
\end{prop}
\begin{proof}
    Recall from \cite[Theorem 5.2.5]{hadzihasanovic2024combinatorics} that pasting and boundaries makes the collection of molecules a strict \( \omega \)\nbd categories. 
    We first show the axioms of interaction between pasting and boundaries. 
    Let \( c \colon O^{m} \to C \) and \( d \colon O^{m'} \to C \) be \( k \)\nbd composable cells in \( C \).
    The pasting diagram \( c \cp{k} d \colon \molecin{(O^m \cp{k} O^{m'})} \) is such that \( \pcell{c \cp{k} d} = \pcell{c} \comp{k} \pcell{d} \).
    Let \( n \geq 0 \) and \( \a \in \set{-, +} \), then
    \begin{equation*}
         \bd{n}{\a} (\pcell{c} \comp{k} \pcell{d}) = \bd{n}{\a} \pcell{c \cp{k} d} = \pcell{\bd{n}{\a} (c \cp{k} d)},
    \end{equation*}
    the latter being equal to
    \begin{equation*}
        \begin{cases}
            \pcell{\bd{n}{\a} c} = \bd{n}{\a} \pcell{c} = \bd{n}{\a} \pcell{d} & \text{if } n < k,\\
            \pcell{\bd{k}{-}c} = \bd{k}{-}\pcell c & \text{if } n = k, \a = -,\\
            \pcell{\bd{k}{+}d} = \bd{k}{+}\pcell d & \text{if } n = k, \a = +,\\
            \pcell{\bd{k}{\a}c \cp{k} \bd{k}{\a} d} = \bd{k}{\a}\pcell c \comp{k} \bd{k}{\a} \pcell d & \text{if } n > k.
        \end{cases}
    \end{equation*}
    Next is unitality.
    Let \( c \colon O^m \to C \) be a cell, and \( k \geq 0 \).
    Then we have the pasting diagram \( c \cp{k} \bd{k}{+} c \colon \molecin{(O^m \cp{k} \bd{k}{+} O^m)} \to C \).
    Since \( (O^m \cp{k} \bd{k}{+} O^m) = O^m \), the principal cell of \( c \cp{k} \bd{k}{+} c \) is \( \pcell{c} \).
    This shows that \( \pcell{c} \comp{k} \bd{k}{+} \pcell{c} = \pcell{c} \).
    Similarly, \(  \bd{-}{k} c \comp{k} \pcell{c} = \pcell{c} \).
    Next, we consider the axiom of associativity. 
    Let \( c, d, e \) be \( k \)\nbd composable cell in \( C \).
    Then again, we have a pasting diagram \( c \cp{k} d \colon \molecin{(\dglobe{m} \cp{k} \dglobe{m'})} \to \C \).
    Then we may paste \( c \cp{k} d \) and \( e \) to obtain a pasting diagram 
    \begin{equation*}
        (c \cp{k} d) \cp{k} e \colon \molecin{(\dglobe{m} \cp{k} \dglobe{m'} \cp{k} \dglobe{m''})} \to C,
    \end{equation*}
    whose principal cell is \( (\pcell c \comp{k} \pcell d) \comp{k} \pcell{e} \).
    Similarly, we have a pasting diagram
    \begin{equation*}
        c \cp{k} (d \cp{k} e) \colon \molecin{(\dglobe{m} \cp{k} \dglobe{m'} \cp{k} \dglobe{m''})} \to C,
    \end{equation*}
    whose principal cell is \( \pcell c \comp{k} (\pcell d \comp{k} \pcell{e}) \).
    By Lemma \ref{lem:strict_functor_determined_by_basis}, 
    \begin{equation*}
         (c \cp{k} d) \cp{k} e = c \cp{k} (d \cp{k} e),
    \end{equation*}
    hence their principal cells are equal as well.
    Proceed similarly for exchange.
    This concludes the proof.
\end{proof} 

Akin to stricter \( \omega \)\nbd categories, strict \( \omega \)\nbd categories are a reflective subcategory of \( \Comp \).
Thus by Proposition \ref{prop:stricter_are_strict}, we have a sequence of full subcategory inclusions
\begin{equation*}
     \somegaCat \incl \omegaCat \incl \Comp.
\end{equation*}
We define
\begin{equation*}
    \rcs \colon \omegaCat \to \somegaCat
\end{equation*}
to be the functor applying the reflector \( \rc \colon \Comp \to \somegaCat  \) to the underlying composition structure of a strict \( \omega \)\nbd category, which is left adjoint and exhibit \( \somegaCat \) as a reflective subcategory of \( \omegaCat \).

\cccom{TODO: say this gives a commutative diagram with the truncations}

\begin{thm}\label{thm:strict_le_3_are_stricter}
    Let \( n \le 3 \), and \( C \) be a strict \( n \)\nbd category.
    Then \( C \) is a stricter \( n \)\nbd category.
\end{thm}
\begin{proof}
    Let \( P \) be a regular directed complex with \( \dim P \le 3 \).
    By \cite[Corollary 8.4.12]{hadzihasanovic2024combinatorics}, \( \molecin{P} \) is a polygraph.
    Thus the map
    \begin{equation*}
        s_P \colon \colim_{x \in P} \imel{P}{x} \to P 
    \end{equation*}
    is an isomorphism in \( \omegaCat \).
    In particular, any strict \( n \)\nbd category \( C \) is local with respect to \( s_P \).
    By Lemma \ref{lem:stricter_n_iff_local_with_dim_le_n}, this concludes the proof.
\end{proof}



\cccom{TODO: omega-cat are the local presheaves for a certain subset of Theta, might even close it under Gray product to give an equivalent definition of the Gray product, uses amar's book on Steiner theory to get a quick proof of the following fact.}

Now recall from \cite[Appendice A]{ara2020joint} that strict \( \omega \)\nbd categories also support a Gray product, defined along a similar method.

\begin{prop} \label{prop:reflection_to_stricter_monoidal}
    The functor \( \rcs \colon \omegaCat \to \somegaCat \) is strong monoidal with respect to the Gray product on stricter and strict \( \omega \)\nbd categories.
\end{prop}
\begin{proof}
    \cccom{TODO}
\end{proof}


\subsection{Stricter polygraphs}

\noindent The following definition is adapted from \cite[8.2.1]{hadzihasanovic2024combinatorics}.
\begin{dfn} [Cellular extension] \label{dfn:cellular_extension}
    Let \( C \) be a stricter \( \omega \)\nbd category.
    A \emph{cellular extension of \( C \)} is a stricter \( \omega \)\nbd category \( C_{\cls{S}} \) together with a pushout diagram 
    \begin{center}
        \begin{tikzcd}[column sep=large]
            {\coprod_{e \in \cls{S}} \bd{}{}U_e} & {\coprod_{e \in \cls{S}} \molecin{U_e}} \\
            C & {C_{\cls{S}}}
            \arrow[""{name=0, anchor=center, inner sep=0}, "{\molecin{\bd{e}{}}}", from=1-1, to=1-2]
            \arrow["{(\bd{}{}e)_{e \in \cls{S}}}"', from=1-1, to=2-1]
            \arrow["{(e)_{e \in \cls{S}}}", from=1-2, to=2-2]
            \arrow[from=2-1, to=2-2]
            \arrow["\lrcorner"{anchor=center, pos=0.125, rotate=180}, draw=none, from=2-2, to=0]
        \end{tikzcd}
    \end{center}
    in \( \somegaCat \), where, for each \( e \in \cls{S} \), \( U_e \) is an atom.
\end{dfn}

\begin{comm}
    The previous definition is a non-standard definition of cellular extension, which is a priori more general than the standard one, which requires each of the atoms \( U_e \) to be globes.
    However, by the following Lemma, we may turn every cellular extension in this sense into one in the restricted sense. 
    See \cite[Comment 8.2.2]{hadzihasanovic2024combinatorics}. 
\end{comm}

\begin{lem} \label{lem:pushout_principal_cell}
    Let \( U \) be an atom of dimension \( n \), and \( s \colon \dglobe{n} \to U \) be the unique subdivision.
    Then the square
    \begin{center}
        \begin{tikzcd}
            {\molecin{(\bd{}{}\dglobe{n})}} & {\molecin{\dglobe{n}}} \\
            {\molecin{(\bd{}{}U)}} & {\molecin{U}}
            \arrow[from=1-1, to=1-2]
            \arrow["{\molecin{(\restr{s}{\bd{}{}\dglobe{n}})}}"', from=1-1, to=2-1]
            \arrow["{\molecin{s}}", from=1-2, to=2-2]
            \arrow[from=2-1, to=2-2]
        \end{tikzcd}
    \end{center}
    is a pushout square in \( \somegaCat \).
\end{lem}
\begin{proof}
    By \cite[Lemma 9.1.12]{hadzihasanovic2024combinatorics}, the square is a pushout in \( \omegaCat \).
    Since all the strict \( \omega \)\nbd categories involved are stricter, we can conclude by an application of the left adjoint functor \( \rcs \). 
\end{proof}

\begin{dfn} [Stricter polygraph]
    A \emph{stricter polygraph} is a stricter \( \omega \)\nbd category \( C \), together with, for each \( n \geq 0 \), a pushout diagram
    \begin{center}
        \begin{tikzcd}[column sep=large]
            {\coprod_{e \in \cls{S}_n} \bd{}{}U_e} & {\coprod_{e \in \cls{S}_n} \molecin{U_e}} \\
            \skel{n - 1}C & {\skel{n} C}
            \arrow[""{name=0, anchor=center, inner sep=0}, "{\molecin{\bd{e}{}}}", from=1-1, to=1-2]
            \arrow["{(\bd{}{}e)_{e \in \cls{S}_n}}"', from=1-1, to=2-1]
            \arrow["{(e)_{e \in \cls{S}_n}}", from=1-2, to=2-2]
            \arrow[from=2-1, to=2-2]
            \arrow["\lrcorner"{anchor=center, pos=0.125, rotate=180}, draw=none, from=2-2, to=0]
        \end{tikzcd}
    \end{center}
    in \( \somegaCat \), exhibiting the \( \skel{n}C \) as a cellular extension of \( \skel{n - 1}C \), and such that for each \( e \in \cls{S}_n \), \( \dim e = n \).
    More generally, we say that a strict functor \( f \colon A \to C \) of composition structure is a \emph{relative stricter polygraph} if \( f \) can be obtain as the transfinite composition of cellular extensions of \( A \).
\end{dfn}

\begin{rmk} \label{rmk:reflection_of_polygraph_is_stricter_polygraph}
    Since the functor \( \rcs \) is left adjoint, if a strict \( \omega \)\nbd category \( C \) is a polygraph in the traditional sense, then \( \rcs C \) is a stricter polygraph.
\end{rmk}

\begin{lem} \label{lem:stricter_regular_complex_are_stricter_polygraph}
    Let \( P \) be a regular direct complex.
    Then \( \molecin{P} \) is a stricter polygraph
\end{lem}
\begin{proof}
    By Corollary \ref{cor:molecin_preserves_pushout_inclusions}, the square
    \begin{center}
        \begin{tikzcd}
            {\coprod_{x \in \gr{n}P} \molecin{\bd{}{}\imel P x}} & {\coprod_{x \in \gr{n}P} \molecin{\imel P x}} \\
            {\skel{n - 1}P} & {\skel{n} P}
            \arrow["{{\molecin{\bd{e}{}}}}", from=1-1, to=1-2]
            \arrow["{{(\bd{}{}\molecin{\mapel x})_{x \in \gr n P}}}"', from=1-1, to=2-1]
            \arrow["{{(\molecin{\mapel x})_{x \in \gr n P}}}", from=1-2, to=2-2]
            \arrow[from=2-1, to=2-2]
        \end{tikzcd}
    \end{center}
    is a pushout in \( \somegaCat \).
    This concludes the proof.
\end{proof}

\subsection{Folk model structure on stricter \texorpdfstring{$\omega$}{}-categories}

\begin{dfn} [Acyclic fibration] \label{dfn:acyclic_fibration}
    Let \( f \colon C \to D \) be a strict functor of compositions structures.
    We say that \( f \) is an \emph{acyclic fibration} if
    \begin{enumerate}
        \item \( f \) is surjective on objects;
        \item for each \( n \geq 0 \), each pair \( c, c' \) of parallel globular \( n \)\nbd cells of \( C \), and each globular \( (n + 1) \)\nbd cell \( v \colon f(c) \gcelto f(c') \), there exists a globular \( (n + 1) \)\nbd cell \( u \colon c \gcelto c' \) such that \( f(u) = v \)
    \end{enumerate}
\end{dfn}

\begin{dfn} [Reversible globular cell]
    Let \( C \) be a composition structure, and \( e \colon x \gcelto y \) be a globular cell of dimension \( n > 0 \) in \( C \).
    We say that \( e \) is reversible if there exists a globular cell \( e^* \colon y \gcelto x \), as well as globular cells \( h \colon e \comp{n - 1} e^* \gcelto x \) and \( h' \colon e^* \comp{n - 1} e \gcelto y \) such that \( h \) and \( h' \) are reversible.
    We write \( x \sim y \) if there exists a reversible globular cell \( e \) of type \( x \gcelto y \).
\end{dfn}

\begin{dfn} [\( \omega \)\nbd equivalence]
    Let \( f \colon C \to D \) be a strict functor of composition structures.
    We say that \( f \) is an \( \omega \)\nbd equivalence if
    \begin{enumerate}
        \item for each globular cell \( d \) in \( D \) of dimension \( 0 \), there exists a globular cell \( c \) in \( C \) such that \( f(c) \sim d \);
        \item for each \( n \geq 0 \), each pair \( c, c' \) of parallel globular \( n \)\nbd cells of \( C \), and each globular \( (n + 1) \)\nbd cell \( v \colon f(c) \gcelto f(c') \), there exists a globular \( (n + 1) \)\nbd cell \( u \colon c \gcelto c' \) such that \( f(u) \sim v \)
    \end{enumerate}
\end{dfn}

\begin{comm}
    The definitions of acyclic fibration, reversibility, and \( \omega \)\nbd equivalence make sense in any composition structure, in particular in a strict or stricter \( \omega \)\nbd category.
    In the former case, we recover the usual definitions, see for instance \cite[19.2.3, 20.1.1, 20.1.11]{ara2025polygraphs}.
\end{comm}

\begin{thm} \label{thm:folk_model_structure}
    There exists a model structure, called the \emph{folk model structure}, on \( \omegaCat \) such that:
    \begin{enumerate}
        \item the weak equivalences are the \( \omega \)\nbd equivalences;
        \item the acyclic fibrations are the one of Definition \ref{dfn:acyclic_fibration};
        \item every strict \( \omega \)\nbd category is fibrant. 
    \end{enumerate} 
\end{thm}
\begin{proof}
    See \cite{lafont2010folk}.
\end{proof}

\noindent We conclude this section by giving an analogue to Theorem \ref{thm:folk_model_structure}, with \( \somegaCat \) in place of \( \omegaCat \).
We let 
\begin{equation*}
    \Icof \eqdef \set{i_n \colon \bd{}{} \globe{n} \incl \globe{n} \mid n \geq 0}.
\end{equation*}
By the small object argument in \( \somegaCat \), each strict functor \( f \colon C \to D \) of stricter \( \omega \)\nbd categories factors as \( f = p i \) where \( i \) is a relative stricter polygraph and \( p \) has the right lifting property against \( \Icof \), that is, \( p \) is an acyclic cofibration.

\begin{dfn} \label{dfn:generating_folk_acyclic_cof}
    For each \( n \geq 0 \), consider the strict functor \( \bd{}{} \globe{n + 1} \to \globe{n} \) sending the two non-trivial globular \( n \)\nbd cells of \( \bd{}{} \globe{n + 1} \) to the only non-trivial globular cell of \( \globe{n} \).
    We fix a factorisation of this strict functor
    \begin{equation*}
        \bd{}{} \globe{n + 1} \stackrel{j}{\to} \swE{n + 1} \stackrel{p}{\to} \globe{n}
    \end{equation*}
    into a stricter cofibration followed by an acyclic fibration, and let \( j_n \) be the composite
    \begin{equation*}
        \globe{n} \incl \bd{}{} \globe{n + 1} \stackrel{j}{\to} \swE{n + 1},
    \end{equation*}
    where \( \globe{n} \incl \bd{}{} \globe{n + 1} \) is the ``source'' inclusion, that is, the one factoring through \( \molecin{(\bd{}{-}\dglobe{n + 1})} \incl \molecin{\dglobe{n + 1}} \).
    We let
    \begin{equation*}
        \sJcof \eqdef \set{j_n \colon \globe{n} \to \swE{n + 1} \mid n \geq 0}.
    \end{equation*} 
\end{dfn}

\begin{dfn} [Stricter \( \omega \)\nbd category of cylinders]
    Let \( C \) be a stricter \( \omega \)\nbd category.
    The \emph{stricter \( \omega \)\nbd category of cylinder} is the stricter \( \omega \)\nbd category 
    \begin{equation*}
       \Gamma(C) \eqdef \homlax(\globe{1}, C). 
    \end{equation*}
\end{dfn}

\begin{rmk} \label{rmk:strict_stricter_same_cylinders}
    By Proposition \ref{prop:reflection_to_stricter_monoidal}, the stricter categories \( \Gamma(C) \) coincides with the strict \( \omega \)\nbd categories of cylinders \cite[Remark 20.2.9]{ara2025polygraphs}, which happen to be stricter, since \( C \) is.
\end{rmk}

\begin{thm} \label{thm:folk_model_structure_on_stricter}
    There is a cofibrantely generated model structure on the category \( \somegaCat \), called the \emph{folk model structure}, where:
    \begin{enumerate}
        \item \( \Icof \) is a set of generating cofibrations;
        \item \( \sJcof \) is a set of generating acyclic cofibrations;
        \item weak equivalences are the \( \omega \)\nbd equivalences;
        \item acyclic fibrations are the one of Definition \ref{dfn:acyclic_fibration}.
        \item all stricter \( \omega \)\nbd categories are fibrant;
        \item cofibrant objects are the stricter polygraphs.
    \end{enumerate}
    Furthermore, this model structure is right transferred from the folk model structure on \( \omegaCat \) along the adjunction 
    \begin{center}
        \begin{tikzcd}
            \somegaCat & \omegaCat,
            \arrow[""{name=0, anchor=center, inner sep=0}, "\iota"', curve={height=12pt}, hook, from=1-1, to=1-2]
            \arrow[""{name=1, anchor=center, inner sep=0}, "\rcs"', curve={height=12pt}, from=1-2, to=1-1]
            \arrow["\dashv"{anchor=center, rotate=-90}, draw=none, from=1, to=0]
        \end{tikzcd}
    \end{center}
    making it a Quillen adjunction.
\end{thm}
\begin{proof}
    This is a direct application of \cite[Proposition 21.3.2]{ara2025polygraphs} with Remark \ref{rmk:strict_stricter_same_cylinders}, noticing that \( \rcs \Icof = \Icof \), and that, by \cite[20.4.7]{ara2025polygraphs}, a generating set \( \Jcof \) of acyclic cofibrations for the folk model structure on \( \omegaCat \) is given by the same construction as (\ref{dfn:generating_folk_acyclic_cof}), but computing the small objects argument in \( \omegaCat \) instead of \( \somegaCat \).
    As a consequence, the classes \( \llp(\rlp(\rcs\Jcof))  \) and \( \llp(\rlp(\sJcof)) \) coincide.
\end{proof}

\begin{rmk} 
    By Lemma \ref{lem:pushout_principal_cell}, the set
    \begin{equation*}
        \set{\molecin{\bd{}{}U} \incl \molecin{U} \mid U \text{ atom}}
    \end{equation*}
    is also a generating set of acyclic cofibrations for the folk model structure on \( \somegaCat \), whose cofibrations are therefore given by the retracts of relative polygraphs.
\end{rmk}

\noindent Finally, we fix \( n \in \mathbb{N} \), and let
\begin{equation*}
    \Icof_n \eqdef \trunc{n}\Icof,\quad\quad \sJcof_n \eqdef \trunc{n}\Jcof.
\end{equation*}

\begin{thm} \label{thm:folk_model_structure_on_stricter_n}
    There is a cofibrantely generated model structure on the category \( \snCat{n} \), called the \emph{folk model structure}, where:
    \begin{enumerate}
        \item \( \Icof_n \) is a set of generating cofibrations;
        \item \( \sJcof_n \) is a set of generating acyclic cofibrations;
        \item weak equivalences are the \( \omega \)\nbd equivalences;
    \end{enumerate}
    Furthermore, this model structure is right transferred from the folk model structure on \( \nCat{n} \) along the adjunction 
    \begin{center}
        \begin{tikzcd}
            \snCat{n} & \nCat{n},
            \arrow[""{name=0, anchor=center, inner sep=0}, "\iota"', curve={height=12pt}, hook, from=1-1, to=1-2]
            \arrow[""{name=1, anchor=center, inner sep=0}, "\rcs"', curve={height=12pt}, from=1-2, to=1-1]
            \arrow["\dashv"{anchor=center, rotate=-90}, draw=none, from=1, to=0]
        \end{tikzcd}
    \end{center}
    making it a Quillen adjunction.
\end{thm}
\begin{proof}
    Same as Theorem \ref{thm:folk_model_structure_on_stricter}.
\end{proof}

\noindent Therefore, we have the following commutative square of left Quillen functors
\begin{center}
    \begin{tikzcd}
        \omegaCat & \somegaCat \\
        {\nCat{n}} & {\snCat{n}.}
        \arrow["\rcs", from=1-1, to=1-2]
        \arrow["{\trunc{n}}"', from=1-1, to=2-1]
        \arrow["{\trunc{n}}", from=1-2, to=2-2]
        \arrow["\rcs"', from=2-1, to=2-2]
    \end{tikzcd}
\end{center}
\section{Diagrammatic sets}

\subsection{Diagrams in a diagrammatic set}

Recall that a diagrammatic set is a presheaf on the category \( \atom \).
We let \( \dgmSet \) be the category of diagrammatic sets.
By \cite[Lemma 2.5]{chanavat2024htpy}, the Yoneda embedding \( \atom \incl \dgmSet \) factors as
\begin{equation*}
    \atom \incl \rdcpx \incl \dgmSet,
\end{equation*} 
and in the sequel, we always identify a regular directed complex with its associated diagrammatic set.

\begin{dfn} [Diagram in a diagrammatic set]
    Let \( U \) be a regular directed complex and \( X \) a diagrammatic set.
    A \emph{diagram of shape \( U \) in \( X \)} is a morphism \( u \colon U \to X \).
    A diagram is called
    \begin{itemize}
        \item a \emph{pasting diagram} if \( U \) is a molecule,
        \item a \emph{round diagram} if \( U \) is a \emph{round} molecule, and
        \item a \emph{cell} if \( U \) is an atom.
    \end{itemize}
    We write \( \dim u \eqdef \dim U \).
    We write respectively \( \Pd X \), \( \Rd X \) and \( \cell X \) the sets of pasting diagrams, round diagrams, and cells in \( X \).
\end{dfn}

\begin{rmk}
    By the Yoneda Lemma, we identify a cell \( u \colon U \to X \) with its corresponding element \( u \in X(U) \).
    Furthermore, since isomorphisms of molecules are unique when they exists, we may safely identify isomorphic pasting diagrams in the slice over \( X \).
\end{rmk}

\begin{dfn} [Subdiagram]
    Let \( u \colon U \to X \) be a pasting diagram in a diagrammatic set \( X \).
    A \emph{subdiagram of \( u \)} is a pair of a pasting diagram \( v \colon V \to X \) and a submolecule inclusion \( \iota \colon V \incl U \) such that \( u \after \iota = v \).
    A subdiagram is \emph{rewritable} if the submolecule inclusion \( \iota \) is rewritable.
    We write \( \iota \colon v \submol u \) for the data of a subdiagram of \( u \), or simply \( v \submol u \) if \( \iota \) is irrelevant or evident from the context. 
\end{dfn}

\begin{dfn} [Composition structure of pasting diagrams]
    Let \( u \colon U \to X \) be a pasting diagram in a diagrammatic set \( X \).
    For \( n \geq 0 \) and \( \a \in \set{-, +} \), we write \( \bd{n}{\a} u \) for the pasting diagram \( \restr{u}{\bd{n}{\a}U} \colon \bd{n}{\a} U \to X \).
    We may omit the index \( n \) if \( n = \dim u - 1 \).
    This makes \( \Pd(X) \) a reflexive \( \omega \)\nbd graph.
    Now let \( u \colon U \to X \) and \( v \colon V \to X \) be pasting diagrams, such that \( \bd{k}{+} u = \bd{k}{-} v \).
    We let \( u \cp{k} v \colon U \cp{k} V \to X \) be the unique pasting diagram determined by the universal property of the pushout \( U \cp{k} V \).
    This makes \( \Pd(X) \) a composition structure.
    % More generally, for any generalised pasting at the \( k \)\nbd boundary \( U \gencp{k} V \), we write \( u \gencp{k} v \colon U \gencp{k} V \to X \) for the pasting diagram determined by the universal property of the pushout \( U \gencp{k} V \).
\end{dfn}
\noindent We often omit the index \( k \) when it is equal to \( \min \set{\dim u, \dim v} - 1 \), and omit \( \iota \) when it is irrelevant or evident from the context.

\begin{rmk}
    \( \Rd(X) \) is a sub-reflexive graph \( \omega \)\nbd graph of \( \Pd(X) \), but is not itself a composition structure, since pasting of round diagrams are generally not round. 
\end{rmk}

\begin{dfn} [Degenerate pasting diagram]
    Let \( u \colon U \to X \) be a diagram.
    We say that \( u \) is \emph{degenerate} if there exists a pair of a diagram \( v \colon V \to X \) and a surjective cartesian map of regular directed complexes \( p \colon U \to V \) such that \( v \after p = u \), and \( \dim v < \dim v \). 
    We let
    \begin{align*}
        \Dgn X &\eqdef \set{u \in \Pd X \mid u \text{ is degenerate}} & \dgn X \eqdef \Dgn X \cap \cell x,\\
        \Nd X &\eqdef \set{u \in \Pd X \mid u \text{ is not degenerate}} & \nd X \eqdef \Nd X \cap \cell x.
    \end{align*}
\end{dfn}

\begin{dfn} [Reverse of a degenerate diagram]
    Let \( u \colon U \to X \) be a degenerate diagram, equal to \( v \after p \) for some diagram \( v \colon V \to X \) and surjective cartesian map \( p \colon U \to V \) with \( n \eqdef \dim u > \dim v \).
    The \emph{reverse of \( u \)} is the diagram \( \rev{u} \eqdef v \after \dual{n}{p} \colon \dual{n}{U} \to X \).
\end{dfn}

\cccom{suggestion: for all the following maps, just give the type of the associated degenerate diagram, wihtout expliciting the combinatorial description of its representation}

\begin{dfn}[Partial cylinder]
    Given a graded poset \( P \) and a closed subset \( K \subseteq P \), the \emph{partial cylinder on \( P \) relative to \( K \)} is the graded poset \( I \times_K P \) obtained as the pushout
    \begin{center}
        \begin{tikzcd}
            {I \times K} & K \\
            {I \times P} & {I \times_K  P}
            \arrow[two heads, from=1-1, to=1-2]
            \arrow[hook', from=1-1, to=2-1]
            \arrow["{(-)}", hook', from=1-2, to=2-2]
            \arrow["q", two heads, from=2-1, to=2-2]
            \arrow["\lrcorner"{anchor=center, pos=0.125, rotate=180}, draw=none, from=2-2, to=1-1]
        \end{tikzcd}  
    \end{center}
    in the category of posets.
    This is equipped with a canonical projection map \( \tau_K \colon I \times_K P \surj P \).
\end{dfn}

\begin{dfn}[Partial Gray cylinder]
	Let \( U \) be a regular directed complex and \( K \subseteq U \) a closed subset.
	The \emph{partial Gray cylinder on \( U \) relative to \( K \)} is the oriented graded poset \( \arr \gray_K U \) whose
    \begin{itemize}
        \item underlying graded poset is \( I \times_K U \), and
        \item orientation is specified, for all \( \a \in \set{+, -} \), by
        \begin{align*}
            \faces{}{\a}(x) & \eqdef \set{(y) \mid y \in \faces{}{\a}x}, \\
            \faces{}{\a}(i, x) & \eqdef \begin{cases}
                \set{(0^\a, x)} + \set{(1, y) \mid y \in \faces{}{-\a}x \setminus K} &
                \text{if \( i = 1 \),} \\
                \set{(i, y) \mid y \in \faces{}{\a}x \setminus K} + 
                \set{(y) \mid y \in \faces{}{\a}x \cap K} &
                \text{otherwise}.
            \end{cases}
        \end{align*}
    \end{itemize}
\end{dfn}

\begin{dfn}[Inverted partial Gray cylinder] \ccnote{would be nice if I don't need this}
	Let \( U \) be a molecule, \( n \eqdef \dim U \), and \( K \subseteq \bd{}{+}U \) a closed subset.
	The \emph{left-inverted partial Gray cylinder on \( U \) relative to \( K \)} is the oriented graded poset \( \lcyl{K} U \) whose
    \begin{itemize}
        \item underlying graded poset is \( I \times_K U \), and
        \item orientation is as in \( \arr \gray_K U \), except for all \( x \in \gr{n}{U} \) and \( \a \in \set{+, -} \)
    \begin{align*}
        \faces{}{-}(1, x) &\eqdef \set{(0^-, x), (0^+, x)} + \set{(1, y) \mid y \in \faces{}{+}x \setminus K}, \\
        \faces{}{+}(1, x) &\eqdef \set{(1, y) \mid y \in \faces{}{-}x}, \\
        \faces{}{\a}(0^+, x) &\eqdef \set{(0^+, y) \mid y \in \faces{}{-\a}x \setminus K} + 
            \set{(y) \mid y \in \faces{}{-\a}x \cap K}.
    \end{align*}
\end{itemize}
	Dually, if \( K \subseteq \bd{}{-}U \), the \emph{right-inverted partial Gray cylinder on \( U \) relative to \( K \)} is the oriented graded poset \( \rcyl{K}{U} \) whose
    \begin{itemize}
        \item underlying graded poset is \( I \times_K U \), and
        \item orientation is as in \( \arr \gray_K U \), except for all \( x \in \gr{n}{U} \) and \( \a \in \set{+, -} \)
        \begin{align*}
            \faces{}{-}(1, x) &\eqdef \set{(1, y) \mid y \in \faces{}{+}x}, \\
            \faces{}{+}(1, x) &\eqdef \set{(0^-, x), (0^+, x)} + \set{(1, y) \mid y \in \faces{}{-}x \setminus K}, \\
            \faces{}{\a}(0^-, x) &\eqdef \set{(0^-, y) \mid y \in \faces{}{-\a}x \setminus K} + 
                \set{(y) \mid y \in \faces{}{-\a}x \cap K}.
        \end{align*}
    \end{itemize}
\end{dfn}

\begin{rmk} \label{rmk:inverted_cylinder_well_def}
	By \cite[Lemma 1.20, Lemma 1.26]{chanavat2024equivalences}, partial Gray cylinders and inverted partial Gray cylinders respect the classes of molecules, round molecules and atoms.
	Moreover, for all molecules \( U \) and closed subset \( K \subseteq U \),
	\begin{itemize}
		\item \( \tau_K \colon \arr \gray_K U \to U \) is a cartesian map of molecules,
		\item if \( p \colon U \to V \) is a cartesian map of molecules with \( \dim V < \dim U \), then \( p \after \tau_K \colon \lcyl{K}{U} \to V \) and \( p \after \tau_K \colon \rcyl{K}{U} \to V \) are cartesian maps of molecules.
	\end{itemize}
\end{rmk}

\begin{dfn} [Higher invertor shapes]
    Let \( U \) be a round molecule.
    The family of \emph{higher invertor shapes on \( U \)} is the family of molecules \( \hcyl s U \) indexed by strings \( s \in \set{L, R}^* \), defined inductively on the length of \( s \) by
    \begin{align*}
        \hcyl{\langle\rangle} U & \eqdef U, \\
        \hcyl{Ls} U &\eqdef \lcyl{\bd{}{+} \hcyl{s} U} (\hcyl{s} U), \\
                \hcyl{Rs} U &\eqdef \rcyl{\bd{}{-} \hcyl{s} U} (\hcyl{s} U).
    \end{align*}
	These are equipped with cartesian maps \( \tau_s \colon \hcyl{s} U \to U \) of their underlying posets, with the property that for all cartesian maps of molecules \( p \colon U \to V \) such that \( \dim V < \dim U \), the composite \( p \after \tau_s \) is a cartesian map of molecules.
\end{dfn}

\begin{dfn} [Unit]
    Let \( u \colon U \to X \) be a pasting diagram.
    The \emph{unit on \( u \)} is the degenerate pasting diagram \( \un u \colon u \celto u \) defined by \( u \after \tau_{\bd{}{} U} \colon \arr \gray_{\bd{}{}U} U \to X \).
\end{dfn}

\begin{dfn} [Equivalence]
    Let \( e \) be a pasting diagram in a diagrammatic set \( X \).
    We say that \( e \) is an \emph{equivalence} if it is reversible in the composition structure \( \Pd(X) \).
    We let
    \begin{equation*}
        \Eqv X \eqdef \set{e \in \Pd(X) \mid e \text{ is an equivalence}},\quad \eqv X = \Eqv X \cap \cell X.
    \end{equation*}
\end{dfn}

\begin{rmk}
    This definition of equivalence coincides, for round diagrams, with the usual notion of equivalence in a diagrammatic set, see \cite[Section 2]{chanavat2024equivalences}.
\end{rmk}

\begin{prop} \label{prop:main_equivalence}
    Let \( X \) be a diagrammatic set.
    Then
    \begin{enumerate}
        \item every degenerate diagram is an equivalence;
        \item any two weak inverses of an equivalence are equivalent to each other;
    \end{enumerate}
    Furthermore, any morphism \( f \colon X \to Y \) of diagrammatic sets sends equivalences to equivalences.
\end{prop}
\begin{proof}
    See \cite[Section]{chanavat2024equivalences} for the case of equivalences that are round diagrams, and notice that all the relevant results go through when considering not necessarily round diagrams.
\end{proof}


\subsection{Model structure for diagrammatic \texorpdfstring{$(\infty, n)$}{(infty, n)}-categories}


Recall that a \emph{marked diagrammatic set} is a diagrammatic set \( X \) together with a set \( A \subset \gr{> 0}{\cell X} \) called the \emph{marked cell}, containing all the degeneracies. 
A morphism of marked diagrammatic sets is a morphism of the underlying diagrammatic sets sending marked cells to marked cells.
We write \( \mdgmSet \) for the category of marked diagrammatic sets and their morphisms. 
Furthermore, if \( P \) is a regular directed complex and \( A \subseteq \gr{> 0}{P} \) is a subset of elements of \( P \) of dimension \( > 0 \), we let \( (P, A) \) be the marked diagrammatic set \( (P, \dgn P \cup \set{\mapel{a} \mid a \in A}) \).

\begin{dfn}
    The functor \( \fun{U} \colon \mdgmSet \to \dgmSet \) forgetting the marking of a marked diagrammatic set has a left adjoint \( \minmark{(-)} \) defined by \( \minmark{X} \eqdef (X, \dgn X) \).
    Given a diagrammatic set, we also let \( \natmark{X} \eqdef (X, \eqv X) \).
    By Proposition \ref{prop:main_equivalence}, this is well defined and extends to a functor \( \natmark{(-)} \colon \dgmSet \to \mdgmSet \). 
\end{dfn}

\begin{dfn} [Marking]
   Let \( j \colon (X, A) \to (Y, B) \) be a morphism of marked diagrammatic sets. 
   We say that \( j \) is a \emph{marking} if \( \fun{U}j \) is an isomorphism.
\end{dfn}

\begin{dfn}
    We recall the definition of \emph{localisation} from \cite[Section 2.4]{chanavat2024model}.
    A \emph{cellular extension} of a diagrammatic set \( X \) is a pushout diagram
    \begin{center}
        \begin{tikzcd}
            {\coprod_{e \in \cls{S}} \bd{}{}U_e} &&& {\coprod_{u \in \cls{S}} U_e} \\
            X &&& {X_\cls{S}}
            \arrow["{(\bd{}{}e)_{e \in \cls{S}}}", from=1-1, to=2-1]
            \arrow["{(e)_{e \in \cls{S}}}", from=1-4, to=2-4]
            \arrow[hook, from=2-1, to=2-4]
            \arrow["{\coprod_{e \in \cls{S}}\bd{U_e}{}}", hook, from=1-1, to=1-4]
            \arrow["\lrcorner"{anchor=center, pos=0.125, rotate=180}, draw=none, from=2-4, to=1-1]
        \end{tikzcd}
    \end{center}
    in \( \dgmSet \) such that for each \( e \in \cls{S} \), \( U_e \) is an atom.

    Let \( (X, A) \) be a marked diagrammatic set.
    We define \( \preloc{X}{A} \) to be the diagrammatic set obtain by, for each cell \( a \colon u \celto v \) in \( A \cap \nd X \),
    \begin{enumerate}
        \item attach cells \( a^L, a^R  \colon v \celto u \), then
        \item attach cells \( \hinv{L}(a) \colon a \cp{} a^L \celto \un(u) \) and \( \hinv{R} \colon \un(v) \celto a^R \cp{} a \).
    \end{enumerate} 
    Then, let \( \order{0}{X} \eqdef X \) and \( \order{0}{A} \eqdef A \).
    Inductively on \( n > 0 \), define
    \begin{equation*}
        \order{n}{X} \eqdef \preloc{\order{n - 1}{X}}{\order{n - 1}{A}}, \quad\quad \order{n}{A} \edef \set{\hinv{R}a, \hinv{L}a \mid a \in \order{n - 1}{A}}.
    \end{equation*}
    We then have a sequence of inclusions
    \begin{equation*}
        \order{0}{X} \incl \order{1}{X} \incl \ldots \incl \order{n}{X} \incl \ldots,
    \end{equation*}
    whose transfinite composition is \( \loc{X}{A} \), the \emph{localisation of \( X \) at \( A \)}, which comes equipped with a canonical inclusion \( X \incl \loc{X}{A} \).
    This can be extended to a colimit preserving functor
    \begin{equation*}
        \Loc \colon \mdgmSet \to \dgmSet.
    \end{equation*}
    By definition, each cell in \( A \) becomes invertible in \( \loc{X}{A} \).
\end{dfn}

\begin{dfn} [Walking equivalence]
    Let \( U \) be an atom.
    The \emph{walking equivalence of shape \( U \)} is the diagrammatic set \( \selfloc{U} \eqdef \Loc (U, \set{\top_U}) \).
    We also write \( V \simeq W \) for the walking equivalence of shape \( V \celto W \), and \( \rglobe{n} \) for \( \dglobe{n - 1} \simeq \dglobe{n - 1} \) for \( n \geq 1 \). 
\end{dfn}

% \begin{dfn}
%     Let \( X \) be a diagrammatic set, and \( u \colon U \to X \) be cell.
%     We write \( \td u \colon \selfloc{U} \to X \) for any extension of \( u \) along the inclusion \( U \incl \selfloc{U} \).
%     If \( u \colon a \celto b \), then we write \( \td u \colon a \simeq b \).
%     Furthermore if \( v \colon V \to X \) is a diagram and \( k \geq 0 \) such that \( u \cp{k} v \) is defined, we also write \( \td u \cp{k} v \colon \selfloc{U} \cp{k} V \to X \) for the morphism \( \td u \cup_{\bd{k}{+}u} v \) whose domain is \( \tilde{U} \cup_{\bd{k}{\a} U} V \).
%     We use similar notations of pasting at a subdiagrams \( - \cpsub{} - \) and \( - \subcp{} - \).
% \end{dfn}

\begin{dfn} [Weak composites]
    Let \( X \) be a diagrammatic set.
    We say that \( X \) has \emph{weak composites} if for each round diagram \( u \colon U \to X \), there exists a cell \( \compos{u} \colon \compos{U} \to X \) parallel to \( u \) such that \( u \simeq \compos{u} \).
    In that case, \( \compos{u} \) is called a weak composite of \( u \).
\end{dfn}

\begin{dfn} [\( (\infty, n) \)\nbd category]
    Let \( n \in \mathbb{N} \cup \set{\infty} \), and \( X \) be a diagrammatic set.
    We say that \( X \) is an \( (\infty, n) \)\nbd category if:
    \begin{enumerate}
        \item \( X \) has weak composites
        \item all cells of dimension \( > n \) are equivalences.
    \end{enumerate}
    
    A morphisms of diagrammatic sets is called a \emph{functor} when its domain and codomain are \( (\infty, n) \)\nbd categories. 
\end{dfn}

\begin{rmk}
    In the case \( n = \infty \), the second condition is void.
\end{rmk}

\begin{dfn}
    Let \( U \) be a round molecule.
    The \emph{walking weak composite of \( U \)} is the inclusion of diagrammatic sets
    \begin{equation*}
        c_U \colon U \incl U \simeq \compos{U}.
    \end{equation*}
    We let \( \Jcomp \) be the set of walking weak composites.
\end{dfn}

\begin{dfn}
    Let \( n \in \mathbb{N} \cup \set{\infty} \).
    We let \( \Jn{n} \eqdef \set{U \incl \selfloc{U} \mid U \text{ atom}, \dim U > n } \).
\end{dfn}

\begin{rmk} \label{rmk:infty_n_cat_iff_rlp_Jcomp_Jn}
    By \cite[Proposition 3.8, Proposition 3.9]{chanavat2024model}, a diagrammatic set \( X \) is an \( (\infty, n) \)\nbd category if and only if it has the right lifting property against \( \Jcomp \cup \Jn{n} \).
\end{rmk}

\begin{lem} \label{lem:isofib_left_right_lift}
    Let \( f \colon X \to Y \) be a functor of \( (\infty, \infty) \)\nbd categories with the right lifting property against \( \Jcomp \), \( u \) be a round diagram in \( X \) and \( v \) a cell in \( Y \).
    Then
    \begin{enumerate}
        \item for all equivalences \( h \colon f(u) \celto v \), there exists an equivalence \( z \colon u \celto v' \) such that \( f(z) = h \);
        \item for all equivalences \( h \colon v \celto f(u) \), there exists an equivalence \( z \colon v' \celto u \) such that \( f(z) = h \).  
    \end{enumerate}
\end{lem}
\begin{proof}
    The first part follows from \cite[Proposition 3.7]{chanavat2024model}.
    Let \( h \colon v \celto f(u) \) be an equivalence.
    Then \( h \) has a weak inverse \( h' \colon f(u) \celto v \), which is itself an equivalence.
    Thus \( h' = f(z') \) for equivalence \( z' \colon v' \celto u \).
    Then, \( z' \) is an equivalence, thus has a weak inverse \( z'' \colon u \celto v' \).
    Since morphisms of diagrammatic set preserves weak inverses, \( f(z'') \) is a weak inverse of \( h' \).
    By Proposition \ref{prop:main_equivalence}, there exists an equivalence \( k \colon f(z'') \celto h \), 
    By the first part of the proof, \( k = f(k') \) for some equivalence \( k' \colon z'' \celto z \).
    In particular, \( h = f(z) \), and \( z \simeq z'' \) is an equivalence.
\end{proof}


\begin{dfn} [Marked horn]
    Let \( (U, A) \) be a marked atom with \( k \eqdef \dim U - 1 \geq 0 \), \( \a \in \set{-, +} \) and \( x \in \maxel{\bd{}{\a} U} \).
    Call \( \Lambda^x_U \eqdef U \setminus \set{\top_U, x} \)
    We say that the inclusion of marked regular directed complexes
    \begin{equation*}
        \lambda^x_U \colon (\Lambda^x_U, \Lambda^x_U \cap A) \incl (U, A) 
    \end{equation*}
    is a \emph{marked horn of \( U \)} if there exists molecules \( (\order{i}{L}, \order{i}{R})_{i = 1}^k \) such that
    \begin{enumerate}
        \item \( \bd{}{\a} U = \order{k}{L} \cp{k - 1} (\ldots \cp{1} \order{1}{L} \cp{0} \clset{x} \cp{0} \order{1}{R} \cp{1} \ldots) \cp{k - 1} \order{k}{R} \);
        \item \( \dim \order{i}{L}, \dim \order{i}{R} \le i \) for each \( 1 \le i \le k \);
        \item \( \gr{i}{\order{i}{L}} \cup \gr{i}{\order{i}{R}} \subseteq A \);
        \item \( x \in A \) if and only if \( \faces{}{-\a} U \subseteq A \).
    \end{enumerate}
    We let \( \Jhorn \) be the set of marked horn.
\end{dfn}

\begin{comm}
    Let \( \lambda^x_U \colon (\Lambda^x_U, \Lambda^x_U \cap A) \to (U, A) \) be a marked horn, and \( W \) be an \( (\infty, \infty) \)\nbd category.
    Recall from \cite[Comment 3.13]{chanavat2024model} that a morphism \( s \colon \Lambda^x_U \to W \) classifying in \( W \) an equation \( \fun{E}x \qeq v \) in the indeterminate \( x \), where \( v \eqdef \restr{s}{\bd{}{+}U} \), and \( \fun{E} \) is a context in the sense of \cite[3.1]{chanavat2024equivalences}.
    Now if \( s \) defines furthermore a morphisms of marked diagrammatic sets \( s \colon (\Lambda^x_U, \Lambda^x_U \cap A) \to \natmark{W} \), then this implies that the context \( \fun{E} \) is weakly invertible.
    By \cite[Lemma 5.10]{chanavat2024equivalences}, this equation has a solution \( u \), witnessed by an equivalence \( h \colon \fun{E}u \celto v \), which is exactly the data of an extension of \( s \) along \( \lambda^x_U \).
\end{comm}

\begin{lem} \label{lem:isofib_rlp_marked_horn}
    Let \( f \colon X \to Y \) be a functor of \( (\infty, \infty) \) category with the right lifting property against \( \Jcomp \),
    Then \( \natmark{f} \) has the right lifting property against \( \Jhorn \).
\end{lem}
\begin{proof}
    Let \( \lambda^x_U \colon (\Lambda^x_U, \Lambda^x_U \cap A) \to (U, A) \) be a marked horn, assume without loss of generality that \( x \in \bd{}{-} U \), and consider a lifting problem
    \begin{center}
        \begin{tikzcd}
            {(\Lambda^x_U, \Lambda^x_U \cap A)} & \natmark{X} \\
            {(U, A)} & {\natmark{Y}.}
            \arrow["s", from=1-1, to=1-2]
            \arrow[from=1-1, to=2-1]
            \arrow["\natmark{f}", from=1-2, to=2-2]
            \arrow["h"', from=2-1, to=2-2]
        \end{tikzcd}
    \end{center}
    By \cite[Theorem 4.9]{chanavat2024model}, both \( \natmark{X} \) and \( \natmark{Y} \) have the right lifting property against \( \Jhorn \).
    Then \( s \) classifies in \( X \) an equation \( \fun{E}x \qeq v \) in the unknown \( x \), where \( \fun{E} \) is a weakly invertible context. 
    Let us choose a solution \( u \) witnessed by an equivalence \( z \colon \fun{E}u \celto v \).
    Moreover, by assumption, \( h \colon (f\fun{E})w \celto f(v) \) is an equivalence in \( Y \).
    In \( Y \), we may form the equation 
    \begin{equation*}
        y \cpsub{f(u)} f(z) \qeq h,
    \end{equation*}
    in the unknown \( y \), which has a solution \( e \colon w \celto f(u) \) witnessed by an equivalence \( k \colon e \cpsub{} f(z) \celto h \).
    By Lemma \ref{lem:isofib_left_right_lift}, \( e = f(e') \) for some equivalence \( e' \colon w' \celto u \).
    Thus \( k \) has type \( f(e' \cpsub{} z) \celto h \).
    By Lemma \ref{lem:isofib_left_right_lift} again, \( k = f(k') \) for some equivalence \( k' \colon e' \cpsub{} z \celto h' \).
    In particular, \( h' \) is an equivalence of type \( \fun{E}w' \celto v \) and \( f(h') = h \).
    If \( \faces{}{+} U \subseteq A \), then \( v \) is an equivalence, hence by \cite[Theorem 5.22]{chanavat2024equivalences}, \( w' \) is an equivalence.
    This shows that \( h' \colon U \to X \) extends to a morphism \( (U, A) \to \natmark{X} \) solving the lifting problem.
    This concludes the proof.
\end{proof}

\begin{prop} \label{prop:model_structre_on_marked_dgm_set}
    For each \( n \in \mathbb{N} \cup \set{\infty} \) exists a model structure on diagrammatic sets, called the \emph{coinductive \( (\infty, n) \)\nbd model structure}, where
    \begin{enumerate}
        \item cofibrations are the monomorphisms;
        \item fibrant objects are of the form \(  \natmark{X} \) for an \( (\infty, n) \)\nbd category \( X \).
    \end{enumerate}
\end{prop}
\begin{proof}
    See \cite[3.24, Theorem 4.9]{chanavat2024model}.
\end{proof}

\begin{thm} \label{thm:n_model_structure_on_dgm_set}
    For each \( n \in \mathbb{N} \cup \set{\infty} \) exists a model structure on diagrammatic sets, called the \emph{\( (\infty, n) \)\nbd model structure}, where
    \begin{enumerate}
        \item cofibrations are the monomorphisms;
        \item fibrant objects are the \( (\infty, n) \)\nbd categories;
        \item \( \Jcomp \cup \Jn{n} \) is a pseudo-generating set of acyclic cofibrations.
    \end{enumerate}
    Furthermore, the adjunction \( \minmark{(-)} \dashv \fun{U} \) is a Quillen equivalence with the coinductive \( (\infty, n) \)\nbd model structure on marked diagrammatic sets.
\end{thm}
\begin{proof}
    The existence as well as the characterisation of fibrant objects, and the Quillen equivalence are given by \cite[3.27, Theorem 4.21, Theorem 4.23]{chanavat2024model}.
    In particular, the fibrant objects of the coinductive \( (\infty, n) \)\nbd model are of the form \( \natmark{X} \) for \( X \) an \( (\infty, n) \)\nbd category.
    By \cite[Lemma 4.17]{chanavat2024model}, one sees that the set \( (\minmark{\Jcomp \cup \Jn{n}}) \) is a set of acyclic cofibrations.
    By Remark \ref{rmk:infty_n_cat_iff_rlp_Jcomp_Jn}, \( \Jcomp \cup \Jn{n} \) detects fibrant objects.
    Since left Quillen equivalence reflect weak equivalence between cofibrant objects, \( \Jcomp \cup \Jn{n} \) is a set of acyclic cofibration.
    Let \( f \colon X \to Y \) be a functor of \( (\infty, n) \)\nbd category with the right lifting property against \( \Jcomp \cup \Jn{n} \).
    We need to prove that \( f \) is a fibration.
    Since \( \fun{U}\natmark{f} = f \), is it enough to prove that \( \natmark{f} \) is a fibration. 
    By Lemma \ref{lem:isofib_rlp_marked_horn}, \( \natmark{f} \) has the right lifting property against \( \Jhorn \).
    By \cite[Theorem 4.22]{chanavat2025gray}, a set of pseudo-generating acyclic cofibration for the coinductive \( (\infty, n) \)\nbd model structure is given by \( \Jhorn \cup J_{\mathsf{mark}} \), where \( J_{\mathsf{mark}} \) is a set of markings.
    But a morphism of marked diagrammatic sets has the right lifting property against a collection of markings \( J \) if and only if its domain has the right lifting property against \( J \).
    Since the domain of \( \natmark{f} \) is the fibrant object \( \natmark{X} \), we conclude.
\end{proof}

\begin{comm}
    In \cite{chanavat2024model}, a pseudo generating set of acyclic cofibrations for the \( (\infty, n) \)\nbd model structure was given by a certain closure \( \mathrm{an}(\Jcomp \cup \Jn{n}) \) of \( \Jcomp \cup \Jn{n} \) under certain Gray products.
    We showed in Theorem \ref{thm:n_model_structure_on_dgm_set} that this closure is unnecessary (which incidentally was already predicted by Hadzihasanovic in \cite[Remark 6.9]{hadzihasanovic2020diagrammatic}).
\end{comm}

\begin{dfn}
    For each \( n \in \mathbb{N} \cup \set{\infty} \), we call \( \fun{W}_n \) be the collection of weak equivalences of the \( (\infty, n) \)\nbd model structure on diagrammatic sets, and \( \fun{W}^{\m{}}_n \) the collection of weak equivalences of the coinductive \( (\infty, n) \)\nbd model structures on marked diagrammatic sets.
\end{dfn}
\begin{rmk}

    Then we have a sequence of inclusions
    \begin{equation*}
        \fun{W}_{\infty} \subseteq \ldots \subseteq \fun{W}_{n + 1} \subseteq \fun{W}_n \subseteq \ldots \subseteq \fun{W}_0.
    \end{equation*}
    Thus, if a certain morphism of diagrammatic sets is proven to belong to \( \fun{W}_n \), it also belongs to \( \fun{W}_k \) for all \( k \le n \).
    This discussion applies equally for the weak equivalences \( \fun{W}^{\m{}}_n \).
\end{rmk}

\begin{prop} \label{prop:loc_preserves_weak_equivalences} \cccom{do we need that?}
    Let \( n \in \mathbb{N} \cup \set{\infty} \).
    Then \( \Loc \fun{W}^{\m{}}_n = \fun{W}_n \).
\end{prop}
\begin{proof}
    Let \( f \colon (X, A) \to (Y, B) \) in \( \fun{W}^{\m{}}_n \).
    By Theorem \ref{thm:n_model_structure_on_dgm_set}, \( \Loc f \) is in \( \fun{W}_n \) if and only if \( \minmark{(\Loc f)} \) is in \( \fun{W}^{\m{}}_n \).
    Now for all marked diagrammatic sets \( (Z, C) \), \( (Z, C) \incl (\loc{Z}{C}, C \cup \dgn (\loc{Z}{C})) \) is in \( \fun{W}^{\m{}}_n \) by \cite[Lemma 4.17]{chanavat2024model}.
    Then the marking \( \minmark{(\Loc (Z, C))} \incl (\loc{Z}{C}, C \cup \dgn (\loc{Z}{C})) \) also belongs to \( \fun{W}^{\m{}}_n \) since it can be constructed as a transfinite composition of pushout along markings of the form
    \begin{equation*}
        \minmark{\selfloc{U}} \incl (\selfloc{U}, \set{U \incl \selfloc{U}} \cup \dgn \selfloc{U}), U \text{ atom},
    \end{equation*}
    which belong to \( \fun{W}^{\m{}}_n \) by definition of the coinductive \( (\infty, n) \)\nbd model structure.
    By the two-out-of-three, we conclude that \( \minmark{(\Loc f)} \) is in \( \fun{W}^{\m{}}_n \).
    Conversely, \( \minmark{\fun{W}_n} \subseteq \fun{W}^{\m{}}_n \), and by the first part of the proof
    \begin{equation*}
        \fun{W}_n = \Loc \minmark{\fun{W}_n} \subseteq \Loc \fun{W}^{\m{}}_n.
    \end{equation*}
    This concludes the proof.
\end{proof}

% \begin{lem}\label{lem:acyclic_on_globes_acyclic_on_all}
%     Let \( f \colon X \to Y \) be a fibration of \( (\infty, n) \)\nbd categories, and suppose that \( f \) has the right lifting property against
%     \begin{equation*}
%         \set{\bd{}{} \dglobe{n} \incl \dglobe{n} \mid n \geq 0}.
%     \end{equation*}
%     Then \( f \) is an acyclic fibration.
% \end{lem}


% \begin{dfn} [Globularisation]
%     Given a round molecule \( V \), we define the atoms \( C^+(V) \eqdef (V \celto \compos{V}) \) and \( C^-(V) \eqdef (\compos{V} \celto V) \).
%     Let \( U \) be an atom of dimension \( n \geq 2 \), we let
%     \begin{equation*}l
%         R(U) \eqdef C^-(\bd{}{-}U) \cp{n - 1} (\ldots \cp{2} C^-(\bd{1}{-}U) \cp{1} U \cp{1} C^+(\bd{1}{+} U) \cp{2} \ldots) \cp{} C^+(\bd{}{+}U),
%     \end{equation*}
%     and let \( G(U) \eqdef R(U) \celto \dglobe{n} \) and \( L(U) \eqdef G(U) \setminus \set{\top_{G(U)}, \top_U} \).
%     Now, calling \( \top^\a_k \) be the maximal element of \( \C^\a(\bd{k}{\a} U) \) in \( R(U) \) for \( \a \in \set{-, +} \) and \( 1 \le k \le n - 1 \), we get that
%     \begin{equation*}
%         \lambda^{\top_U} \colon (L(U), \bigcup_{k = 1}^{n - 1} \set{\top^\a_k}) \incl (G(U), {\top_{G(U)}, \top_U} \cup \bigcup \bigcup_{k = 1}^{n - 1} \set{\top^\a_k})
%     \end{equation*}
%     is a marked horn. 
%     The \emph{globularisation} of \( U \) is the morphism of diagrammatic set
% \end{dfn}

% \begin{prop}\label{prop:infty_n_cat_htpy_colimit_of_its_globes}
%     Let \( n \in \mathbb{N} \cup \set{\infty} \), \( Y \) be an \( (\infty, n) \)\nbd category, and \( \iota \colon X \subseteq Y \) a subpresheaf of \( Y \) such that for all \( n \geq 0 \), \( X(\dglobe{n}) = Y(\dglobe{n}) \).
%     Then \( \iota \) is an acyclic cofibration in the \( (\infty, n) \)\nbd model structure.
% \end{prop}
% \begin{proof}
%     Factors \( \iota = f \after j \) where \( j \colon X \to X' \) is an acyclic cofibration and \( p \colon X' \to Y \) is a fibration.
%     In particular, \( X' \) is fibrant. 
%     We show by induction on \(  \) 
%     Consider a lifting problem 
%     \begin{center}
%         \begin{tikzcd}
%             {\bd{}{} \dglobe{n}} & {X'} \\
%             {\dglobe{n}} & Y
%             \arrow[from=1-1, to=1-2]
%             \arrow[hook', from=1-1, to=2-1]
%             \arrow["f", from=1-2, to=2-2]
%             \arrow["u"', from=2-1, to=2-2]
%         \end{tikzcd}
%     \end{center}
%     By assumption \( u = \iota \after v \) for some \( v \colon \dglobe{n} \to A \).
%     Then 
% \end{proof}
\section{Folk model structure on stricter \texorpdfstring{$n$}{n}-categories}


\bibliographystyle{alpha}
\small\bibliography{main.bib}

\end{document}
