\section{Stricter \texorpdfstring{$n$}{n}-categories}

\subsection{Definitions and properties}

\begin{dfn} [Reflexive \( \omega \)\nbd graph]
    A \emph{reflexive \( \omega \)\nbd graph} is a set, whose element are called the \emph{globular cells}, \( C \) together with, for each \( k \geq 0 \), operators
    \begin{equation*}
        \bd{k}{-}, \bd{k}{+} \colon C \to C,
    \end{equation*}
    called the \emph{input and output \( k \)\nbd boundary}, respectively, satisfying the following axioms.
    \begin{enumerate}
        \item for all \( c \in C \), there exists \( k \geq 0 \) such that \( \bd{k}{-} c = c = \bd{k}{+} c \); the \emph{dimension} of \( c \), written \( \dim c \), is the minimum of all such values of \( k \);
        \item for all \( c \in C \), all \( k, n \geq 0 \) and all \( \a, \beta \in \set{-, +} \),
        \begin{equation*}
            \bd{k}{\a}(\bd{n}{\beta} c) = 
            \begin{cases}
                \bd{k}{\a} c & k < n,
                \bd{n}{\beta}, k \ge n.
            \end{cases}
        \end{equation*}
    \end{enumerate}
    A \emph{morphism of reflexive \( \omega \)\nbd graph} is a function of the underlying set commuting with the boundary operators.
\end{dfn}

\noindent If \( C \) is a reflexive \( \omega \)\nbd graph, the set of \emph{\( k \)\nbd composable pairs of globular cells} is the set 
\begin{equation*}
    C \times_k C \eqdef \set{(c, d) \in C \times C \mid \bd{k}{+} c = \bd{k}{-} d}.
\end{equation*}
Given a globular cell \( c \) and \( \a \in \set{-, +} \), we write \( \bd{}{\a} c \) in place of \( \bd{\dim c - 1}{\a} c \), and \( c \colon u^- \gcelto u^+ \) to signify that \( \bd{}{\a} c = u^\a \), and call \( u^- \gcelto u^+ \) the \emph{type of \( c \)}. 
Two globular cells of the same type are said to be \emph{parallel}.
We say that a globular cell \( c \) is an \emph{object} if \( \dim c = 0 \). 

\begin{dfn} [Composition structure]
    A composition structure is a reflexive \( \omega \)\nbd graph \( C \) together with, for all \( k \geq 0 \), an operation
    \begin{equation*}
        - \comp{k} - \colon C \times_k C \to C,
    \end{equation*}
    call the \emph{\( k \)\nbd composition}.
    If \( C, D \) are composition structures, a \emph{strict functor} \( f \colon C \to D \) is a morphism of the underlying reflexive graph respecting the \( k \)\nbd composition.
    We denote \( \Comp \) the category of composition structures and strict functors.
\end{dfn}

\begin{dfn} [Basis for composition structure]
    Let \( C \) be a composition structure, and \( \cls{S} \) be a subset of the globular cells of \( C \).
    We say that \( \cls{S} \) is a \emph{generating set for \( C \)} if the closure of \( \cls{S} \) under \( - \comp{k} - \) is equal to \( C \).
    We say that a generating set is a \emph{basis for \( C \)} if for any other generating set \( \cls{T} \subseteq \cls{S} \), then \( \cls{T} = \cls{S} \).
\end{dfn}

\begin{lem}\label{lem:strict_functor_determined_by_basis}
    Let \( f, g \colon C \to D \) be strict functors and let \( \cls{S} \) be a generating set for \( C \) such that for all \( c \in \cls{C} \), \( f(c) = g(c) \).
    Then \( f = g \).
\end{lem}
\begin{proof}
    See \cite[Lemma 5.1.23]{hadzihasanovic2024combinatorics}.
\end{proof}

\noindent Recall\ccnote{cannot use Nd here} from \cite[Section 5.2]{hadzihasanovic2024combinatorics} that given a regular directed complex \( P \), the set \( \molecin{P} \eqdef \Nd(P) \), together with the boundary operators \( \bd{k}{\a} \) and the pasting operations \( - \cp{k} - \) is a composition structure with basis \( \atomin{P} \eqdef \nd(P) \).
Furthermore, by \cite[Theorem 6.2.32]{hadzihasanovic2024combinatorics}, this assignment extends to a functor
\begin{equation*}
    \molecin{-} \colon \rdcpx \to \Comp.
\end{equation*}
We recall as well that \( \molecin{-} \) is also functorial with respect to subdivisions \( s \colon P \sd Q \) of regular directed complexes.

\begin{dfn} [Globe]
    Let \( n \geq 0 \).
    The \emph{\( n \)\nbd globe} is the composition structure defined by \( \globe{n} \eqdef \molecin{\dglobe{n}} \), and its \emph{boundary} is the composition structure given by \( \bd{}{}\globe{n} \eqdef \molecin{\bd{}{}\dglobe{n}} \).
\end{dfn}

\begin{rmk}
    As usual, a strict functor \( u \colon O^n \to C \) classifies a globular cell of \( C \), and a strict functor \( \bd{}{} O^n \to C \) classifies two parallel cells \( u, v \colon O^{n - 1} \to C \).
\end{rmk}

\begin{dfn} [Diagram in a composition structure]
    Let \( C \) be a composition structure and \( P \) be a regular directed complex.
    A \emph{diagram of shape \( P \) in \( C \)} is a strict functor \( \F \colon \molecin{P} \to C \).
    If \( P = U \) is a molecule, we speak of \emph{pasting diagram}, and in that case, for \( k \geq 0 \) and \( \a \in \set{-, +} \), we write \( \bd{k}{\a} F \) for the restriction of \( \F \) along the strict functor \( \molecin{\bd{k}{\a} U} \to \molecin{U} \).
    We write \( \Dgm(C) \) for the collection of all pasting diagrams of \( C \).
    % Finally, if \( U \) is an atom, we speak of \emph{atomic diagram} in \( C \), and write \( \dgm(C) \) for the collection of all atomic diagram of \( C \).
\end{dfn}

\begin{dfn} [Principal cell]
    Let \( \F \colon \molecin{U} \to C \) be a pasting diagram.
    The \emph{principal cell of \( \F \)} is the cell \( \pcell{\F} \eqdef \F(\idd{U}) \).
\end{dfn}

% \noindent For each pair of molecules \( U, V \) and \( k \geq 0 \) such that \( U \cp{k} V \) is defined, there is a canonical strict functor \( s^k_{UV} \)
% \begin{center}
%     \begin{tikzcd}
%         {\molecin{\bd{k}{+}U}} & {\molecin{V}} \\
%         {\molecin{U}} & {\molecin{U}\cup\molecin{V}} \\
%         && {\molecin{U \cp{k} V}}.
%         \arrow[""{name=0, anchor=center, inner sep=0}, from=1-1, to=1-2]
%         \arrow[from=1-1, to=2-1]
%         \arrow[from=1-2, to=2-2]
%         \arrow[curve={height=-18pt}, from=1-2, to=3-3]
%         \arrow[from=2-1, to=2-2]
%         \arrow[curve={height=12pt}, from=2-1, to=3-3]
%         \arrow["{s^k_{UV}}"{description}, from=2-2, to=3-3]
%         \arrow["\lrcorner"{anchor=center, pos=0.125, rotate=180}, draw=none, from=2-2, to=0]
%     \end{tikzcd}
% \end{center}
\noindent For each regular directed complex \( P \), there is a canonical strict functor
\begin{equation*}
    s_P \colon \colim_{x \in P} \molecin{\imel{P}{x}} \to \molecin{P}
\end{equation*}
We let \( S \) be the set of all the strict functors \( s_U \) for \( U \) a molecule.

\begin{dfn}[Stricter \( \omega \)\nbd category]
    A \emph{stricter \( \omega \)\nbd category} is a composition structure \( C \) which is local with respect to \( S \).
    We let \( \somegaCat \) be the full subcategory of \( \omegaCat \) on stricter \( \omega \)\nbd categories.
\end{dfn}

\noindent Since \( \Comp \) is locally presentable, as a category of models of a limit sketch, and \( S \) is a small set, the full subcategory inclusion \( \iota \colon \somegaCat \incl \Comp \) is reflective \cite{freyd1972continuous}, and we denote by 
\begin{equation*}
    \rc \colon \Comp \to \somegaCat
\end{equation*}
the left adjoint of \( \iota \).

\begin{dfn} [Matching family and amalgamation]
    Let \( P \) be a regular directed complex, and \( C \) be a composition structures.
    A \( P \)\nbd matching family in \( C \) is a cone 
    \begin{equation*}
        \set{\F_x \colon \molecin{\imel{P}{x}} \to C}_{x \in P}
    \end{equation*}
    under the \( P \)\nbd shaped diagram \( x \mapsto \molecin{\imel{P}{x}} \).    
    An \emph{amalgamation} of this matching family is a strict functor 
    \begin{equation*}
        \amalg_{x\in P} \F_x \colon \molecin{P} \to C
    \end{equation*}
    such that for all \( x \in P \), \( (\amalg_y \F_y) \after \molecin{\mapel{x}} = \F_x \).
\end{dfn}

\begin{rmk}\label{rmk:data_matching family}
    The data of a \( P \)\nbd matching family 
    \begin{equation*}
        \set{\F_x \colon \molecin{\imel{P}{x}} \to C}_{x \in P}
    \end{equation*}
    in \( C \) is given by an element \( c_x \in C \) for each \( x \in P \).
    Indeed, given a matching family \( \set{\F_x}_{x \in P} \), define \( c_x \eqdef \pcell{\F_x} \).
    By functoriality, if \( x \le y \), then \( c_x = \F_y(\clset{x} \incl \clset{y}) \), thus by Lemma \ref{lem:strict_functor_determined_by_basis}, this data entirely determines the matching family \( \set{\F_x}_{x \in P} \).
    Of course, not all data of this type gives rise to a matching family. 
\end{rmk}

\begin{rmk}
    Thus, a composition structure \( C \) is a stricter \( \omega \)\nbd category if for all molecules \( U \), all \( U \)\nbd matching families in \( C \) have a unique amalgamation.
\end{rmk}

\begin{lem}\label{lem:at_most_one_lift}
    Let \( C \) be a composition structure, \( P \) be a regular directed complex, and \( \set{\F_x}_{x \in P} \) be a matching family. 
    Then \( \set{\F_x}_{x \in P} \) has at most one amalgamation.
\end{lem}
\begin{proof}
    Immediate by Lemma \ref{lem:strict_functor_determined_by_basis}.
\end{proof}

\begin{comm} \label{comm:well_defined_amalgamation}
    Given a \( P \)\nbd matching family \( \set{\F_x} \) in \( C \), we thus have a candidate amalgamation \( \F \colon \molecin{P} \to C \) defined by sending an element \( \mapel{x} \in \atomin{P} \) to \( \pcell{\F_x} \).
    Then, using induction on submolecules (a variant of \cite[Comment 4.1.7]{hadzihasanovic2024combinatorics}), to show that \( \F \) is a well defined strict functor, we may take an arbitrary element \( w \colon W \to P \) in \( \molecin{P} \), and prove that \( \F \after \molecin{w} \) is well defined under the hypothesis that for all proper subdiagrams \( w' \) of \( w \), \( \F \after \molecin{w'} \) is well defined. 
    The base case on subdiagrams \( w' \) of \( w \) of dimension \( 0 \) is always true in that case.
    Here, ``well-defined'' means that the value of \( \F(w) \) is independent of the chosen decomposition \( w = w_1 \cp{k} w_2 \).
\end{comm}

\begin{lem} \label{lem:stricter_iff_local_wrt_pasting}
    Let \( C \) be composition structure.
    The following are equivalent.
    \begin{enumerate}
        \item \( C \) is a stricter \( \omega \)\nbd category;
        \item for all regular directed complexes \( P \), \( C \) is local with respect to \( s_P \);
        \item for all pairs of molecules \( U, V \) and \( k \geq 0 \) such that \( U \cp{k} V \) is defined, each lifting problem
        \begin{center}
            \begin{tikzcd}
                {\molecin{U} \cup \molecin{V}} & C \\
                {\molecin{(U \cp{k} V)}}
                \arrow[from=1-1, to=1-2]
                \arrow[from=1-1, to=2-1]
            \end{tikzcd}
        \end{center}
        has a (necessarily unique) solution.
    \end{enumerate}
\end{lem}
\begin{proof}
    Suppose that \( C \) is stricter, consider a regular directed complex \( P \) and a \( P \)\nbd matching family \( \set{\F_x \colon \molecin{\imel{P}{x}} \to C}_{x \in P} \).
    Then for all element \( w \colon W \to P \), \( \set{\F_x}_{x \in W} \) defines \( W \)\nbd matching family in \( C \), whose amalgamation is \( \F \after \molecin{w} \).
    This shows that \( \amalg \F_x \) is a well defined strict functor.
    Conversely, if \( C \) is local with respect to all the strict functors \( s_P \) where \( P \) is a regular directed complex, then it is also the case for all the functor \( s_U \) where \( U \) is a molecule.
    This shows the first two conditions are equivalent.
    Finally, the last condition is clearly necessary, since any functor \( \molecin{U} \cup \molecin{V} \to C \) defines in particular a \( (U \cp{k} V) \)\nbd matching family in \( C \).
    Conversely, we show it is sufficient.
    Let \( P \) be a regular directed complex.
    We show that \( C \) is local with respect to \( s_p \).
    Let \( \set{\F_x} \) be a \( P \)\nbd matching family in \( C \).
    We show that the candidate amalgamation \( \F \colon \molecin{P} \to C \) is a strict functor as per Comment \ref{comm:well_defined_amalgamation}.
    Let \( w \colon W \to P \) in \( \molecin{P} \), and suppose that \( \F \after \molecin{w'} \) is well defined for all proper submolecules \( w' \) of \( w \).
    Then either \( w \) is in \( \atomin{P} \), in which cases we are done since \( \F \after \molecin{w} = \F_x \) for some \( x \in P \), or \( w = w_1 \cp{k} w_2 \) for some decomposition \( W = W_1 \cp{k} W_2 \).
    Then, by inductive hypothesis, we have a strict functor \( (\F \after \molecin{w_1}, \F \after \molecin{w_2}) \colon \molecin{W_1} \cup \molecin{W_2} \to C \).
    By hypothesis, this extends to a strict functor \( \F' \colon \molecin{(W_1 \cp{k} W_2)} \to C \), which is equal to \( \F \after \molecin{w} \) by Lemma \ref{lem:strict_functor_determined_by_basis}.
    This shows that \( \F \after \molecin{w} \) is well defined and concludes the proof.
\end{proof}

\begin{prop} \label{prop:regular_directed_complex_stricter}
    Let \( P \) be a regular directed complex.
    Then \( \molecin{P} \) is a stricter \( \omega \)\nbd category.
\end{prop}
\begin{proof}
    Let \( Q \) be a regular directed complex, and consider a \( Q \)\nbd matching family \( \set{\F_x \colon \molecin{\imel{Q}{x}} \to \molecin{P}} \) in \( \molecin{P} \).
    We want to show that the candidate amalgamation \( \F \colon \molecin{Q} \to \molecin{P} \) is well defined, but for each \( w \colon W \to Q \) in \( \F(w) \) is given by the canonical morphism \( \colim_{x \in W} \pcell{\F_{w(x)}} \to P \), which is independent of the chosen decomposition of \( w \).
    This concludes the proof.
\end{proof}

\noindent Therefore, the functor \( \molecin{-} \colon \rdcpx \to \Comp \) factors through the subcategory \( \somegaCat \).

\begin{cor} \label{cor:regular_directed_complex_colimit_of_itself}
    Let \( P \) be a regular directed complex.
    Then in \( \somegaCat \),
    \begin{equation*}
        s_p \colon \colim_{x \in P} \molecin{\imel{P}{x}} \cong \molecin{P}.
    \end{equation*}
\end{cor}
\begin{proof}
    By the Yoneda Lemma and Lemma \ref{lem:stricter_iff_local_wrt_pasting}, \( \rc(s_P) \) is an isomorphisms in \( \somegaCat \).
    Since \( \rc \) is left adjoint, we conclude by Proposition \ref{prop:regular_directed_complex_stricter}.
\end{proof}

\begin{cor} \label{cor:molecin_preserves_pushout_inclusions}
    The functor \( \molecin{-} \colon \rdcpx \to \somegaCat \) preserves all pushouts of inclusions. 
\end{cor}

\begin{dfn} [Pasting in a stricter \( \omega \)\nbd category]
    Let \( C \) be a stricter \( \omega \)\nbd category, consider two pasting diagrams \( \F \colon \molecin{U} \to C \), \( \G \colon \molecin{V} \to C \), and \( k \geq 0 \) such that \( \bd{k}{+} \F = \bd{k}{-} \G \).
    Then we write \( \F \cp{k} \G \colon \molecin{(U \cp{k} V)} \to C \) for the strict functor determined by the universal properties of pushout given by Corollary \ref{cor:molecin_preserves_pushout_inclusions}.
    More generally, if a generalised pasting \( U \gencp{k} V \) given by a span \( (i \colon U \cap V \incl U, j \colon U \cap V \incl V) \) is defined and such that \( \F \after \molecin{i} = \G \after \molecin{j} \), we write \( \F \gencp{k} \G \) for the strict functor determined by universal property of the pushout.
    If the generalised pasting is given by a pasting at a submolecules \( U \cpsub{\iota} V \) or \( V \subcp{\iota} V \), we write accordingly \( \F \cpsub{\iota} \G \) and \( \G \subcp{\iota} \F \).
\end{dfn}

\begin{dfn} [Stricter complex]
    We say that a stricter \( \omega \)\nbd category is a \emph{stricter regular complex} if it is of the form \( \molecin{P} \) for some regular directed complex \( P \), and let \( \omegaReg \) the full subcategory of \( \somegaCat \) on stricter regular complexes, and \( \omegaMol \) its full subcategory on stricter regular complexes \( \molecin{U} \) where \( U \) is a molecule.
\end{dfn}

\begin{dfn}
    Let \( X \colon \opp{\omegaMol} \to \Set \) be an \( S \)\nbd local presheaf.\ccnote{define local presheaf}
    The \emph{stricter \( \omega \)\nbd category associated to \( X \)} is the composition structure 
    \begin{equation*}
        C \eqdef \coprod_{n \geq 0} X(\dglobe{n}),
    \end{equation*}
    whose boundary operators are induced by the strict functor
    \begin{equation*}
        \bd{k}{\a} \colon \molecin{\dglobe{k}} \to \molecin{\dglobe{n}}, 
    \end{equation*}
    and \( k \)\nbd composition operation is induced by the strict functor
    \begin{equation*}
        \molec{\dglobe{n}} \to \molecin{(\dglobe{n} \cp{k} \dglobe{n})}.
    \end{equation*}
\end{dfn}

\begin{rmk}\label{rmk:local_presheaf_defines_stricter}
    The composition structure \( C \) is indeed a stricter \( \omega \)\nbd category, by definition of the locality with respect to \( S \).
    Furthermore, any natural transformation \( f \colon X \to Y \) between \( S \)\nbd local presheaves induces a strict functor between the associated stricter \( \omega \)\nbd categories.
    This defines a functor
    \begin{equation*}
        [\opp{\omegaMol}, \Set] \to \somegaCat.
    \end{equation*}
\end{rmk}

\begin{prop} \label{prop:stricter_cat_are_local_presheaves}
    The category of \( S \)\nbd local presheaves over \( \omegaMol \) is equivalent to the category of stricter \( \omega \)\nbd categories.
\end{prop}
\begin{proof}
    We construct an inverse up to natural isomorphism to the functor of Remark \ref{rmk:local_presheaf_defines_stricter}.
    Let \( C \) be a stricter \( \omega \)\nbd category.
    Then the presheaf defined by \(\molecin{U} \mapsto \somegaCat(\molecin{U}, C) \) is \( S \)\nbd local by definition. 
    One checks directly that this functor is the desired inverse up to natural isomorphism.
\end{proof}

\begin{cor} \label{cor:diagrams_are_dense} 
    Let \( C \) be a stricter \( \omega \)\nbd category.
    Then the canonical strict functor
    \begin{equation*}
        \phi \colon \colim_{u \in \Dgm(C)} \molecin{U} \to C,
    \end{equation*}
    is an isomorphism.
    That is, the category \( \omegaMol \) is dense in \( \somegaCat \).
\end{cor}
\begin{proof}
    Follows from the density formula for presheaves and Proposition \ref{prop:stricter_cat_are_local_presheaves}.
\end{proof}

\begin{dfn} [Stricter \( n \)\nbd category]
    Let \( n \geq 0 \).
    A \emph{\( n \)\nbd composition structure} is a composition structure \( C \) such that for all globular cells \( c \in C \), we have \( \dim c \le n \).
    If \( C \) was a stricter \( \omega \)\nbd category, we speak of \emph{stricter \( n \)\nbd category}.
    We denote by \( \nComp{n} \) and \( \snCat{n} \) the full subcategories of \( \Comp \) and \( \somegaCat \) on \( n \)\nbd composition structures and stricter \( n \)\nbd categories, respectively. 
\end{dfn}

\begin{dfn} 
    The inclusion \( \iota_n \colon \nComp{n} \incl \Comp \) has a right adjoint \( \skel{n} \) defined by
    \begin{equation*}
        \skel{n}(C) \eqdef \set{c \in \C \mid \dim c \le n},
    \end{equation*}
    and a left adjoint \( \trunc{n} \) defined by
    \begin{equation*}
        \trunc{n}(C) \eqdef \skel{n - 1}(C) \cup \set{[c] \mid c \in C, \dim c = n},
    \end{equation*}
    where \( [-] \) denote the equivalence class on the globular cells of \( C \) of dimension \( n \) generated by \( \bd{}{-} d \sim \bd{}{+} d \) for all globular cells \( d \) of dimension \( n + 1 \). 
    By convention, \( \skel{-1}(C) = \emptyset \).
\end{dfn}

\begin{rmk}
    By \cite[Proposition 5.2.14]{hadzihasanovic2024combinatorics}, if \( P \) is a regular directed complex, \( \skel{n} \molecin{P} \) is naturally isomorphic to \( \molecin{(\skel{n}P)} \).
\end{rmk}

\begin{lem} \label{lem:stricter_n_iff_local_with_dim_le_n}
    Let \( C \) be an \( n \)\nbd composition structure.
    The following are equivalent.
    \begin{enumerate}
        \item \( C \) is a stricter \( n \)\nbd category;
        \item for all regular directed complex with \( \dim P \le n \), \( C \) is local with respect to \( s_P \);
        \item for all pairs of molecules \( U, V \) with \( \dim U, \dim V \le n \) and \( k \geq 0 \) such that \( U \cp{k} V \) is defined, each lifting problem
            \begin{center}
                \begin{tikzcd}
                    {\molecin{U} \cup \molecin{V}} & C \\
                    {\molecin{(U \cp{k} V)}}
                    \arrow[from=1-1, to=1-2]
                    \arrow[from=1-1, to=2-1]
                \end{tikzcd}
            \end{center}
            has a (necessarily unique) solution.
    \end{enumerate}
\end{lem}
\begin{proof}
    Suppose \( C \) is a stricter \( n \)\nbd category, then by Lemma \ref{lem:stricter_iff_local_wrt_pasting}, the last two conditions holds.
    Now suppose the second condition holds, consider a regular directed complex \( P \) and a \( P \)\nbd matching family \( \set{\F_x \colon \molecin{\imel{P}{x}} \to C} \) with candidate amalgamation \( \F \).
    Restricting this matching family to \( x \in \gr{\le n}{P} \) and using the assumption, we have a an amalgamation 
    \begin{equation*}
        \gr{\le n}{\F} \colon \amalg_{x \in \gr{\le n}{P}} \F_x \colon \gr{\le n}{P} \to C.
    \end{equation*}
    Let \( x \in P \) with \( \dim x > n \).
    Then \( \dim \pcell{\F_x} \le n \), hence for any \( \a \in \set{-, +} \), \( \pcell{\bd{n}{\a} \F_x} = \pcell{\F(\bd{n}{\a} x \to P)} \).
    Using this fact, an induction on the submolecules of any \( w \colon W \to P \) shows that \( \F(w) = \gr{\le n}{\F}(\bd{n}{\a} w) \), proving that \( \F \) is well defined.
    This shows that \( C \) is stricter.
    For the last condition, one reason as in the proof of Lemma \ref{lem:stricter_iff_local_wrt_pasting}.
\end{proof}

\begin{lem} \label{lem:truncation_stricter_are_stricter}
    Let \( n \geq 0 \), and \( C \) be a stricter \( \omega \)\nbd category.
    Then \( \skel{n}(C) \) and \( \trunc{n}(C) \) are stricter \( n \)\nbd categories.
\end{lem}
\begin{proof}
    We use the second point of Lemma \ref{lem:stricter_n_iff_local_with_dim_le_n}.
    Let \( P \) be a regular directed complex with \( \dim P \le n \), and \( \set{\F_x \colon \molecin{\imel{P}{x}} \to \skel{n}(C)}_{x \in P} \) be a \( P \)\nbd matching family in \( \skel{n}(C) \).
    Then post-composing each \( \F_x \) by the unit \( \skel{n}(C) \to C \), and using the fact that \( C \) is stricter, the amalgamation defines a strict functor \( \F \colon \molecin{P} \to C \).
    Then, since \( \skel{n}(P) = P \), \( \skel{n}(\F) \) is the desired amalgamation of the matching family.

    Now consider a matching family \( \set{\G_x \colon \molecin{\imel{P}{x}} \to \trunc{n}(C)} \).
    By definition of \( \trunc{n}(C) \), and since \( \skel{n - 1}(C) \subseteq \trunc{n}(C) \) is stricter, we have the amalgamation
    \begin{equation*}
        \gr{< n }{\G} \eqdef \amalg_{x \in \gr{< n}{P}} \G_x \colon \molecin{\gr{< n}{P}} \to \trunc{n}(C).
    \end{equation*}
    For each \( x \in P \) of dimension \( n \) such that, choose a representative \( u_x \colon W_x \to C \) of the cell \( \pcell{\G_x} \) in \( \trunc{n}(C) \) (if \( \dim \pcell{\G_x} \le n \), then we mean that we let \( u_x \eqdef \pcell{\G_x} \)).
    We claim that \( \set{u_x}_{x \in P} \) give rise to a matching family \( \set{\G'_x \colon \molecin{\imel{P}{x}} \to C}_{x \in P} \), as per Remark \ref{rmk:data_matching family}.
    This is clear for all \( x \in \gr{< n}{P} \), since this is the data associated to the matching family \( \set{\G_x}_{x \in \gr{< n}{P}} \). 
    Then, if \( \dim x = n \), then \( \G'_x \colon \molecin{\imel{P}{x}} \to C \) is a well defined strict functor.
    Indeed, \( \pcell{\G'_x} = u_x \), and for all \( k < n \) and \( \a \in \set{-, +} \) we have \( \bd{k}{\a} u_x = \gr{< n}{G}(\bd{k}{\a} x \incl P ) \).
    since \( \bd{k}{\a} \pcell{\G_x} \) is independent of the chosen representative of \( \pcell{\G_x} \).
    Since \( C \) is stricter, this defines a strict functor \( \G' \colon \molecin{P} \to C \). 
    Post-composing \( \G' \) with the counit \( \varepsilon_C \colon C \to \trunc{n}(C) \), we obtain the strict functor \( \varepsilon_C \after \G' \), which is the candidate amalgamation \( \G \) by Lemma \ref{lem:strict_functor_determined_by_basis}.
    This shows that \( \trunc{n}(C) \) is stricter, and concludes the proof.
\end{proof}

\begin{dfn}[\( n \)\nbd skeletong and \( n \)\nbd truncation.]
    Let \( n \geq 0 \) and \( C \) be a stricter \( n \)\nbd category.
    The \emph{\( n \)\nbd skeleton} of \( C \) is the stricter \( n \)\nbd category \( \skel{n}(C) \).
    The \emph{\( n \)\nbd truncation} of \( C \) is the stricter \( n \)\nbd category \( \trunc{n}(C) \).
\end{dfn}

\noindent Thus, the adjoint triple \( \trunc{n} \dashv \iota_n \dashv \skel{n} \) restricts the adjoint triple
\begin{center}
    \begin{tikzcd}
        {\snCat{n}} && \somegaCat.
        \arrow["{\iota_n}"{description}, from=1-1, to=1-3]
        \arrow["{\skel{n}}", shift left=2, curve={height=-12pt}, from=1-3, to=1-1]
        \arrow["{\trunc{n}}"', shift right=2, curve={height=12pt}, from=1-3, to=1-1]
    \end{tikzcd}
\end{center}
Notice that given an stricter \( \omega \)\nbd category \( C \), we have a chain of inclusions 
\begin{equation*}
    \skel{-1} C \incl \skel{0} C \incl \skel{1} C \incl \ldots \incl \skel{n} C \incl \ldots
\end{equation*}
whose colimit \( \somegaCat \) is \( C \).

We conclude this part with the notion of \emph{stricter polygraph}. 
The following definition is adapted from \cite[8.2.1]{hadzihasanovic2024combinatorics}.

\begin{dfn} [Cellular extension] \label{dfn:cellular_extension}
    Let \( C \) be a stricter \( \omega \)\nbd category.
    A \emph{cellular extension of \( C \)} is a stricter \( \omega \)\nbd category \( C_{\cls{S}} \) together with a pushout diagram 
    \begin{center}
        \begin{tikzcd}[column sep=large]
            {\coprod_{e \in \cls{S}} \bd{}{}U_e} & {\coprod_{e \in \cls{S}} \molecin{U_e}} \\
            C & {C_{\cls{S}}}
            \arrow[""{name=0, anchor=center, inner sep=0}, "{\molecin{\bd{e}{}}}", from=1-1, to=1-2]
            \arrow["{(\bd{}{}e)_{e \in \cls{S}}}"', from=1-1, to=2-1]
            \arrow["{(e)_{e \in \cls{S}}}", from=1-2, to=2-2]
            \arrow[from=2-1, to=2-2]
            \arrow["\lrcorner"{anchor=center, pos=0.125, rotate=180}, draw=none, from=2-2, to=0]
        \end{tikzcd}
    \end{center}
    in \( \somegaCat \), where, for each \( e \in \cls{S} \), \( U_e \) is an atom.
\end{dfn}

\begin{comm}
    This is a non-standard definition of cellular extension, which is a priori more general than the usual one, which requires each of the atoms \( U_e \) to be globes.
    However, by the following Lemma, we may turn every cellular extension in this sense into one in the restricted sense. 
    See \cite[Comment 8.2.2]{hadzihasanovic2024combinatorics}. 
\end{comm}

\begin{lem} \label{lem:pushout_principal_cell} \ccnote{problem of dependency, but in fact prove this result for all comap substitution, it will be needed}
    Let \( U \) be an atom of dimension \( n \), and \( s \colon \dglobe{n} \to U \) be the unique subdivision.
    Then the square
    \begin{center}
        \begin{tikzcd}
            {\molecin{(\bd{}{}\dglobe{n})}} & {\molecin{\dglobe{n}}} \\
            {\molecin{(\bd{}{}U)}} & {\molecin{U}}
            \arrow[from=1-1, to=1-2]
            \arrow["{\molecin{(\restr{s}{\bd{}{}\dglobe{n}})}}"', from=1-1, to=2-1]
            \arrow["{\molecin{s}}", from=1-2, to=2-2]
            \arrow[from=2-1, to=2-2]
        \end{tikzcd}
    \end{center}
    is a pushout square in \( \somegaCat \).
\end{lem}
\begin{proof}
    By \cite[Lemma 9.1.12]{hadzihasanovic2024combinatorics}, the square is a pushout in \( \omegaCat \).
    Since all the strict \( \omega \)\nbd categories involved are stricter, we can conclude by an application of the left adjoint functor \( \rcs \). 
\end{proof}

\begin{dfn} [Stricter polygraph]
    A \emph{stricter polygraph} is a stricter \( \omega \)\nbd category \( C \), together with, for each \( n \geq 0 \), a pushout diagram
    \begin{center}
        \begin{tikzcd}[column sep=large]
            {\coprod_{e \in \cls{S}_n} \bd{}{}U_e} & {\coprod_{e \in \cls{S}_n} \molecin{U_e}} \\
            \skel{n - 1}C & {\skel{n} C}
            \arrow[""{name=0, anchor=center, inner sep=0}, "{\molecin{\bd{e}{}}}", from=1-1, to=1-2]
            \arrow["{(\bd{}{}e)_{e \in \cls{S}_n}}"', from=1-1, to=2-1]
            \arrow["{(e)_{e \in \cls{S}_n}}", from=1-2, to=2-2]
            \arrow[from=2-1, to=2-2]
            \arrow["\lrcorner"{anchor=center, pos=0.125, rotate=180}, draw=none, from=2-2, to=0]
        \end{tikzcd}
    \end{center}
    in \( \somegaCat \), exhibiting the \( \skel{n}C \) as a cellular extension of \( \skel{n - 1}C \), and such that for each \( e \in \cls{S}_n \), \( \dim e = n \).
    The set \( \cls{S} = \coprod_{k \geq 0} \cls{S}_k \) is called the \emph{generating set} of \( C \).
    More generally, we say that a strict functor \( f \colon A \to C \) of composition structure is a \emph{relative stricter polygraph} if \( f \) can be obtain as the transfinite composition of cellular extensions of \( A \).
\end{dfn}

\begin{lem} \label{lem:stricter_regular_complex_are_stricter_polygraph}
    Let \( P \) be a regular direct complex.
    Then \( \molecin{P} \) is a stricter polygraph
\end{lem}
\begin{proof}
    By Corollary \ref{cor:molecin_preserves_pushout_inclusions}, the square
    \begin{center}
        \begin{tikzcd}
            {\coprod_{x \in \gr{n}P} \molecin{\bd{}{}\imel P x}} & {\coprod_{x \in \gr{n}P} \molecin{\imel P x}} \\
            {\skel{n - 1}P} & {\skel{n} P}
            \arrow["{{\molecin{\bd{e}{}}}}", from=1-1, to=1-2]
            \arrow["{{(\bd{}{}\molecin{\mapel x})_{x \in \gr n P}}}"', from=1-1, to=2-1]
            \arrow["{{(\molecin{\mapel x})_{x \in \gr n P}}}", from=1-2, to=2-2]
            \arrow[from=2-1, to=2-2]
        \end{tikzcd}
    \end{center}
    is a pushout in \( \somegaCat \).
    This concludes the proof.
\end{proof}

\subsection{Gray product of stricter \texorpdfstring{$\omega$}{}-categories}

Recall that if \( P, Q \) are regular directed complexes, the basis \( \atomin{(P \gray Q)} \) of the stricter regular complex \( \molecin{P \gray Q} \) is given exactly by the morphisms \( u \gray v \) for \( u \in \atomin{P} \) and \( v \in \atomin{Q} \).  

\begin{dfn} [Gray product stricter complexes] \label{dfn:gray_product_of_stricter_regular_complexes}
    Let \( P, Q \) be regular directed complexes.
    The \emph{Gray product} of \( \molecin{P} \) and \( \molecin{Q} \) is the stricter regular complex
    \begin{equation*}
        \molecin{P} \gray \molecin{Q} \eqdef \molecin{(P \gray Q)}.
    \end{equation*}
    If \( \F \colon \molecin{P} \to \molecin{P'} \) and \( \G \colon \molecin{Q} \to \molecin{P'} \) are two strict functors, we let 
    \begin{equation*}
        \F \gray \G \colon \molecin{P} \gray \molecin{Q} \to  \molecin{P'} \gray \molecin{Q'}
    \end{equation*}
    by sending a basis element \( u \gray v \) to \( \F(u) \gray \G(v) \).
\end{dfn}

\begin{prop} \label{prop:gray_stricter_regular_complex_monoidal}
    The Gray product determines a monoidal structure on the category \( \omegaReg \), whose monoidal unit is the terminal stricter \( \omega \)\nbd category \( \globe{0} \).
\end{prop}
\begin{proof}
    Let \( \F \colon \molecin{P} \to \molecin{P'} \) and \( \G \colon \molecin{Q} \to \molecin{Q'} \) be two strict functors, and \( w \colon W \to P \gray Q \) be a morphism.
    We show that \( \fun{H} \eqdef (\F \gray \G) \after \molecin{w} \) is well defined by induction on \( \dim w \), then on the subdiagrams \( w' \) of \( w \).
    The base cases where \( \dim w = 0 \) and \( w' \) is a point are clear.
    Suppose inductively that \( \dim w > 0 \) and that \( (\F \gray \G) \after \molecin{w'} \) is well defined for all proper subdiagrams \( w' \) of \( w \).
    Consider first the case where that \( W \) is an atom.
    Then by definition, \( \pcell{\F \gray \G} = \pcell{\F} \gray \pcell{\G} \).
    Then, for \( k \geq 0 \) and \( \a \in \set{-, +} \), we have
    \begin{align*}
        \bd{k}{\a} \pcell{\F \gray \G} &= \bd{k}{\a} (\pcell{\F} \gray \pcell{\G}) \\
                                       &= \bigcup_{i = 1}^k \pcell{\bd{i}{\a} \F} \gray \pcell{\bd{k - i}{(-)^i\a} \G} \\
                                       &= \pcell{\bigcup_{i = 1}^k \bd{i}{\a} \F \gray \bd{k - i}{(-)^i\a \G}} \\
                                       &= \pcell{\bd{k}{\a} (\F \gray \G)},
    \end{align*}
    where we used the inductive hypothesis in the case \( k < \dim W \).
    Then, consider an element \( u \in \molecin{W} \) which is not the principal cell. 
    Necessarily \( \dim u < \dim w \), thus
    \begin{align*}
        \bd{k}{\a} \fun{H}(u) &= \bd{k}{\a} \pcell{(\F \gray \G) \after \molecin{(w \after u)}} \\
                                                         &= \pcell{\bd{k}{\a} ((\F \gray \G) \after \molecin{(w \after u)})} \\
                                                         &= \fun{H}(\bd{k}{\a} u).
    \end{align*}
    This shows that \( (\F \gray \G) \after w \) is a morphism of the underlying reflexive globular graph. 
    Now let \( u, u' \in \molecin{W} \) and \( 0 \le k \le \dim u, \dim u' \) such that \( u \cp{k} u' \) is defined.
    Since \( \atomin{W} \) is a basis for \( \molecin{W} \), we get that \( \dim (u \cp{k} u') < n \), and thus by inductive hypothesis, 
    \( (\F \gray \G)  \after \molecin{w \after (u \cp{k} u')} \) is well defined.
    Then,
    \begin{align*}
        ((\F \gray \G)  \after \molecin{w})(u \cp{k} u') &= \pcell{\fun{H} \after \molecin{(u \cp{k} u')}} \\
                                                         &= \pcell{\fun{H} \after \molecin{u}} \cp{k} \pcell{\fun{H} \after \molecin{u'}} \\
                                                         &= \fun{H}(u) \cp{k} \fun{H}(u').
    \end{align*}
    This shows that \( \fun{H} \) is a strict functor.
    Finally, suppose \( w \) is not an atom.
    Given \( x \in W \), we have \( (a, b) \in P \gray Q \) such that \( w(x) = (a, b) \).
    By inductive hypothesis on the subdiagrams, 
    \begin{equation*}
        \fun{H}_x \eqdef (\F \after \molecin{\mapel{a}}) \gray (\G \after \molecin{\mapel{b}}) \colon \molecin{(\imel{P}{a} \gray \imel{Q}{b})} \to \molecin{(P' \gray Q')}
    \end{equation*}
    is a strict functor, and the collection \( \set{\fun{H}_x}_{x \in W} \) is a \( W \)\nbd matching family in \( \molecin{(P' \gray Q')} \), whose amalgamation is \( \fun{H} \), which is therefore a well defined strict functor.
    This shows that \( \F \gray \G \) is a strict functor.
    Since for all regular directed complexes \( P \), \( \pt \gray P = P = P \gray \pt \), we deduce that the monoidal unit is \( \molecin{\pt} = \globe{0} \).
    Functoriality of \( - \gray - \) is straightforwards.
    This concludes the proof.
\end{proof}

\begin{rmk}
    If \( \F \colon \molecin{P} \to \molecin{P'} \) and \( \G \colon \molecin{Q} \to \molecin{P'} \) are strict functors, \( u \in \molecin{P} \), and \( v \in \molecin{Q} \), then 
    \begin{equation*}
       (\F \gray \G)(u \gray v) = \F(u) \gray \G(v). 
    \end{equation*}
\end{rmk}

\begin{rmk}
    Since the point is a molecule, and the Gray product of two molecules is a molecule, the category \( \omegaMol \) inherit from \( \omegaReg \) of the monoidal structure given by the Gray product. 
\end{rmk}

\begin{dfn} 
    Let \( P \) be a regular directed complex and \( C \) be a stricter \( \omega \)\nbd category.
    By \cite[Lemma 7.2.8]{hadzihasanovic2024combinatorics}, the presheaf on \( \omegaMol \) given by
    \begin{equation*}
        \molecin{U} \mapsto \somegaCat(\molecin{(P \gray U)}, C) 
    \end{equation*}
    is \( S \)\nbd local.
    By Proposition \ref{prop:stricter_cat_are_local_presheaves}, this defines the stricter \( \omega \)\nbd category 
    \begin{equation*}
        \homlax(\molecin{P}, C),
    \end{equation*}
    whose \( n \)\nbd cells are strict functors \( \F \colon \molecin{(P \gray \dglobe{n})} \to C \).
    Dually, we define the stricter \( \omega \)\nbd category
    \begin{equation*}
        \homcolax(\molecin{P}, C),
    \end{equation*}
    whose \( n \)\nbd cells are strict functors \( \F \colon \molecin{(\dglobe{n} \gray P)} \to C \).
\end{dfn}

\begin{lem}
    Let \( U, V \) be molecules, \( C \) be a stricter \( \omega \)\nbd category.
    Then there are bijections
    \begin{align*}
        \somegaCat(\molecin{(U \gray V)}, C) &\cong \somegaCat(\molecin{U}, \homcolax(\molecin{V}, C)) \\
                                             &\cong \somegaCat(\molecin{V}, \homlax(\molecin{U}, C)),
    \end{align*}
    natural in \( \molecin{U}, \molecin{V} \) and \( C \).
\end{lem}
\begin{proof}
    Follows directly by Proposition \ref{prop:stricter_cat_are_local_presheaves} and the Yoneda Lemma. 
\end{proof}

Applying \cite[Th\'eor\`eme 5.3]{ara2020joint} together with the previous result, and Corollary \ref{cor:diagrams_are_dense}, we get the following definition.
\begin{dfn} [Gray product of stricter \( \omega \)\nbd categories]
    Let \( C, D \) be stricter \( \omega \)\nbd categories.
    The Gray product of \( C \) and \( D \) is the stricter \( \omega \)\nbd category
    \begin{equation*}
        \colim_{\substack{u \in \Dgm(C),\\ v \in \Dgm(D)}} \molecin{(U \gray V)}.
    \end{equation*}
    This determines a biclosed monoidal structure whose monoidal unit is the terminal stricter \( \omega \)\nbd category \( \globe{0} \) and such that the inclusion
    \begin{equation*}
        \omegaMol \to \somegaCat
    \end{equation*}
    is strong monoidal.
\end{dfn}

\begin{rmk}
    Let \( P, Q \) be regular directed complexes.
    Since \( - \gray - \) is biclosed and using Corollary \ref{cor:regular_directed_complex_colimit_of_itself}, we get
    \begin{align*}
        \molecin{P} \gray \molecin{Q} &\cong \colim_{x \in P, y \in Q} \molecin{\imel{P}{x}} \gray \molecin{\imel{Q}{y}} \\
                                      &\cong \colim_{(x, y) \in P \gray Q} \molecin{(\imel{P}{x} \gray \imel{Q}{y})} \\
                                      &= \molecin{(P \gray Q)}.
    \end{align*}
    Thus the notation is consistent with Definition \ref{dfn:gray_product_of_stricter_regular_complexes}, so that the inclusion
    \begin{equation*}
        \omegaReg \incl \somegaCat
    \end{equation*}
    is also strong monoidal.
\end{rmk}

\subsection{Strict and stricter categories}

The goal of this section is to clarify the relationship between strict and stricter categories.

\begin{prop} \label{prop:stricter_are_strict}
    Let \( C \) be a stricter \( \omega \)\nbd category.
    Then \( C \) is a strict \( \omega \)\nbd category.
\end{prop}
\begin{proof}
    Recall from \cite[Theorem 5.2.5]{hadzihasanovic2024combinatorics} that pasting and boundaries makes the collection of molecules a strict \( \omega \)\nbd categories. 
    We first show the axioms of interaction between pasting and boundaries. 
    Let \( c \colon O^{m} \to C \) and \( d \colon O^{m'} \to C \) be \( k \)\nbd composable cells in \( C \).
    The pasting diagram \( c \cp{k} d \colon \molecin{(O^m \cp{k} O^{m'})} \) is such that \( \pcell{c \cp{k} d} = \pcell{c} \comp{k} \pcell{d} \).
    Let \( n \geq 0 \) and \( \a \in \set{-, +} \), then
    \begin{equation*}
         \bd{n}{\a} (\pcell{c} \comp{k} \pcell{d}) = \bd{n}{\a} \pcell{c \cp{k} d} = \pcell{\bd{n}{\a} (c \cp{k} d)},
    \end{equation*}
    the latter being equal to
    \begin{equation*}
        \begin{cases}
            \pcell{\bd{n}{\a} c} = \bd{n}{\a} \pcell{c} = \bd{n}{\a} \pcell{d} & \text{if } n < k,\\
            \pcell{\bd{k}{-}c} = \bd{k}{-}\pcell c & \text{if } n = k, \a = -,\\
            \pcell{\bd{k}{+}d} = \bd{k}{+}\pcell d & \text{if } n = k, \a = +,\\
            \pcell{\bd{k}{\a}c \cp{k} \bd{k}{\a} d} = \bd{k}{\a}\pcell c \comp{k} \bd{k}{\a} \pcell d & \text{if } n > k.
        \end{cases}
    \end{equation*}
    Next is unitality.
    Let \( c \colon O^m \to C \) be a cell, and \( k \geq 0 \).
    Then we have the pasting diagram \( c \cp{k} \bd{k}{+} c \colon \molecin{(O^m \cp{k} \bd{k}{+} O^m)} \to C \).
    Since \( (O^m \cp{k} \bd{k}{+} O^m) = O^m \), the principal cell of \( c \cp{k} \bd{k}{+} c \) is \( \pcell{c} \).
    This shows that \( \pcell{c} \comp{k} \bd{k}{+} \pcell{c} = \pcell{c} \).
    Similarly, \(  \bd{-}{k} c \comp{k} \pcell{c} = \pcell{c} \).
    Next, we consider the axiom of associativity. 
    Let \( c, d, e \) be \( k \)\nbd composable cell in \( C \).
    Then again, we have a pasting diagram \( c \cp{k} d \colon \molecin{(\dglobe{m} \cp{k} \dglobe{m'})} \to \C \).
    Then we may paste \( c \cp{k} d \) and \( e \) to obtain a pasting diagram 
    \begin{equation*}
        (c \cp{k} d) \cp{k} e \colon \molecin{(\dglobe{m} \cp{k} \dglobe{m'} \cp{k} \dglobe{m''})} \to C,
    \end{equation*}
    whose principal cell is \( (\pcell c \comp{k} \pcell d) \comp{k} \pcell{e} \).
    Similarly, we have a pasting diagram
    \begin{equation*}
        c \cp{k} (d \cp{k} e) \colon \molecin{(\dglobe{m} \cp{k} \dglobe{m'} \cp{k} \dglobe{m''})} \to C,
    \end{equation*}
    whose principal cell is \( \pcell c \comp{k} (\pcell d \comp{k} \pcell{e}) \).
    By Lemma \ref{lem:strict_functor_determined_by_basis}, 
    \begin{equation*}
         (c \cp{k} d) \cp{k} e = c \cp{k} (d \cp{k} e),
    \end{equation*}
    hence their principal cells are equal as well.
    Proceed similarly for exchange.
    This concludes the proof.
\end{proof} 

Akin to stricter \( \omega \)\nbd categories, strict \( \omega \)\nbd categories are a reflective subcategory of \( \Comp \).
Thus by Proposition \ref{prop:stricter_are_strict}, we have a sequence of full subcategory inclusions
\begin{equation*}
     \somegaCat \incl \omegaCat \incl \Comp.
\end{equation*}
We define
\begin{equation*}
    \rcs \colon \omegaCat \to \somegaCat
\end{equation*}
to be the functor applying the reflector \( \rc \colon \Comp \to \somegaCat  \) to the underlying composition structure of a strict \( \omega \)\nbd category, which is left adjoint and exhibit \( \somegaCat \) as a reflective subcategory of \( \omegaCat \).

\cccom{TODO: say this gives a commutative diagram with the truncations}

\begin{thm}\label{thm:strict_le_3_are_stricter}
    Let \( n \le 3 \), and \( C \) be a strict \( n \)\nbd category.
    Then \( C \) is a stricter \( n \)\nbd category.
\end{thm}
\begin{proof}
    Let \( P \) be a regular directed complex with \( \dim P \le 3 \).
    By \cite[Corollary 8.4.12]{hadzihasanovic2024combinatorics}, \( \molecin{P} \) is a polygraph.
    Thus the map
    \begin{equation*}
        s_P \colon \colim_{x \in P} \imel{P}{x} \to P 
    \end{equation*}
    is an isomorphism in \( \omegaCat \).
    In particular, any strict \( n \)\nbd category \( C \) is local with respect to \( s_P \).
    By Lemma \ref{lem:stricter_n_iff_local_with_dim_le_n}, this concludes the proof.
\end{proof}

\cccom{TODO: strict and stricter polygraph: say polygraph with basis gives stricter with basis, and stricter with basis reconstruct a polygraph with basis that strictifies to itself}

\cccom{TODO: omega-cat are the local presheaves for a certain subset of Theta, might even close it under Gray product to give an equivalent definition of the Gray product, uses amar's book on Steiner theory to get a quick proof of the following fact.}

Now recall from \cite[Appendice A]{ara2020joint} that strict \( \omega \)\nbd categories also support a Gray product, defined along a similar method.

\begin{prop} \label{prop:reflection_to_stricter_monoidal}
    The functor \( \rcs \colon \omegaCat \to \somegaCat \) is strong monoidal with respect to the Gray product on stricter and strict \( \omega \)\nbd categories.
\end{prop}
\begin{proof}
    \cccom{TODO}
\end{proof}

\subsection{Suspension of stricter \texorpdfstring{$\omega$}{}-categories}

\begin{dfn} 
    Let \( C \) be a composition structure, and \( a, b \) be \( 0 \)\nbd dimensional globular cells.
    We define the composition structure
    \begin{equation*}
        C(a, b) \eqdef \set{u \in C \mid \bd{0}{-} u = a, \bd{0}{+} u = b},    
    \end{equation*}
    whose boundary operators and \( k \)\nbd composition are induced by the one of \( C \) shifted by \( 1 \).
\end{dfn}

\begin{lem} \label{lem:hom_of_stricter_is_stricter}
    Let \( C \) be a stricter \( \omega \)\nbd category, and \( a, b \) be two \( 0 \)\nbd dimensional globular cells of \( C \).
    Then \( C(a, b) \) is a stricter \( \omega \)\nbd category.
\end{lem}
\begin{proof}
    Let \( U \) be a molecule.
    Any \( U \)\nbd matching family 
    \begin{equation*}
        \set{\F_x \colon \molecin{\imel{U}{x}} \to \C(a, b)}_{x \in U}
    \end{equation*}
    determines a \( \sus{U} \)\nbd matching family 
    by letting \( \F_{\sus{x}} \eqdef \F_x \) for \( x \in U \), and \( \F_{\bot^-}, \F_{\bot^+} \) be the strict functors from \( \molecin{\pt} \) classifying \( a \) and \( b \) respectively. 
    This matching family has an amalgamation \( \sus{\F} \colon \molecin{\sus{U}} \to C \), showing that the amalgamation of \( \set{\F_x}_{x \in U} \) is well defined.
\end{proof}

\begin{comm}
    Thus, any stricter \( \omega \)\nbd category can canonically be seen as a category enriched in stricter \( \omega \)\nbd categories.
    The converse is not true. \ccnote{add a ref here to the counter example of a 4-stricter which is not strict, but is strict enriched in stricter}
    We do not know any general condition for a category enriched in a stricter \( \omega \)\nbd category to be itself a stricter \( \omega \)\nbd category.
    We know however from Proposition \ref{prop:stricter_are_strict} that it is at least a strict \( \omega \)\nbd category.
\end{comm}

\begin{dfn} [Suspension]
    Let \( C \) be a composition structure.
    The \emph{suspension of \( C \)} is the composition structure \( \sus{C} \) with two objects \( \bot^+, \bot^- \) and such that
    \begin{equation*}
        \sus{C}(\bot^\a, \bot^\beta) \eqdef 
        \begin{cases}
            C       & \text{if } (\a, \beta) = (-, +),\\
            \set{*} & \text{if } \a = \beta,\\
            \varnothing & \text{else.}
        \end{cases}
    \end{equation*}
    The boundary operator and \( k \)\nbd composition operation are induced by the one of \( C \). 
\end{dfn}

\noindent If \( C \) is a stricter \( \omega \)\nbd category, then \( \sus{C} \) is a strict \( \omega \)\nbd category, since it is the suspension of its underlying strict \( \omega \)\nbd category.
The goal of this section is to show that it is in fact a stricter \( \omega \)\nbd category. 

\begin{dfn} [Collapsible closed subset]
    Let \( U \) be a molecule, \( K \subseteq U \) be a closed subset, and \( \beta \in \set{-, +} \).
    We say that \( K \) is \emph{\( \beta \)\nbd collapsible} if \( K \) is non-empty and for all \( x \in U \), if \( \bd{0}{-\beta} x \in K \) then \( x \in K \).
    In that case, we let \( \coll K U \) be oriented graded poset whose underlying set is
    \begin{equation*}
        \set{\zcoll} \coprod \set{\icoll u \mid u \in U \setminus K},
    \end{equation*}
    and oriented covering diagram given by
    \begin{equation*}
        \cofaces{}{\a} x \eqdef
        \begin{cases}
            \set{ \icoll v \mid v \in \cofaces{}{\a} u} & \text{if } x = \icoll u,\\
            \set{ \icoll u \mid \dim u = 1, \faces{}{\a} u \in K} &\text{if } x = \bullet, \a = \beta\\
            \varnothing &\text{else.}
        \end{cases}
    \end{equation*}
    The underlying poset of \( \coll{K}{U} \) fits into the commutative square
    \begin{equation} \label{tik:square_collapse}
        \begin{tikzcd} 
            K & \pt \\
            U & {\coll K U}
            \arrow[""{name=0, anchor=center, inner sep=0}, two heads, from=1-1, to=1-2]
            \arrow[hook, from=1-1, to=2-1]
            \arrow[hook, from=1-2, to=2-2]
            \arrow[two heads, from=2-1, to=2-2]
            \arrow["\lrcorner"{anchor=center, pos=0.125, rotate=180}, draw=none, from=2-2, to=0]
        \end{tikzcd}
    \end{equation}
    where \( \pt \to \coll{K}{U} \) picks out the element \( \zcoll \), and \( \mapcoll \colon U \to \coll{K}{U} \) is the order preserving map defined by
    \begin{equation*}
        x \mapsto
        \begin{cases}
            \icoll x & \text{if } x \in U \setminus K\\
            \zcoll & \text{otherwise.}
        \end{cases}
    \end{equation*}
\end{dfn}

\begin{lem} \label{lem:collapsible_is_puhsout}
    Let \( U \) be a molecule, \( \beta \in \set{-, +} \) and \( K \subseteq U \) be a \( \beta \)\nbd collapsible subset.
    Then the commutative square (\ref{tik:square_collapse}) is a pushout in the category of posets.
\end{lem}
\begin{proof}
    Since the underlying square is a pushout of set, for the statement to be true, we claim that it is enough that for every element \( u \in U \setminus K \) covering an element \( k \) of \( K \), then either \( \dim u = 1 \), or there exists \( v \in U \setminus K \) such that \( v < u \) and \( v \) covers an element of \( K \).
    Indeed, in that case, the elements covering \( \zcoll \) in \( \coll{K}{U} \) with the universal partial order will be exactly of the form \( \icoll{u} \) for \( u \in U \setminus K \) with \( \dim u = 1 \). 
    Therefore, suppose that \( u \in U \setminus K \) of dimension \( n \geq 2 \) covers an element \( k \) of \( K \).
    Recall that in the augmentation of \( U \), we have a chain of covering \( u_{-1} \eqdef \bot \to u_0 \to u_1 \to \ldots \to u_n \eqdef u \) such that \( u_i \in \faces{}{\beta} u_{i + 1} \) for all \( - 1 \le i < n \) and \( \set{u_0} = \bd{0}{\beta} u \).
    Then none of the \( u_i \) belong to \( K \), since otherwise, \( u_0 \) would be in \( K \), hence so would be \( u \).
    Then pick any chain \( k_{- 1} = \bot \to k_1 \to \ldots \to k_{n - 1} \eqdef k \) from \( \bot \) to \( k \), which necessarily belong to \( K \), since it is closed.
    Then we have in the covering diagram of \( \augm{U} \) 
    \begin{center}
        \begin{tikzcd}
            & u \\
            {k_{n - 1}} && {u_{n - 1}} \\
            {k_1} && {u_0} \\
            & \bot.
            \arrow[no head, from=1-2, to=2-1]
            \arrow[no head, from=1-2, to=2-3]
            \arrow[dashed, no head, from=2-1, to=3-1]
            \arrow[dashed, no head, from=2-3, to=3-3]
            \arrow[no head, from=3-1, to=4-2]
            \arrow[from=3-3, to=4-2]
        \end{tikzcd}
    \end{center}
    By Lemma \ref{lem:diamond_transitive}, there exists a sequence of diamonds
    \begin{center}
        \begin{tikzcd}
            & {x_i} \\
            {y_i} && {y'_i} & {0 \le i \le r } \\
            & {z_i}
            \arrow[no head, from=1-2, to=2-1]
            \arrow[no head, from=1-2, to=2-3]
            \arrow[no head, from=2-1, to=3-2]
            \arrow[no head, from=2-3, to=3-2]
        \end{tikzcd}
    \end{center}
    rewriting the left chain to the right chain.
    Let \( i_0 \eqdef \max \set{i \mid y_i \in K} \), which exists since \( y_1 \in K \).
    If \( x_{i_0} < u \), then we are done since \( x_{i_0} \) covers \( y_{i_0} \in K \).
    Else \( x_{i_0} = u \).
    Then, either \( i_0 = r \), in which case \( y'_{i_0} = u_j \) for some \( j \), or \( i_0 < r \), in which case \( y'_{i_0} = y_{i + 1} \notin K \).
    In any case, \( y'_{i_0} \) is not in \( K \).
    Since \( y_{i_0} \in K \) is of dimension \( \geq 1 \) and \( K \) is closed, \( z_{i_0} \in K \).
    Thus we found a \( y'_{i_0} < u \) which is not in \( K \) and covers \( z_{i_0} \in K \).
    This concludes the proof.
\end{proof}


\begin{lem} \label{lem:path_from_zero_bd_to_all_points}
    Let \( U \) be a molecule, \( \a \in \set{-, +} \) and \( x \in \gr{0}{U} \).
    Then there exists \( k \geq 0 \) and a inclusion \( \iota \colon k\dglobe{1} \incl U \) such that \( \bd{0}{\a} \iota = \bd{0}{\a} U \) and \( \bd{0}{-\a} \iota = x \).
\end{lem}
\begin{proof}
    We proceed by induction on the submolecules of \( V \) of \( U \).
    The base case where \( V \) is the point is trivial.
    Suppose inductively that the statement is true of all proper submolecules \( V \) of \( U \).
    First suppose that \( U \) is an atom.
    If \( \dim U = 0 \), then the statement is trivial, otherwise \( \dim U > 0 \) and \( x \in \bd{}{\beta} U \) for some \( \beta \in \set{-, +} \).
    We conclude by inductive hypothesis and globularity.
    Now suppose split into \( U = V \cp{k} W \), and suppose that \( x \in V \), the case \( x \in W \) is symmetrical.
    Then either \( k > 0 \), hence \( \bd{0}{} V = \bd{0}{} U \) and we may conclude by inductive hypothesis on \( V \), or \( k = 0 \).
    If \( \a = - \), then \( \bd{0}{-} V = \bd{0}{-} U \), and we conclude by inductive hypothesis on \( V \) again.
    Else, \( \a = + \) and \( \bd{0}{+} U = \bd{0}{+} W \).
    Consider by inductive hypothesis an inclusion \( \iota \colon k\dglobe{1} \incl V \) from \( x \) to \( \bd{0}{+} V \).
    Then \( \bd{-}{0} W \) is isomorphic to \( k'\dglobe{1} \) for some \( k' > 0 \), hence \( \iota \cp{0} \bd{1}{-} \idd{W} \colon (k + k') \dglobe{1} \incl U \) is the desired path.
\end{proof}

\begin{lem} \label{lem:collapsible_connected_with_all_closure_element}
    Let \( U \) be a molecule, \( \beta \in \set{-, +} \), \( K \subseteq U \) be a \( \beta \)\nbd collapsible subset, and \( V \submol U \).
    Then \( V \cap K \) is either either empty or connected.
    In the latter case, \( V \cap K \subseteq V \) is \( \beta \)\nbd collapsible and contains \( \bd{0}{-\beta} V \).
\end{lem}
\begin{proof}
    Assume that \( \beta = - \), the case \( \beta = + \) is dual.
    Suppose that \( V \cap K \) is non-empty.
    We show that \( x \eqdef \bd{0}{-} V \) is in \( V \cap K \).
    Let \( z \in V \cap K \), then \( y \eqdef \bd{0}{-} z \in V \cap K \).
    By Lemma \ref{lem:path_from_zero_bd_to_all_points}, there exist \( k \geq 0 \) and \( \iota \colon k\arr \to V \) such that \( \bd{0}{} \iota = (x, y) \).
    Since \( y = \bd{0}{+} \iota \in K \) and \( K \) is collapsible, \( \cofaces{}{+} y \in K \).
    Since \( K \) is closed, \( \faces{}{-} \cofaces{}{+} y \in K \).
    Iterating this process, we find indeed that \( x \in K \).
    Then, suppose that \( V \cap K = F \cup G \) for two closed subsets \( F, G \) with \( F \cap G = \emptyset \), and suppose without loss of generality that \( x \in F \).
    Let \( y \in \gr{0}{G} \). 
    As previously, we have an inclusion \( \iota \colon k\arr \incl F \cup G \) such that \( \bd{0}{}\iota = (x, y) \).
    Then \( \iota(k\arr) = \invrs{\iota}(F) \cup \invrs{\iota}(G) \) and \( \invrs{\iota}F \cap \invrs{G} = \emptyset \).
    By \cite[Lemma 3.3.13]{hadzihasanovic2024combinatorics}, \( \invrs{\iota}F = \emptyset \) or \( \invrs{\iota}G = \emptyset \), a contradiction.
    This proves that \(  \gr{0}{G} = \emptyset \), thus that \( G = \emptyset \).
    This shows that \( V \cap K \) is connected.
    Finally, let \( x \in V \) such that \( \bd{0}{+} x \in K \).
    Since \( K \) is collapsible, \( x \in V \cap K \).
    Hence \( V \cap K \subseteq V \) is collapsible.
    This concludes the proof.
\end{proof}

\begin{lem} \label{lem:collapsible_negbeta_boundary_collapse_all}
    Let \( U \) be a molecule, \( \beta \in \set{-, +} \), and \( K \subseteq U \) be a \( \beta \)\nbd collapsible subset such that \( \bd{0}{-\beta} U \subseteq K \).
    Then \( K = U \).
\end{lem}
\begin{proof}
    We proceed induction on the layering dimension \( \ell \) of \( U \).
    If \( \ell = -1 \), then \( U \) is an atom, thus \( U = \clset{\top_U} \subseteq K \) by definition.
    Inductively, let \( \ell \geq 0 \). Then \( U \) admits a \( \ell \)\nbd layering
    \begin{equation*}
        U \eqdef \order{1}{U} \cp{\ell} \ldots \cp{\ell} \order{r}{U}
    \end{equation*}
    with \( r \geq 2 \) such that \( \order{i}{U} \) has layering dimension \( < \ell \) for all \( 1 \le i \le r \).
    Since \( \bd{0}{+} U = \bd{0}{+} \order{i}{U} \) and, by Lemma \ref{lem:collapsible_connected_with_all_closure_element}, \( \order{i}{U} \cap K \) is collapsible, we deduce by inductive hypothesis that \( \order{i}{U} = \order{i}{U} \cap K \).
    Therefore \( U = U \cap K \).
    This concludes the proof.
\end{proof}

\begin{lem} \label{lem:collapsible_mapcoll_preserve_boundaries}
    Let \( U \) be a molecule, \( \beta \in \set{-, +} \), \( K \subseteq U \) be a \( \beta \)\nbd collapsible subset, \( \a \in \set{-, +} \) and \( k \geq 0 \).
    Then 
    \begin{equation*}
        \mapcoll (\bd{k}{\a} U) = \bd{k}{\a} (\coll{K}{U}),
    \end{equation*}
    in particular \( \bd{0}{\beta} (\coll{K}{U}) = \set{\zcoll} \).
    Furthermore, \( \coll{K}{U} \) is globular, and if \( U \) is round, so is \( \coll{K}{U} \).
\end{lem}
\begin{proof}
    Without loss of generality, we assume that \( \beta = - \) and that \( K \) is a proper subset of \( U \), otherwise the statement is trivial.
    By Lemma \ref{lem:collapsible_connected_with_all_closure_element}, \( \bd{0}{-} U \subseteq K \).
    Suppose that \( k > 0 \), then
    \begin{equation*}
        \bd{k}{\a} (\coll{K}{U}) = \clos \faces{k}{\a} (\coll{K}{U}) \cup \gr{< k}{\clset{\maxel (\coll{K}{U})}} = \mapcoll(\bd{k}{\a} U) \cup \set{\zcoll}.
    \end{equation*}
    Since \( k > 0 \), Lemma \ref{lem:collapsible_connected_with_all_closure_element} implies that \( \set{\zcoll} = p(\bd{0}{-} U) \subseteq p(\bd{k}{\a} U) \).
    Thus \(  \mapcoll (\bd{k}{\a} U) = \bd{k}{\a} (\coll{K}{U}) \).
    Next suppose \( k = 0 \) and \( \a = + \).
    Then since \( K \) is a proper subset, \( \set{x} = \bd{0}{+} U \) is not a subset of \( K \) by Lemma \ref{lem:collapsible_negbeta_boundary_collapse_all}. 
    Now by definition of \( \coll{K}{U} \), one sees that \( \mapcoll (\bd{0}{+} U) = \bd{0}{+} (\coll{K}{U}) \).
    The next thing to see is therefore that \( \set{\zcoll} = \bd{0}{-} \coll{K}{U} \).
    By construction, \( \cofaces{}{+} \zcoll = \emptyset \), thus we have one inclusion.
    Let \( \icoll{u} \in \gr{0}{(U \setminus K)} \).
    If \( u \in \bd{0}{-} U \) then \( u \in K \) by Lemma \ref{lem:collapsible_connected_with_all_closure_element}, a contradiction.
    Thus there exists \( v \in \cofaces{}{+} u \), and since \( K \) is closed, \( v \notin K \).
    We find that \( \icoll{v} \in \cofaces{}{+} \icoll{u} \), meaning that \( u \notin \bd{0}{-} (\coll{K}{U}) \).
    This proves that \( \set{\zcoll} = \bd{0}{-} \coll{K}{U} \).
    By Lemma \ref{lem:collapsible_is_puhsout} and formal properties, for all closed subsets \( V \subseteq U \) such that \( V \) is a molecule, we have 
    \begin{equation*}
        p(V) \cong
        \begin{cases}
            V & \text{if } V \cap K = \emptyset, \\
            \coll{(V \cap K)}{V} & \text{else}.
        \end{cases}
    \end{equation*}
    With this together with the first part of the proof, one sees that \( \coll{K}{U} \) is globular.
    Last, suppose that \( U \) is round of dimension \( n \geq 0 \).
    Let \( k < n \), and let \( x \in (\bd{k}{-} (\coll{K}{U})) \cap (\bd{k}{+} (\coll{K}{U})) \).
    We must show that \( x \in \bd{k - 1}{}  (\coll{K}{U}) \)
    Suppose first that \( x = \zcoll \), then \( \set{x} = \bd{0}{-} (\coll{K}{U}) \).
    If \( k > 0 \) we are done by globularity, else \( k = 0 \).
    In that case, \( \zcoll \in \bd{0}{+} (\coll{K}{U}) \), so \( \bd{0}{+} U \subseteq K \), hence \( \coll{K}{U} = \pt \) by Lemma \ref{lem:collapsible_negbeta_boundary_collapse_all}.
    Thus \( n = 0 = k \), contradiction.
    Suppose now that \( x = \icoll{u} \) for some \( u \in U \setminus K \).
    By the first part of the proof, \( u \in \bd{k}{-} U \cap \bd{k}{+} U \), and since \( U \) is round, \( u \in \bd{k - 1}{} U \), so that by the first part of the proof again, \( \icoll{u} \in \bd{k - 1}{} (\coll{K}{U}) \).
    This concludes the proof.
\end{proof}

\begin{prop} \label{prop:collapsible_collapse_to_molecules}
    Let \( U \) be a molecule, \( \beta \in \set{-, +} \) and \( K \subseteq U \) be a \( \beta \)\nbd collapsible subset.
    Then \( \coll{K}{U} \) is a molecule and the canonical map \( \mapcoll \colon U \surj \coll{K}{U} \) is a final map of molecules.
\end{prop}
\begin{proof}
    Recall that a map of poset \( L \to \pt \) is final just when \( L \) is connected. 
    Let \( \mapcoll \colon U \surj \coll{K}{U} \) be the canonical map.
    Let \( V \submol U \) be a submolecule of \( U \). 
    If \( V \cap K = \emptyset \), then \( \restr{\mapcoll}{V} \colon V \surj \mapcoll(V) \) is the identity.
    Otherwise, by Lemma \ref{lem:collapsible_is_puhsout}, the diagram
    \begin{center}
        \begin{tikzcd}
            {V \cap K} & \pt \\
            V & {p(V)}
            \arrow[two heads, from=1-1, to=1-2]
            \arrow[from=1-1, to=2-1]
            \arrow[from=1-2, to=2-2]
            \arrow[two heads, from=2-1, to=2-2]
        \end{tikzcd}
    \end{center}
    is a pushout.
    By Lemma \ref{lem:collapsible_connected_with_all_closure_element}, \( V \cap K \) is connected, thus \( V \cap K \surj \pt \) is final.
    Since final maps of posets are stable under pushouts, we conclude that \( \restr{\mapcoll}{V} \colon V \surj \mapcoll(V) \) is final.
    This proves that for all submolecules \( V \submol U \), \( \restr{\mapcoll}{V} \colon V \surj \mapcoll(V) \) is final.
    Now let \( x \in U \), \( k \geq 0 \) and \( \a \in \set{-, +} \).
    Using either the fact that \( \restr{p}{\bd{k}{\a} x} \) is the identity if \( \clset{x} \cap K = \emptyset \) and Lemma \ref{lem:collapsible_mapcoll_preserve_boundaries} otherwise, we find that \( \mapcoll(\bd{k}{\a} x) = \bd{k}{\a} \mapcoll(x) \).
    This proves that, in case \( \coll{K}{U} \) is a regular directed complex, \( \mapcoll \) is a final map of regular directed complexes.

    We prove by induction on the submolecules \( V \) of \( U \) that \( \mapcoll(V) \) is a molecule.
    The base case where \( V \) is a point is clear.
    Suppose inductively that \( \mapcoll(V) \) is a molecule for all proper submolecules of \( U \).
    Suppose first that \( U \) is an atom.
    By inductive hypothesis, for \( \a \in \set{-, +} \), \( p(\bd{}{\a} U) \) is a molecule, and is round by Lemma \ref{lem:collapsible_mapcoll_preserve_boundaries}.
    Thus \( p(\bd{}{-} U) \celto p(\bd{}{+} U) \) is a well defined atom, and one sees that, unless \( K = U \), it is by construction isomorphic to \( \coll{K}{U} \).
    If \( K = U \), then \( U = \pt \) is also an atom. 
    Now suppose that \( U \) is a molecule.
    Then by inductive hypothesis, \( \coll{K}{U} \) is a regular directed complex.
    We conclude by \cite[Proposition 6.2.33]{hadzihasanovic2024combinatorics} and the first part of the proof.
\end{proof}

\begin{lem} \label{lem:has_two_point_is_susp}
    Let \( U \) be a molecule such that \( \gr{0}{U} = \bd{0}{} U \).
    Then \( U = \sus{U'} \) for some molecule \( U' \).
\end{lem}
\begin{proof}
    We proceed by induction on the submolecules \( V \) of \( U \).
    The base case where \( V \) is a point is vacuously true, since \( \gr{0}{V} \) have exactly one element.
    Suppose inductively that the statement is true of all submolecules of \( U \)
    Suppose first that \( U \) is an atom.
    We distinguish two cases.
    Either \( U = \dglobe{1} \), in which case it is \( \sus{\pt} \), or \( U = V \cell W \) with \( \dim V = \dim W \geq 1 \).
    Then, 
    \begin{equation*} \label{eq:grade_boundary_zero}
         \bd{0}{} V \subseteq \gr{0}{V} \subseteq \gr{0}{U} = \bd{0}{} U = \bd{0}{} V,
    \end{equation*}
    thus \( \gr{0}{V} = \bd{0}{} V \).
    Similarly, \( \gr{0}{W} = \bd{0}{} W \).
    By inductive hypothesis on the submolecules, \( V = \sus{V'} \) and \( W = \sus{W'} \) for some molecules \( V' \) and \( W' \).
    By \cite[Proposition 7.3.16]{hadzihasanovic2024combinatorics}, \( U = \sus{(V' \cell W')} \).
    Suppose next that \( U \) splits into \( V \cp{k} W \) with \( \dim V, \dim W > k \).
    If \( k = 0 \), then \( \bd{0}{-} V \cup \bd{0}{+} V \cup \bd{0}{+} W \subseteq \gr{0}{U} \).
    By hypothesis on the dimension of \( V \) and \( W \), \( \bd{0}{-} V \cup \bd{0}{+} V \cup \bd{0}{+} W \) has exactly three elements, contradicting the assumption.
    Thus \( k > 0 \) and (\ref{eq:grade_boundary_zero}) also holds.
    We deduce by inductive hypothesis that \( V = \sus{V'} \) and \( W = \sus{W'} \).
    By \cite[Proposition 7.3.16]{hadzihasanovic2020diagrammatic} again, \( U = \sus{(V' \cp{k - 1} W')} \).
    This concludes the proof.
\end{proof}

\begin{prop} \label{prop:desuspension}
    Let \( U \) be a molecule, and \( K^- \coprod K^+ \) be a partition of \( \gr{0}{U} \) such that for all \( \a \in \set{-, +} \), \( K^\beta \subseteq U \) is \( \beta \)\nbd collapsible.
    Then there exists a molecule \( V \), together with a commutative diagram
    \begin{center}
        \begin{tikzcd}
            {K^- \coprod K^+} & {\pt \coprod \pt} \\
            U & {\sus{V}}
            \arrow[two heads, from=1-1, to=1-2]
            \arrow[from=1-1, to=2-1]
            \arrow["{(\bot^-, \bot^+)}", from=1-2, to=2-2]
            \arrow["q"', two heads, from=2-1, to=2-2]
        \end{tikzcd}
    \end{center}
    whose underlying diagram in \( \Pos \) is a pushout, and such that \( q \) is a final map of molecules.
\end{prop}
\begin{proof}
    By Proposition \ref{prop:collapsible_collapse_to_molecules}, we have a final map of molecules \( p \colon U \surj \coll{K^-}{U} \) is a molecule, and it is straightforwards to see that \( p(K^+) \subseteq \coll{K^-}{U} \) is collapsible.
    Applying Proposition \ref{prop:collapsible_collapse_to_molecules} again, we have a final map of molecules \( q \colon U \surj W \), which is the pushout of \( K^- \coprod K^- \to \pt \coprod \pt \) along the inclusion \( K^+ \coprod K^- \incl U \).
    By Lemma \ref{lem:collapsible_mapcoll_preserve_boundaries}, the inclusion \( \pt \coprod \pt \incl W \) has image \( \bd{0}{} W \).
    To conclude, we need to show that \( W = \sus{V} \) for some molecule \( V \).
    Since \( q \) is a surjective map of molecules, \( \gr{0}{W} = q(\gr{0}{U}) = q(K^-) \coprod q(K^+) = \pt \coprod \pt \).
    We conclude by Lemma \ref{lem:has_two_point_is_susp}.
\end{proof}

\begin{thm} \label{thm:suspension_of_stricter}
    Let \( C \) be a stricter \( \omega \)\nbd category.
    Then \( \sus{C} \) is a stricter \( \omega \)\nbd category.
\end{thm}
\begin{proof}
    Let \( U \) be a molecule and \( \set{\F_x \colon \molecin{\imel{U}{x}} \to \sus{C}}_{x \in U} \) be a \( U \)\nbd matching family in \( \sus{C} \).
    For \( \beta \in \set{-, +} \), let \( K^\beta \eqdef \set{x \in U \mid \pcell{\F_x} = \bot^\a} \).
    We claim that \( K^\beta \subseteq U \) is either empty or \( \beta \)\nbd contractible.
    Let \( x \in U \), and suppose that \( \bd{0}{-\beta} x \in K^\beta \).
    Then \( \bd{0}{-\beta} \pcell{F_x} = \bot^{\beta} \).
    If \( \pcell{\F_x} \neq \bot^\beta \), then by construction of \( \sus{C} \), \( \bd{0}{-\beta} \pcell{\F_x} = \bot^{-\beta} \).
    Thus \( \pcell{\F_x} = \bot^\beta \), hence \( x \in K^\beta \).
    This shows that \( K^\beta \) is either empty or \( \beta \)\nbd contractible.
    Now \( U = K^- \coprod K^+ \).
    Suppose that there exists \( \beta \in \set{-, +} \) such that \( K^\beta \) is empty and let \( x \in U \).
    If \( \pcell{\F_x} \in C \), we would have \( x \in K^\beta \).
    Thus \( \pcell{\F_x} = \bot^{-\beta} \), and \( K^{-\beta} = U \).
    Therefore, the candidate amalgamation \( \F \colon \molecin{U} \to \sus{C} \) is well defined since it factors as
    \begin{equation*}
        \molecin{U} \to \molecin{\pt} \to \sus{C},
    \end{equation*}
    where \( \molecin{\pt} \to \sus{C} \) classifies \( \bot^{-\beta} \).
    Now suppose that \( K^- \coprod K^+ \) forms a partition of \( U \).
    Then Proposition \ref{prop:desuspension} applies, and we have in particular a molecule \( U' \) and a final map of molecules \( q \colon U \surj \sus{U'} \). 
    By \cite[Theorem 6.2.35]{hadzihasanovic2024combinatorics}, \( \molecin{q} \) is a strict functor.
    Then, one checks that the family of globular cells of \( C \)
    \begin{equation*}
        \set{\pcell{\F_x} \mid q(x) = \sus{y}}_{y \in U'}
    \end{equation*} 
    defines, as per Remark \ref{rmk:data_matching family}, a matching family \( \set{\G_y \colon \molecin{\imel{U'}{y}} \to C}_{y \in U'} \).
    Because \( C \) is stricter, it admits a well defined amalgamation \( \G \colon \molecin{U'} \to C \).
    Finally, the candidate amalgamation \( \F \colon \molecin{U} \to \sus{C} \) is well defined, since it factors as \( \F = \sus{\G} \after \molecin{q} \).
    This concludes the proof.
\end{proof}

\begin{dfn} [Bipointed stricter \( \omega \)\nbd category]
    A \emph{bipointed stricter \( \omega \)\nbd category} is given by a stricter \( \omega \)\nbd category \( C \) together with a pair of objects \( (a, b) \).
    We let \( \bpt\somegaCat \) be the category of bipointed stricter \( \omega \)\nbd category and strict functors that respect the bipointing.
\end{dfn}

\begin{cor} \label{cor:adjunction_hom_suspension}
    There is an adjunction
    \begin{equation*}
        \sus{} \colon \somegaCat \leftrightarrows \bpt\somegaCat \cocolon \hom,
    \end{equation*}
    where \( \sus{C} \) is the suspension of \( C \) bipointed by \( (\bot^-, \bot^+) \), and \( \hom(C, a, b) \) is given by \( C(a, b) \).
    Furthermore, the functor
    \begin{equation*}
        \sus{} \colon \somegaCat \to \somegaCat
    \end{equation*}
    preserves connected colimits.
\end{cor}
\begin{proof}
    That the pair of functors is well defined follows from Lemma \ref{lem:hom_of_stricter_is_stricter} and Theorem \ref{thm:suspension_of_stricter}.
    That this form an adjunction follows from standard arguments.
    Last, since \( \bpt\somegaCat \) is a coslice construction, \( \sus{} \colon \somegaCat \to \somegaCat \) preserves connected colimits by \cite[Proposition 3.3.8]{riehl2019context}.
\end{proof}

\subsection{Folk model structure on stricter \texorpdfstring{$\omega$}{}-categories}

\begin{dfn} [Acyclic fibration] \label{dfn:acyclic_fibration}
    Let \( f \colon C \to D \) be a strict functor of compositions structures.
    We say that \( f \) is an \emph{acyclic fibration} if
    \begin{enumerate}
        \item \( f \) is surjective on objects;
        \item for each \( n \geq 0 \), each pair \( c, c' \) of parallel globular \( n \)\nbd cells of \( C \), and each globular \( (n + 1) \)\nbd cell \( v \colon f(c) \gcelto f(c') \), there exists a globular \( (n + 1) \)\nbd cell \( u \colon c \gcelto c' \) such that \( f(u) = v \)
    \end{enumerate}
\end{dfn}

\begin{dfn} [Reversible globular cell]
    Let \( C \) be a composition structure, and \( e \colon x \gcelto y \) be a globular cell of dimension \( n > 0 \) in \( C \).
    We say that \( e \) is reversible if there exists a globular cell \( e^* \colon y \gcelto x \), as well as globular cells \( h \colon e \comp{n - 1} e^* \gcelto x \) and \( h' \colon e^* \comp{n - 1} e \gcelto y \) such that \( h \) and \( h' \) are reversible.
    In that case, \( e^* \) is called a \emph{weak inverse} of \( e \).
    We write \( x \sim y \) if there exists a reversible globular cell \( e \) of type \( x \gcelto y \).
\end{dfn}

\begin{dfn} [\( \omega \)\nbd equivalence]
    Let \( f \colon C \to D \) be a strict functor of composition structures.
    We say that \( f \) is an \( \omega \)\nbd equivalence if
    \begin{enumerate}
        \item for each globular cell \( d \) in \( D \) of dimension \( 0 \), there exists a globular cell \( c \) in \( C \) such that \( f(c) \sim d \);
        \item for each \( n \geq 0 \), each pair \( c, c' \) of parallel globular \( n \)\nbd cells of \( C \), and each globular \( (n + 1) \)\nbd cell \( v \colon f(c) \gcelto f(c') \), there exists a globular \( (n + 1) \)\nbd cell \( u \colon c \gcelto c' \) such that \( f(u) \sim v \)
    \end{enumerate}
\end{dfn}

\begin{comm}
    The definitions of acyclic fibration, reversibility, and \( \omega \)\nbd equivalence make sense in any composition structure, in particular in a strict or stricter \( \omega \)\nbd category.
    In the former case, we recover the usual definitions, seepoly for instance \cite[19.2.3, 20.1.1, 20.1.11]{ara2025polygraphs}.
\end{comm}

\begin{thm} \label{thm:folk_model_structure}
    There exists a model structure, called the \emph{folk model structure}, on \( \omegaCat \) such that:
    \begin{enumerate}
        \item the weak equivalences are the \( \omega \)\nbd equivalences;
        \item the acyclic fibrations are the one of Definition \ref{dfn:acyclic_fibration};
        \item every strict \( \omega \)\nbd category is fibrant. 
    \end{enumerate} 
\end{thm}
\begin{proof} 
    See \cite{lafont2010folk}.
\end{proof}

\noindent We conclude this section by giving an analogue to Theorem \ref{thm:folk_model_structure}, with \( \somegaCat \) in place of \( \omegaCat \).
We let 
\begin{equation*}
    \Icof \eqdef \set{i_n \colon \bd{}{} \globe{n} \incl \globe{n} \mid n \geq 0}.
\end{equation*}
By the small object argument in \( \somegaCat \), each strict functor \( f \colon C \to D \) of stricter \( \omega \)\nbd categories factors as \( f = p i \) where \( i \) is a relative stricter polygraph and \( p \) has the right lifting property against \( \Icof \), that is, \( p \) is an acyclic cofibration.

\begin{dfn} \label{dfn:generating_folk_acyclic_cof}
    For each \( n \geq 0 \), consider the strict functor \( \bd{}{} \globe{n + 1} \to \globe{n} \) sending the two non-trivial globular \( n \)\nbd cells of \( \bd{}{} \globe{n + 1} \) to the only non-trivial globular cell of \( \globe{n} \).
    We fix a factorisation of this strict functor
    \begin{equation*}
        \bd{}{} \globe{n + 1} \stackrel{j}{\to} \swE{n + 1} \stackrel{p}{\to} \globe{n}
    \end{equation*}
    into a stricter cofibration followed by an acyclic fibration, and let \( j_n \) be the composite
    \begin{equation*}
        \globe{n} \incl \bd{}{} \globe{n + 1} \stackrel{j}{\to} \swE{n + 1},
    \end{equation*}
    where \( \globe{n} \incl \bd{}{} \globe{n + 1} \) is the ``source'' inclusion, that is, the one factoring through \( \molecin{(\bd{}{-}\dglobe{n + 1})} \incl \molecin{\dglobe{n + 1}} \).
    We let
    \begin{equation*}
        \sJcof \eqdef \set{j_n \colon \globe{n} \to \swE{n + 1} \mid n \geq 0}.
    \end{equation*} 
\end{dfn}

\begin{dfn} [Stricter \( \omega \)\nbd category of cylinders]
    Let \( C \) be a stricter \( \omega \)\nbd category.
    The \emph{stricter \( \omega \)\nbd category of cylinder} is the stricter \( \omega \)\nbd category 
    \begin{equation*}
       \Gamma(C) \eqdef \homlax(\globe{1}, C). 
    \end{equation*}
\end{dfn}

\begin{rmk} \label{rmk:strict_stricter_same_cylinders}
    By Proposition \ref{prop:reflection_to_stricter_monoidal}, the stricter categories \( \Gamma(C) \) coincides with the strict \( \omega \)\nbd categories of cylinders \cite[Remark 20.2.9]{ara2025polygraphs}, which happen to be stricter, since \( C \) is.
\end{rmk}

\begin{thm} \label{thm:folk_model_structure_on_stricter}
    There is a cofibrantely generated model structure on the category \( \somegaCat \), called the \emph{folk model structure}, where:
    \begin{enumerate}
        \item \( \Icof \) is a set of generating cofibrations;
        \item \( \sJcof \) is a set of generating acyclic cofibrations;
        \item weak equivalences are the \( \omega \)\nbd equivalences;
        \item acyclic fibrations are the one of Definition \ref{dfn:acyclic_fibration}.
        \item all stricter \( \omega \)\nbd categories are fibrant;
        \item cofibrant objects are the stricter polygraphs.
    \end{enumerate}
    Furthermore, this model structure is right transferred from the folk model structure on \( \omegaCat \) along the adjunction 
    \begin{center}
        \begin{tikzcd}
            \somegaCat & \omegaCat,
            \arrow[""{name=0, anchor=center, inner sep=0}, "\iota"', curve={height=12pt}, hook, from=1-1, to=1-2]
            \arrow[""{name=1, anchor=center, inner sep=0}, "\rcs"', curve={height=12pt}, from=1-2, to=1-1]
            \arrow["\dashv"{anchor=center, rotate=-90}, draw=none, from=1, to=0]
        \end{tikzcd}
    \end{center}
    making it a Quillen adjunction.
\end{thm}
\begin{proof}
    This is a direct application of \cite[Proposition 21.3.2]{ara2025polygraphs} with Remark \ref{rmk:strict_stricter_same_cylinders}, noticing that \( \rcs \Icof = \Icof \), and that, by \cite[20.4.7]{ara2025polygraphs}, a generating set \( \Jcof \) of acyclic cofibrations for the folk model structure on \( \omegaCat \) is given by the same construction as (\ref{dfn:generating_folk_acyclic_cof}), but computing the small objects argument in \( \omegaCat \) instead of \( \somegaCat \).
    As a consequence, the classes \( \llp(\rlp(\rcs\Jcof)) \) and \( \llp(\rlp(\sJcof)) \) coincide.
\end{proof}

\begin{rmk} 
    By Lemma \ref{lem:pushout_principal_cell}, the set
    \begin{equation*}
        \set{\molecin{\bd{}{}U} \incl \molecin{U} \mid U \text{ atom}}
    \end{equation*}
    is also a generating set of acyclic cofibrations for the folk model structure on \( \somegaCat \), whose cofibrations are therefore given by the retracts of relative polygraphs.
\end{rmk}

\noindent Finally, we fix \( n \in \mathbb{N} \), and let
\begin{equation*}
    \Icof_n \eqdef \trunc{n}\Icof,\quad\quad \sJcof_n \eqdef \trunc{n}\Jcof.
\end{equation*}

\begin{thm} \label{thm:folk_model_structure_on_stricter_n}
    There is a cofibrantely generated model structure on the category \( \snCat{n} \), called the \emph{folk model structure}, where:
    \begin{enumerate}
        \item \( \Icof_n \) is a set of generating cofibrations;
        \item \( \sJcof_n \) is a set of generating acyclic cofibrations;
        \item weak equivalences are the \( \omega \)\nbd equivalences;
    \end{enumerate}
    Furthermore, this model structure is right transferred from the folk model structure on \( \nCat{n} \) along the adjunction 
    \begin{center}
        \begin{tikzcd}
            \snCat{n} & \nCat{n},
            \arrow[""{name=0, anchor=center, inner sep=0}, "\iota"', curve={height=12pt}, hook, from=1-1, to=1-2]
            \arrow[""{name=1, anchor=center, inner sep=0}, "\rcs"', curve={height=12pt}, from=1-2, to=1-1]
            \arrow["\dashv"{anchor=center, rotate=-90}, draw=none, from=1, to=0]
        \end{tikzcd}
    \end{center}
    making it a Quillen adjunction.
\end{thm}
\begin{proof}
    Same as Theorem \ref{thm:folk_model_structure_on_stricter}.
\end{proof}

\noindent Therefore, we have the following commutative square of left Quillen functors
\begin{center}
    \begin{tikzcd}
        \omegaCat & \somegaCat \\
        {\nCat{n}} & {\snCat{n}.}
        \arrow["\rcs", from=1-1, to=1-2]
        \arrow["{\trunc{n}}"', from=1-1, to=2-1]
        \arrow["{\trunc{n}}", from=1-2, to=2-2]
        \arrow["\rcs"', from=2-1, to=2-2]
    \end{tikzcd}
\end{center}