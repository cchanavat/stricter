\section{Stricter \texorpdfstring{$\omega$}{ω}-categories}

\subsection{Definitions and properties}

\begin{dfn} [Reflexive \( \omega \)\nbd graph]
    A \emph{reflexive \( \omega \)\nbd graph} is a set, whose element are called the \emph{globular cells}, \( C \) together with, for each \( k \geq 0 \), operators
    \begin{equation*}
        \bd{k}{-}, \bd{k}{+} \colon C \to C,
    \end{equation*}
    called the \emph{input and output \( k \)\nbd boundary}, respectively, satisfying the following axioms.
    \begin{enumerate}
        \item for all \( c \in C \), there exists \( k \geq 0 \) such that \( \bd{k}{-} c = c = \bd{k}{+} c \); the \emph{dimension} of \( c \), written \( \dim c \), is the minimum of all such values of \( k \);
        \item for all \( c \in C \), all \( k, n \geq 0 \) and all \( \a, \beta \in \set{-, +} \),
        \begin{equation*}
            \bd{k}{\a}(\bd{n}{\beta} c) = 
            \begin{cases}
                \bd{k}{\a} c & \text{if }k < n, \\
                \bd{n}{\beta} c& \text{else.}
            \end{cases}
        \end{equation*}
    \end{enumerate}
    A \emph{morphism of reflexive \( \omega \)\nbd graph} is a function of the underlying set commuting with the boundary operators.
\end{dfn}

\noindent If \( C \) is a reflexive \( \omega \)\nbd graph, the set of \emph{\( k \)\nbd composable pairs of globular cells} is the set 
\begin{equation*}
    C \times_k C \eqdef \set{(c, d) \in C \times C \mid \bd{k}{+} c = \bd{k}{-} d}.
\end{equation*}\ccnote{maybe change to source target instead of bd }
Given a globular cell \( c \) and \( \a \in \set{-, +} \), we write \( \bd{}{\a} c \) in place of \( \bd{\dim c - 1}{\a} c \), and \( c \colon u^- \gcelto u^+ \) to signify that \( \bd{}{\a} c = u^\a \), and call \( u^- \gcelto u^+ \) the \emph{type of \( c \)}. 
Two globular cells of the same type are said to be \emph{parallel}.
We say that a globular cell \( c \) is an \emph{object} if \( \dim c = 0 \). 

\begin{dfn} [Composition structure]
    A composition structure is a reflexive \( \omega \)\nbd graph \( C \) together with, for all \( k \geq 0 \), an operation
    \begin{equation*}
        - \comp{k} - \colon C \times_k C \to C,
    \end{equation*}
    call the \emph{\( k \)\nbd composition}.
    If \( C, D \) are composition structures, a \emph{strict functor} \( f \colon C \to D \) is a morphism of the underlying reflexive graph respecting the \( k \)\nbd composition operations for all \( k \geq 0 \).
    We denote \( \Comp \) the category of composition structures and strict functors.
\end{dfn}

\begin{rmk}
    The category \( \Comp \) is equivalent to a category of models of a limit sketch; use for instance a simpler version of \cite[Proposition 14.2.4]{ara2025polygraphs}.
    In particular, it is locally presentable, complete, and cocomplete \cite{adamek1994locally}.
\end{rmk}

\begin{dfn} [Basis for composition structure]
    Let \( C \) be a composition structure, and \( \cls{S} \) be a subset of the globular cells of \( C \).
    We say that \( \cls{S} \) is a \emph{generating set for \( C \)} if the closure of \( \cls{S} \) under \( - \comp{k} - \) is equal to \( C \).
    We say that a generating set is a \emph{basis for \( C \)} if for any other generating set \( \cls{T} \subseteq \cls{S} \), then \( \cls{T} = \cls{S} \).
\end{dfn}

\begin{lem}\label{lem:strict_functor_determined_by_basis}
    Let \( f, g \colon C \to D \) be strict functors of composition structures and let \( \cls{S} \) be a generating set for \( C \) such that for all \( c \in \cls{C} \), \( f(c) = g(c) \).
    Then \( f = g \).
\end{lem}
\begin{proof}
    See \cite[Lemma 5.1.23]{hadzihasanovic2024combinatorics}.
\end{proof}

\begin{dfn} [Molecules in a regular directed complexes]
    Let \( P \) be a regular directed complex.
    We let \( \molecin{P} \) to be the composition structure whose
    \begin{itemize}
        \item globular cells are the morphisms \( u \colon U \to P \) with \( U \) a molecule;
        \item boundary operators are defined by \( u \mapsto \bd{k}{ \a} u \), for all \( k \geq 0 \) and \( \a \in \set{-, +} \);
        \item the \( k \)\nbd composition operation is given by the pasting \( (u, v) \mapsto u \cp{k} v \), for all \( k \geq 0 \). 
    \end{itemize}    
\end{dfn}

\begin{rmk}
    Since isomorphisms of molecules are unique when they exist, we may safely identify identity two isomorphic objects \( u \colon U \to P \) and \( u' \colon U' \to P \) in the slice over \( P \).
\end{rmk}

\noindent Recall from \cite[Section 5.2, Section 6]{hadzihasanovic2024combinatorics} that the subset \( \atomin{P} \) of \( \molecin{P} \) on morphisms \( u \colon U \to P \) whose domain is an atom, is a basis for \( \molecin{P} \).
Furthermore, the assignment \( P \mapsto \molecin{P} \) extends to functors
\begin{align*}
    \molecin{-} &\colon \rdcpxmap \to \Comp \\
    \molecin{-} &\colon \opp{\rdcpxcomap} \to \Comp. \\    
\end{align*} 

\begin{dfn} [Globe]
    Let \( n \geq 0 \).
    The \emph{\( n \)\nbd globe} is the composition structure defined by \( \globe{n} \eqdef \molecin{\dglobe{n}} \), and its \emph{boundary} is the composition structure given by \( \bd{}{}\globe{n} \eqdef \molecin{(\bd{}{}\dglobe{n})} \).
\end{dfn}

\begin{dfn} [Pasting diagram in a composition structure]
    Let \( C \) be a composition structure and \( U \) be a molecule.
    A \emph{pasting diagram of shape \( U \) in \( C \)} is a strict functor \( u \colon \molecin{P} \to C \).
    If \( U \) is an atom, we say that \( u \) is a \emph{cell}.
    We let \( \dim u \eqdef \dim U \), and for \( k \geq 0 \) and \( \a \in \set{-, +} \), we write \( \bd{k}{\a} u \) for the restriction of \( u \) along the strict functor \( \molecin{\bd{k}{\a} U} \to \molecin{U} \), and may omit \( k \) when it is equal to \( \dim u - 1 \).
    Finally, we write \( u \colon u^- \celto u^+ \) to mean that \( \bd{}{\a} u = u^\a \), for all \( \a \in \set{-, +} \), and call \( u^- \celto u^+ \) the \emph{type} of \( u \).
    Two diagrams are \emph{parallel} if they have the same type.
\end{dfn}

\begin{dfn} [Principal cell]
    Let \( \F \colon \molecin{U} \to C \) be a pasting diagram.
    The \emph{principal cell of \( \F \)} is the globular cell \( \pcell{\F} \eqdef \F(\idd{U}) \).
\end{dfn}

\begin{rmk}
    We warn the reader on the difference between cells and globular cells. 
    A globular cell \( c \in C \) of dimension \( n \geq 0 \) is classified uniquely by the cell \( u \colon \globe{n} \to C \) such that \( \pcell{u} = c \).
    Conversely, a cell \( u \colon \globe{n} \to C \) factors uniquely as \( \molecin{\tau} \after v \), where \( \tau \colon \dglobe{n} \to \dglobe{\dim \pcell{u}} \) is the unique surjection of atoms, and \( v \colon \globe{\dim \pcell{u}} \to \C \) classifies \( \pcell{u} \).
\end{rmk}

\begin{dfn}
    For each regular directed complex \( P \), there is a canonical strict functor
    \begin{equation*}
        s_P \colon \colim_{x \in P} \molecin{\imel{P}{x}} \to \molecin{P},
    \end{equation*}
    where the colimit is computed in \( \Comp \).
    We let \( S \) be the set
    \begin{equation*}
        S \eqdef \set{s_U \colon \colim_{x \in U} \molecin{\imel{U}{x}} \to \molecin{U} \mid U \text{ molecule}}.
    \end{equation*}
\end{dfn}

\begin{dfn}[Stricter \( \omega \)\nbd category]
    A \emph{stricter \( \omega \)\nbd category} is a composition structure \( C \) which is local with respect to \( S \).
    We let \( \somegaCat \) be the full subcategory of \( \omegaCat \) on stricter \( \omega \)\nbd categories.
\end{dfn}

\noindent Since \( \Comp \) is locally presentable and \( S \) is a small set, the full subcategory inclusion \( \iota \colon \somegaCat \incl \Comp \) is reflective \cite{freyd1972continuous}, and we denote by 
\begin{equation*}
    \rc \colon \Comp \to \somegaCat
\end{equation*}
the left adjoint of \( \iota \).

\begin{dfn} [Matching family and amalgamation]
    Let \( P \) be a regular directed complex, and \( C \) be a composition structures.
    A \emph{\( P \)\nbd matching family in \( C \)} is a cone 
    \begin{equation*}
        \set{\F_x \colon \molecin{\imel{P}{x}} \to C}_{x \in P}
    \end{equation*}
    under the \( P \)\nbd shaped diagram \( x \mapsto \molecin{\imel{P}{x}} \).    
    An \emph{amalgamation} of this matching family is a strict functor 
    \begin{equation*}
        \amalg_{x\in P} \F_x \colon \molecin{P} \to C
    \end{equation*}
    such that for all \( y \in P \), \( (\amalg_x \F_x) \after \molecin{\mapel{y}} = \F_y \).
\end{dfn}

\begin{rmk}
    Thus, a composition structure \( C \) is a stricter \( \omega \)\nbd category if for all molecules \( U \), all \( U \)\nbd matching families in \( C \) have a unique amalgamation.
\end{rmk}

\begin{rmk}\label{rmk:data_matching family}
    The data of a \( P \)\nbd matching family 
    \begin{equation*}
        \set{\F_x \colon \molecin{\imel{P}{x}} \to C}_{x \in P}
    \end{equation*}
    in \( C \) is given by a family of globular cells \( \set{c_x \in C}_{x \in P} \).
    Indeed, given a matching family \( \set{\F_x}_{x \in P} \), define \( c_x \eqdef \pcell{\F_x} \).
    By functoriality, if \( x \le y \), then \( c_x = \F_y(\imel{P}{x} \incl \imel{P}{y}) \).
    Since \( \atomin{\imel{P}{y}} \) is a basis for \( \molecin{\imel{P}{y}} \), Lemma \ref{lem:strict_functor_determined_by_basis} implies that the data of \( \set{c_x}_{x \in \imel{P}{y}} \) entirely determines \( \F_y \colon \molecin{\imel{P}{y}} \to C \).
    Of course, not all data of this type give rise to a matching family. 
\end{rmk}

\begin{lem}\label{lem:at_most_one_lift}
    Let \( C \) be a composition structure, \( P \) be a regular directed complex, and \( \set{\F_x}_{x \in P} \) be a matching family. 
    Then \( \set{\F_x}_{x \in P} \) has at most one amalgamation.
\end{lem}
\begin{proof}
    Immediate by Lemma \ref{lem:strict_functor_determined_by_basis}.
\end{proof}

\begin{lem} \label{lem:well_define_from_regular_iff_well_defined_from_molecules}
    Let \( C \) be a composition structure, \( P \) be a regular directed complex, \( \set{\F_x \colon \molecin{\imel{P}{x}} \to C}_{x \in P} \) be a \( P \)\nbd matching family in \( C \). 
    Suppose that for all \( w \colon W \to P \) in \( \molecin{P} \), the \( W \)\nbd matching family \( \set{\F_{w(x)} \colon \imel{P}{w(x)} \to X}_{x \in W} \) in \( C \) has a well defined amalgamation.
    Then \( \set{\F_x}_{x \in P} \) has a well defined amalgamation.
\end{lem}
\begin{proof}
    For \( w \colon W \to P \) in \( \molecin{P} \), we let \( \F_w \) be the strict functor 
    \begin{equation*}
        \amalg_{x \in W} \F_{w(x)} \colon \molecin{W} \to C
    \end{equation*}
    We claim that \( \F \colon \molecin{P} \to C \) defined by \( w \mapsto \pcell{\F_w} \) is a strict functor.
    Let \( w \colon W \to P \) in \( \molecin{P} \), and suppose that \( w = w^- \cp{k} w^+ \) for some \( k \geq 0 \) and \( w^\a \colon W^\a \to P \) in \( \molecin{P} \), for all \( \a \in \set{-, +} \).
    Then
    \begin{equation*}
        \pcell{\F_w} = \F_w(W^- \incl W) \comp{k} \F_w(W^+ \incl W) = \pcell{\F_{w^-}} \comp{k} \pcell{\F_{w^+}}.
    \end{equation*}
    Thus \( \F(w) = \F(w^-) \comp{k} \F(w^+) \).
    This concludes the proof.
\end{proof}

\begin{comm} \label{comm:well_defined_amalgamation}
    Given a \( P \)\nbd matching family \( \set{\F_x} \) in \( C \), we thus have a \emph{candidate amalgamation} \( \F \colon \molecin{P} \to C \) defined on the basis \( \atomin{P} \) by \( \mapel{x} \mapsto \pcell{\F_x} \).
    Then, \( \F \) is a well defined strict functor if for all morphisms \( u \colon U \to P \) with \( U \) a molecule, if \( u = u_1 \cp{k} u_2 \), then \( \F(u) = \F(u_1) \comp{k} \F(u_2) \).
    To show this, we may use \emph{induction on submolecule} (see \cite[Comment 4.1.7]{hadzihasanovic2024combinatorics}) as follows.
    Take an arbitrary element \( w \colon W \to P \) in \( \molecin{P} \), and prove that \( \F \after \molecin{w} \) is a well defined strict functor under the hypothesis that for all proper subdiagrams \( w' \) of \( w \), \( \F \after \molecin{w'} \) is a well defined strict functor.
    The base case on subdiagrams \( w' \) of \( w \) of dimension \( 0 \) is always true in that case. 
    % % Then, using induction on submolecules (a variant of \cite[Comment 4.1.7]{hadzihasanovic2024combinatorics}), to show that \( \F \) is a well defined strict functor, we may take an arbitrary element \( w \colon W \to P \) in \( \molecin{P} \), and prove that \( \F \after \molecin{w} \) is well defined under the hypothesis that for all proper subdiagrams \( w' \) of \( w \), \( \F \after \molecin{w'} \) is well defined. 
\end{comm}

\begin{lem} \label{lem:stricter_iff_local_wrt_pasting}
    Let \( C \) be composition structure.
    The following are equivalent.
    \begin{enumerate}
        \item \( C \) is a stricter \( \omega \)\nbd category;
        \item for all regular directed complexes \( P \), \( C \) is local with respect to \( s_P \);
        \item for all pairs of molecules \( U, V \), and \( k \geq 0 \) such that \( U \cp{k} V \) is defined, each lifting problem
        \begin{center}
            \begin{tikzcd}
                {\molecin{U} \cup \molecin{V}} & C \\
                {\molecin{(U \cp{k} V)}}
                \arrow[from=1-1, to=1-2]
                \arrow[from=1-1, to=2-1]
            \end{tikzcd}
        \end{center}
        has a (necessarily unique) solution.
    \end{enumerate}
\end{lem}
\begin{proof}
    That the first point implies the second follows from Lemma \ref{lem:well_define_from_regular_iff_well_defined_from_molecules} and Lemma \ref{lem:at_most_one_lift}.
    Conversely, if \( C \) is local with respect to all the strict functors \( s_P \) where \( P \) is a regular directed complex, then it is also the case for all the strict functors \( s_U \) where \( U \) is a molecule.
    This shows the first two conditions are equivalent.
    Finally, the last condition is clearly necessary, since any functor \( \molecin{U} \cup \molecin{V} \to C \) defines in particular a \( (U \cp{k} V) \)\nbd matching family in \( C \).
    Conversely, we show it is sufficient.
    Let \( P \) be a regular directed complex.
    We show that \( C \) is local with respect to \( s_p \).
    Let \( \set{\F_x} \) be a \( P \)\nbd matching family in \( C \).
    We show that the candidate amalgamation \( \F \colon \molecin{P} \to C \) is a strict functor as per Comment \ref{comm:well_defined_amalgamation}.
    Let \( w \colon W \to P \) in \( \molecin{P} \), and suppose that \( \F \after \molecin{w'} \) is well defined for all proper submolecules \( w' \) of \( w \).
    Then either \( w \) is in \( \atomin{P} \), in which cases we are done since \( \F \after \molecin{w} = \F_x \) for some \( x \in P \), or \( w = w_1 \cp{k} w_2 \) for some decomposition \( W = W_1 \cp{k} W_2 \).
    Then, by inductive hypothesis, we have a strict functor \( (\F \after \molecin{w_1}, \F \after \molecin{w_2}) \colon \molecin{W_1} \cup \molecin{W_2} \to C \).
    By hypothesis, this extends to a strict functor \( \F' \colon \molecin{(W_1 \cp{k} W_2)} \to C \), which is equal to \( \F \after \molecin{w} \) by Lemma \ref{lem:strict_functor_determined_by_basis}.
    This shows that \( \F \after \molecin{w} \) is well defined and concludes the proof.
\end{proof}

\begin{prop} \label{prop:regular_directed_complex_stricter}
    Let \( P \) be a regular directed complex.
    Then \( \molecin{P} \) is a stricter \( \omega \)\nbd category.
\end{prop}
\begin{proof}
    Let \( Q \) be a regular directed complex, and consider a \( Q \)\nbd matching family \( \set{\F_x \colon \molecin{\imel{Q}{x}} \to \molecin{P}} \) in \( \molecin{P} \).
    We want to show that the candidate amalgamation \( \F \colon \molecin{Q} \to \molecin{P} \) is well defined, but for each \( w \colon W \to Q \) in \( \F(w) \) is given by the canonical morphism \( \colim_{x \in W} \pcell{\F_{w(x)}} \to P \), which is independent of the chosen decomposition of \( w \).
    We conclude by Lemma \ref{lem:stricter_iff_local_wrt_pasting}.
\end{proof}

\noindent Therefore, the functor \( \molecin{-} \colon \rdcpx \to \Comp \) factors through the subcategory \( \somegaCat \).

\begin{cor} \label{cor:regular_directed_complex_colimit_of_itself}
    Let \( P \) be a regular directed complex.
    Then in \( \somegaCat \),
    \begin{equation*}
        s_p \colon \colim_{x \in P} \molecin{\imel{P}{x}} \cong \molecin{P}.
    \end{equation*}
\end{cor}
\begin{proof}
    By the Yoneda Lemma and Lemma \ref{lem:stricter_iff_local_wrt_pasting}, \( \rc(s_P) \) is an isomorphisms in \( \somegaCat \).
    Since \( \rc \) is left adjoint, we conclude by Proposition \ref{prop:regular_directed_complex_stricter}.
\end{proof}

\begin{cor} \label{cor:molecin_preserves_pushout_inclusions}
    The functor \( \molecin{-} \colon \rdcpx \to \somegaCat \) preserves all pushouts of inclusions. 
\end{cor}

\begin{dfn} [Pasting in a stricter \( \omega \)\nbd category]
    Let \( C \) be a stricter \( \omega \)\nbd category, consider two pasting diagrams \( \F \colon \molecin{U} \to C \), \( \G \colon \molecin{V} \to C \), and \( k \geq 0 \) such that \( \bd{k}{+} \F = \bd{k}{-} \G \).
    Then we write \( \F \cp{k} \G \colon \molecin{(U \cp{k} V)} \to C \) for the strict functor determined by the universal properties of pushout given by Corollary \ref{cor:molecin_preserves_pushout_inclusions}.
    More generally, if a generalised pasting \( U \gencp{k} V \) given by a span \( (i \colon U \cap V \incl U, j \colon U \cap V \incl V) \) is defined and such that \( \F \after \molecin{i} = \G \after \molecin{j} \), we write \( \F \gencp{k} \G \) for the strict functor determined by universal property of the pushout.
    If the generalised pasting is given by a pasting at a submolecules \( U \cpsub{\iota} V \) or \( V \subcp{\iota} V \), we write accordingly \( \F \cpsub{\iota} \G \) and \( \G \subcp{\iota} \F \).
\end{dfn}

\begin{dfn} [Stricter \( n \)\nbd category]
    Let \( n \geq 0 \).
    A \emph{\( n \)\nbd composition structure} is a composition structure \( C \) such that for all globular cells \( c \in C \), we have \( \dim c \le n \).
    If \( C \) was a stricter \( \omega \)\nbd category, we speak of \emph{stricter \( n \)\nbd category}.
    We denote by \( \nComp{n} \) and \( \snCat{n} \) the full subcategories of \( \Comp \) and \( \somegaCat \) on \( n \)\nbd composition structures and stricter \( n \)\nbd categories, respectively. 
\end{dfn}

\begin{dfn} 
    The inclusion \( \iota_n \colon \nComp{n} \incl \Comp \) has a right adjoint \( \skel{n} \) defined by
    \begin{equation*}
        \skel{n}(C) \eqdef \set{c \in \C \mid \dim c \le n},
    \end{equation*}
    and a left adjoint \( \trunc{n} \) defined by
    \begin{equation*}
        \trunc{n}(C) \eqdef \skel{n - 1}(C) \cup \set{[c] \mid c \in C, \dim c = n},
    \end{equation*}
    where \( [-] \) denote the equivalence class on the globular cells of \( C \) of dimension \( n \) generated by \( \bd{}{-} d \sim \bd{}{+} d \) for all globular cells \( d \) of dimension \( n + 1 \). 
    By convention, \( \skel{-1}(C) = \emptyset \).
\end{dfn}

\begin{rmk}
    By \cite[Proposition 5.2.14]{hadzihasanovic2024combinatorics}, if \( P \) is a regular directed complex, \( \skel{n} \molecin{P} \) is naturally isomorphic to \( \molecin{(\skel{n}P)} \).
\end{rmk}

\begin{lem} \label{lem:stricter_n_iff_local_with_dim_le_n}
    Let \( C \) be an \( n \)\nbd composition structure.
    The following are equivalent.
    \begin{enumerate}
        \item \( C \) is a stricter \( n \)\nbd category;
        \item for all regular directed complex with \( \dim P \le n \), \( C \) is local with respect to \( s_P \);
        \item for all pairs of molecules \( U, V \) with \( \dim U, \dim V \le n \) and \( k \geq 0 \) such that \( U \cp{k} V \) is defined, each lifting problem
            \begin{center}
                \begin{tikzcd}
                    {\molecin{U} \cup \molecin{V}} & C \\
                    {\molecin{(U \cp{k} V)}}
                    \arrow[from=1-1, to=1-2]
                    \arrow[from=1-1, to=2-1]
                \end{tikzcd}
            \end{center}
            has a (necessarily unique) solution.
    \end{enumerate}
\end{lem}
\begin{proof}
    Suppose \( C \) is a stricter \( n \)\nbd category 
    In particular, \( C \) is a stricter \( \omega \)\nbd category, so by Lemma \ref{lem:stricter_iff_local_wrt_pasting}, the last two conditions holds.
    Now suppose the second condition holds, consider a regular directed complex \( P \) and a \( P \)\nbd matching family \( \set{\F_x \colon \molecin{\imel{P}{x}} \to C} \) with candidate amalgamation \( \F \).
    Restricting this matching family to \( \set{\F_x}_{x \in \gr{\le n}{P}} \), and using the assumption, we have an amalgamation 
    \begin{equation*}
        \gr{\le n}{\F} \colon \amalg_{x \in \gr{\le n}{P}} \F_x \colon \gr{\le n}{P} \to C.
    \end{equation*}
    Let \( x \in P \) with \( \dim x > n \).
    Then \( \dim \pcell{\F_x} \le n \), hence for any \( \a \in \set{-, +} \), \( \pcell{\bd{n}{\a} \F_x} = \pcell{\F(\bd{n}{\a} x \to P)} \).
    Using this fact, an induction on the submolecules of any \( w \colon W \to P \) shows that \( \F(w) = \gr{\le n}{\F}(\bd{n}{\a} w) \), proving that \( \F \) is well defined.
    This shows that \( C \) is stricter.
    The last condition is shown equivalent to the second one using a similar argument than in the proof of Lemma \ref{lem:stricter_iff_local_wrt_pasting}.
    This concludes the proof.
\end{proof}

\begin{lem} \label{lem:truncation_stricter_are_stricter}
    Let \( n \geq 0 \), and \( C \) be a stricter \( \omega \)\nbd category.
    Then \( \skel{n}(C) \) and \( \trunc{n}(C) \) are stricter \( n \)\nbd categories.
\end{lem}
\begin{proof}
    We use the second point of Lemma \ref{lem:stricter_n_iff_local_with_dim_le_n}.
    Let \( P \) be a regular directed complex with \( \dim P \le n \), and \( \set{\F_x \colon \molecin{\imel{P}{x}} \to \skel{n}(C)}_{x \in P} \) be a \( P \)\nbd matching family in \( \skel{n}(C) \).
    Then post-composing each \( \F_x \) by the counit \( \skel{n}(C) \to C \), and using the fact that \( C \) is stricter, the amalgamation defines a strict functor \( \F \colon \molecin{P} \to C \).
    Then, since \( \skel{n}(P) = P \), \( \skel{n}(\F) \) is the desired amalgamation of the matching family.

    Now consider a \( P \)\nbd matching family \( \set{\G_x \colon \molecin{\imel{P}{x}} \to \trunc{n}(C)}_{x \in P} \) in \( \trunc{n}C \).
    By definition of \( \trunc{n}(C) \), and since \( \skel{n - 1}(C) \subseteq \trunc{n}(C) \) is stricter, we have the amalgamation
    \begin{equation*}
        \gr{< n }{\G} \eqdef \amalg_{x \in \gr{< n}{P}} \G_x \colon \molecin{\gr{< n}{P}} \to \trunc{n}(C).
    \end{equation*}
    For each \( x \in P \) of dimension \( n \), choose a representative \( u_x \colon W_x \to C \) of the cell \( \pcell{\G_x} \) in \( \trunc{n}(C) \) (if \( \dim \pcell{\G_x} \le n \), then we mean that we let \( u_x \eqdef \pcell{\G_x} \)).
    We claim that \( \set{u_x}_{x \in P} \) give rise to a matching family \( \set{\G'_x \colon \molecin{\imel{P}{x}} \to C}_{x \in P} \), as per Remark \ref{rmk:data_matching family}.
    This is clear for all \( x \in \gr{< n}{P} \), since this is the data associated to the matching family \( \set{\G_x}_{x \in \gr{< n}{P}} \). 
    If \( \dim x = n \), then \( \bd{}{}\G'_x \colon \molecin{\bd{}{}\imel{P}{x}} \to C \) extends by letting \( \idd{x} \mapsto u_x \) to a strict functor \( \G'_x \colon \molecin{\imel{P}{x}} \to c \).
    Indeed, for all \( k < n \) and \( \a \in \set{-, +} \) we have \( \bd{k}{\a} u_x = \gr{< n}{G}(\bd{k}{\a} x \incl P ) \), since \( \bd{k}{\a} \pcell{\G_x} \) is independent of the chosen representative \( u_x \) of \( \pcell{\G_x} \).
    Thus \( \set{\G'_x}_{x \in P} \) is a \( P \)\nbd matching family in \( C \), which admits an amalgamation \( \G' \colon \molecin{P} \to C \). 
    Post-composing \( \G' \) with the unit \( \eta \colon C \to \trunc{n}(C) \), we obtain the strict functor \( \eta \after \G' \), which is the candidate amalgamation \( \G \) by Lemma \ref{lem:strict_functor_determined_by_basis}.
    This shows that \( \trunc{n}(C) \) is stricter, and concludes the proof.
\end{proof}

\begin{dfn}[\( n \)\nbd skeletong and \( n \)\nbd truncation.]
    Let \( n \geq 0 \) and \( C \) be a stricter \( n \)\nbd category.
    The \emph{\( n \)\nbd skeleton} of \( C \) is the stricter \( n \)\nbd category \( \skel{n}(C) \).
    The \emph{\( n \)\nbd truncation} of \( C \) is the stricter \( n \)\nbd category \( \trunc{n}(C) \).
\end{dfn}

\noindent Thus, the adjoint triple \( \trunc{n} \dashv \iota_n \dashv \skel{n} \) restricts the adjoint triple
\begin{center}
    \begin{tikzcd}
        {\snCat{n}} && \somegaCat.
        \arrow["{\iota_n}"{description}, from=1-1, to=1-3]
        \arrow["{\skel{n}}", shift left=2, curve={height=-12pt}, from=1-3, to=1-1]
        \arrow["{\trunc{n}}"', shift right=2, curve={height=12pt}, from=1-3, to=1-1]
    \end{tikzcd}
\end{center}
Notice that given an stricter \( \omega \)\nbd category \( C \), we have a chain of inclusions 
\begin{equation*}
    \skel{-1} C \incl \skel{0} C \incl \skel{1} C \incl \ldots \incl \skel{n} C \incl \ldots
\end{equation*}
whose colimit \( \somegaCat \) is \( C \).

The following definition is adapted from \cite[8.2.1]{hadzihasanovic2024combinatorics}.
\begin{dfn} [Cellular extension] \label{dfn:cellular_extension}
    Let \( C \) be a stricter \( \omega \)\nbd category.
    A \emph{cellular extension of \( C \)} is a stricter \( \omega \)\nbd category \( C_{\cls{S}} \) together with a pushout diagram 
    \begin{center}
        \begin{tikzcd}[column sep=large]
            {\coprod_{e \in \cls{S}} \bd{}{}U_e} & {\coprod_{e \in \cls{S}} \molecin{U_e}} \\
            C & {C_{\cls{S}}}
            \arrow[""{name=0, anchor=center, inner sep=0}, "{\molecin{\bd{e}{}}}", from=1-1, to=1-2]
            \arrow["{(\bd{}{}e)_{e \in \cls{S}}}"', from=1-1, to=2-1]
            \arrow["{(e)_{e \in \cls{S}}}", from=1-2, to=2-2]
            \arrow[from=2-1, to=2-2]
            \arrow["\lrcorner"{anchor=center, pos=0.125, rotate=180}, draw=none, from=2-2, to=0]
        \end{tikzcd}
    \end{center}
    in \( \somegaCat \), where, for each \( e \in \cls{S} \), \( U_e \) is an atom.
\end{dfn}

\begin{comm}  \label{comm:def_cellular_extension}
    This is a non-standard definition of cellular extension, which is a priori more general than the usual one, which requires each of the atoms \( U_e \) to be globes.
    However, by the following Lemma applied to the unique subdivision \( s \colon \dglobe{n} \sd U \), we may turn every cellular extension in this sense into one in the restricted sense. 
    See \cite[Comment 8.2.2]{hadzihasanovic2024combinatorics}. 
\end{comm}

\begin{dfn} [Generalised substitution]
    Let \( U, V, P \) be regular directed complexes, \( s \colon U \sd V \) be a subdivision with formal dual \( c \), and \( \iota \colon U \incl P \) be an inclusion.
    The \emph{generalised substitution of \( U \) for \( V \) in \( P \)} is the oriented graded poset \( \subs{P}{V}{U}_s \) whose underlying set is \( (P \setminus \iota(U)) \coprod V \), with partial order defined for each \( z \in \subs{P}{V}{U}_s  \) and \( \a \in \set{-, +} \), by
    \begin{equation*}
        \cofaces{}{\a} z \eqdef 
        \begin{cases}
            \cofaces{P}{\a} z, & \text{if } z \in P \setminus \iota(U), \\
            \cofaces{V}{\a} z + \cup \set{\cofaces{P}{\a} y \mid y = \iota(c(z)), \dim y = \dim z} & \text{if } z \in V.
        \end{cases}
    \end{equation*}
    This comes equipped with an evident commutative square
\end{dfn}

\begin{lem} \label{lem:generalised_substitution}
    Let \( U, V, P \) be regular directed complexes, \( s \colon U \sd V \) be a subdivision with formal dual \( c \), and \( \iota \colon U \incl P \) be an inclusion.
    Then \( \subs{P}{V}{U}_s \) is a regular directed complex, the formal dual of \( c' \colon \subs{P}{V}{U}_s \to P \) is a subdivision \( s' \), and the square of strict functors
    \begin{center}
        \begin{tikzcd}
            {\molecin{U}} & {\molecin{P}} \\
            {\molecin{V}} & {\molecin{({\subs P V U}_s)}}
            \arrow[""{name=0, anchor=center, inner sep=0}, "{{{\molecin{\iota}}}}", from=1-1, to=1-2]
            \arrow["{{{\molecin{s}}}}"', from=1-1, to=2-1]
            \arrow["{{{\molecin{s'}}}}", from=1-2, to=2-2]
            \arrow["{\molecin{\iota'}}"', from=2-1, to=2-2]
            \arrow["\lrcorner"{anchor=center, pos=0.125, rotate=180}, draw=none, from=2-2, to=0]
        \end{tikzcd}
    \end{center}
    is a pushout in \( \somegaCat \).
\end{lem}
\begin{proof}
    That \( \subs{P}{V}{U}_s \) is a regular directed complex and \( s' \colon P \sd \subs{P}{V}{U}_s \) is a subdivision follows from \cite[Proposition 1.46, Comment 1.48]{chanavat2025semistrict}.
    We prove that the square is a pushout.
    Consider a stricter \( \omega \)\nbd category \( C \), together with strict functors \( \F \colon \molecin{P} \to C \) and \( \G \colon \molecin{V} \to C \) such that \( \F \after \molecin{\iota} = \G \after \molecin{s} \).
    Let \( \set{\F_x}_{x \in P} \) and \( \set{\G_x}_{x \in V} \) be the matching family \( \F \) and \( \G \) are the amalgamation of, respectively.
    For each \( x \in {\subs P V U}_s \), we let 
    \begin{equation} \label{eq:H_matching_of}
        \fun{H}_x \eqdef 
        \begin{cases}
            \F_x & \text{if } x \in V \\
            \G_x & \text{else}.
        \end{cases}
    \end{equation}
    By construction, \( \set{\fun{H}}_{x \in {\subs P V U}_s} \) is a matching family in \( C \).
    Since \( C \) is stricter, it as an amalgamation \( \fun{H} \colon \molecin{({\subs P V U}_s)} \to C \).
    By Lemma \ref{lem:strict_functor_determined_by_basis}, \( \fun{H} \after \molecin{\iota'} = \F \) and \( \fun{H} \after \molecin{s'} = \G \).
    Any other strict functor \( \fun{H}' \) with this property is necessarily the amalgamation of the same matching family.
    By Lemma \ref{lem:at_most_one_lift}, this proves uniqueness and concludes the proof. 
\end{proof}

\begin{cor} \label{cor:pushout_principal_cell}
    Let \( s \colon U \sd V \) be a subdivision between atoms.
    Then the square
    \begin{center}
        \begin{tikzcd}
            {\molecin{(\bd{}{}U)}} & {\molecin{U}} \\
            {\molecin{(\bd{}{}V)}} & {\molecin{V}}
            \arrow[from=1-1, to=1-2]
            \arrow["{\molecin{(\restr{s}{\bd{}{}U})}}"', from=1-1, to=2-1]
            \arrow["{\molecin{s}}", from=1-2, to=2-2]
            \arrow[from=2-1, to=2-2]
        \end{tikzcd}
    \end{center}
    is a pushout square in \( \somegaCat \).
\end{cor}
\begin{proof}
    By Lemma \ref{lem:generalised_substitution}.
\end{proof}

\begin{dfn} [Stricter polygraph]
    A \emph{stricter polygraph} is a stricter \( \omega \)\nbd category \( C \), together with, for each \( n \geq 0 \), a pushout diagram
    \begin{center}
        \begin{tikzcd}[column sep=large]
            {\coprod_{e \in \cls{S}_n} \bd{}{}U_e} & {\coprod_{e \in \cls{S}_n} \molecin{U_e}} \\
            \skel{n - 1}C & {\skel{n} C}
            \arrow[""{name=0, anchor=center, inner sep=0}, "{\molecin{\bd{e}{}}}", from=1-1, to=1-2]
            \arrow["{(\bd{}{}e)_{e \in \cls{S}_n}}"', from=1-1, to=2-1]
            \arrow["{(e)_{e \in \cls{S}_n}}", from=1-2, to=2-2]
            \arrow[from=2-1, to=2-2]
            \arrow["\lrcorner"{anchor=center, pos=0.125, rotate=180}, draw=none, from=2-2, to=0]
        \end{tikzcd}
    \end{center}
    in \( \somegaCat \), exhibiting \( \skel{n}C \) as a cellular extension of \( \skel{n - 1}C \), and such that for each \( e \in \cls{S}_n \), \( \dim e = n \).
    The set \( \cls{S} = \coprod_{k \geq 0} \cls{S}_k \) is called the \emph{generating set} of \( C \).
    More generally, we say that a strict functor \( f \colon A \to C \) of composition structure is a \emph{relative stricter polygraph} if \( f \) can be obtain as the transfinite composition of cellular extensions of \( A \).
\end{dfn}

\begin{lem} \label{lem:stricter_regular_complex_are_stricter_polygraph}
    Let \( P \) be a regular direct complex.
    Then \( \molecin{P} \) is a stricter polygraph
\end{lem}
\begin{proof}
    By Corollary \ref{cor:molecin_preserves_pushout_inclusions}, the square
    \begin{center}
        \begin{tikzcd}
            {\coprod_{x \in \gr{n}P} \molecin{\bd{}{}\imel P x}} & {\coprod_{x \in \gr{n}P} \molecin{\imel P x}} \\
            {\skel{n - 1}P} & {\skel{n} P}
            \arrow["{{\molecin{\bd{e}{}}}}", from=1-1, to=1-2]
            \arrow["{{(\bd{}{}\molecin{\mapel x})_{x \in \gr n P}}}"', from=1-1, to=2-1]
            \arrow["{{(\molecin{\mapel x})_{x \in \gr n P}}}", from=1-2, to=2-2]
            \arrow[from=2-1, to=2-2]
        \end{tikzcd}
    \end{center}
    is a pushout in \( \somegaCat \).
    This concludes the proof.
\end{proof}

\noindent We conclude this section by describing \( \somegaCat \) as a subcategory of a category of presheaves.

\begin{dfn}
    We let \( \omegaReg \) be the full subcategories of \( \somegaCat \) on stricter \( \omega \)\nbd categories of the form \( \molecin{P} \), for \( P \) a \emph{finite} regular directed complex.
\end{dfn}

\begin{comm}
    The finiteness condition is only here to ensure that \( \omegaReg \) is a small category. 
\end{comm}

\begin{dfn} [Rewrite and pasting condition]
    Let \( X \colon \opp{\omegaReg} \to \Set \) be a presheaf. 
    We say that \( X \) satisfies \emph{the rewrite condition} if for all atoms \( U \) of dimension \( n \), letting \( s \colon \dglobe{n} \sd U \) be the unique subdivision, \( X \) sends the square
    \begin{center}
        \begin{tikzcd}
            {\molecin{(\bd{}{}\dglobe{n})}} & {\molecin{\dglobe{n}}} \\
            {\molecin{(\bd{}{}U)}} & {\molecin{U}}
            \arrow[""{name=0, anchor=center, inner sep=0}, from=1-1, to=1-2]
            \arrow["{{\molecin{(\restr{s}{\bd{}{}\dglobe{n}})}}}"', from=1-1, to=2-1]
            \arrow["{{\molecin{s}}}", from=1-2, to=2-2]
            \arrow[from=2-1, to=2-2]
        \end{tikzcd}
    \end{center}
    to a pullback square.
    We say that \( X \) satisfies \emph{pasting condition} if for all finite regular directed complex \( P \), \( X \) sends the cone 
    \begin{equation*}
        \set{\molecin{\imel{P}{x}} \to \molecin{P}}_{x \in P}
    \end{equation*}
    over \( \molecin{P} \) to a limit cone.
    We let \( \omegaRegSet \) be the full subcategory of \( [\opp{\omegaReg}, \Set] \) on presheaves satisfying the rewrite and pasting conditions.
\end{dfn}

\begin{rmk}
    Let \( X \colon \opp{\omegaReg} \to \Set \) be a presheaf, and \( P \) a finite regular directed complex. 
    We may still call a cone \( \set{t_x \colon \molecin{\imel{P}{x}} \to X}_{x \in P} \) a \( P \)\nbd matching family in \( X \), and an extension along \( s_P \) an amalgamation.
    Then \( X \) satisfies the pasting condition if for all finite regular directed complexes \( P \), all \( P \)\nbd matching families in \( X \) have a unique amalgamation.
\end{rmk}

\begin{dfn}
    Let \( X \colon \opp{\omegaReg} \to \Set \) be a presheaf, \( U \) be a molecule of dimension \( n \geq 0 \), and \( t \colon \molecin{U} \to X \) a natural transformation.
    Then there exists a unique strict functor \( \F \colon \molecin{\dglobe{n}} \to \molecin{U} \) classifying \( \idd{U} \).
    The \emph{principal cell of \( t \)} is the natural transformation \( \pcell{t} \eqdef t \after \F \colon \molecin{\dglobe{n}} \to X \).
\end{dfn}

\begin{dfn}
    Let \( X \colon \opp{\omegaReg} \to \Set \) be a presheaf satisfying the rewrite and pasting conditions.
    The \emph{composition structure associated to \( X \)} is the composition structure 
    \begin{equation*}
        \phi(X) \eqdef \coprod_{n \geq 0} X(\dglobe{n}),
    \end{equation*}
    whose boundary operators are induced by the strict functor
    \begin{equation*}
        \bd{k}{\a} \colon \molecin{\dglobe{k}} \to \molecin{\dglobe{n}}, 
    \end{equation*}
    and \( k \)\nbd composition operation is induced by the strict functor
    \begin{equation*}
        \molecin{\dglobe{n}} \to \molecin{(\dglobe{n} \cp{k} \dglobe{n})},
    \end{equation*}
    classifying \( \idd{(\dglobe{n} \cp{k} \dglobe{n})} \).
    This extends to a functor 
    \begin{equation*}
        \phi \colon \omegaRegSet \to \Comp.
    \end{equation*}
\end{dfn}

\begin{lem} \label{lem:stricter_hom_set_iso_presheaf}
    Let \( X \colon \opp{\omegaReg} \to \Set \) be a presheaf satisfying the rewrite and pasting conditions, and \( P \) be a finite regular directed complex.
    Then there is a bijection
    \begin{equation*}
        X(\molecin{P}) \cong \Comp(\molecin{P}, \phi(X)),
    \end{equation*}
    natural in \( X \) and \( U \).
\end{lem}
\begin{proof}
    Let \( t \in X(\molecin{P}) \), which by the Yoneda Lemma, is a natural transformation \( t \colon \molecin{P} \to X \).
    We define \( \F \colon \molecin{P} \to \phi(X) \) by sending \( w \) in \( \molecin{P} \) to \( \pcell{t \after \molecin{w}} \), which is, by definition of \( \phi(X) \), a strict functor of compositions structures.
    Conversely, let \( \F \colon \molecin{P} \to \phi(X) \) be a strict functor, and let \( \set{\F_x}_{x \in P} \) be the matching family of which it is the amalgamation.
    We construct by induction on \( n \), a matching family \( \set{t_x \colon \molecin{\imel{P}{x}} \to X}_{x \in \gr{\le n}{P}} \) such that \( \pcell{t_x} = \pcell{\F_x} \).    
    For \( x \in \gr{0}{P} \), we let \( t_x \colon \molecin{\pt} \to X \) classifying \( \pcell{\F_x} \in X(\globe{0}) \).
    Let \( n > 0 \) and suppose inductively that the matching family is constructed for \( x \in \gr{< n}{P} \).
    Let \( x \in \gr{n}{P} \), and let \( s \colon \dglobe{n} \sd \imel{P}{x} \) be the unique subdivision.
    Then we have a commutative diagram
    \begin{center}
        \begin{tikzcd}
            {\molecin{(\bd{}{}\dglobe{n})}} & {\molecin{\dglobe{n}}} \\
            {\molecin{(\bd{}{}\imel{P}{x})}} & X
            \arrow[from=1-1, to=1-2]
            \arrow["{{\molecin{(\restr{s}{\bd{}{}\dglobe{n}})}}}"', from=1-1, to=2-1]
            \arrow["{\pcell{\F_x}}", from=1-2, to=2-2]
            \arrow["{t_{\bd{}{} \imel{P}{x}}}"', from=2-1, to=2-2]
        \end{tikzcd}
    \end{center}
    where \( t_{\bd{}{} \imel{P}{x}} \) is the amalgamation of \( \set{t_y \colon \molecin{\imel{P}{y}} \to X}_{y \in \bd{}{} \imel{P}{x}} \).
    Since \( X \) satisfies the rewrite condition, this data defines a unique natural transformation \( t_x \colon \molecin{\imel{P}{x}} \to X \) such that \( t_x \after \molecin{s} = \pcell{\F_x} \).
    This concludes the construction of the matching family.
    Since \( X \) satisfies the pasting condition, the matching family admits an amalgamation \( t \colon \molecin{P} \to X \), defining the desired element of \( X(\molecin{P}) \).
    It is straightforwards to check naturality and that those two transformations are inverse of each other.
\end{proof}

\begin{prop} \label{prop:stricter_cat_are_local_presheaves}
    The categories \( \omegaRegSet \) and \( \somegaCat \) are equivalent, realised by the functors \( X \mapsto \phi(X) \) and \( C \mapsto \somegaCat(\molecin{-}, C) \). 
\end{prop}
\begin{proof}
    For all presheaf \( X \) in \( \omegaRegSet \), we claim that \( \phi(X) \) is a stricter \( \omega \)\nbd category.
    Indeed, a \( U \)\nbd matching family \( \set{\F_x \colon \molecin{\imel{U}{x}} \to \phi(X)}_{x \in U} \) defines by Lemma \ref{lem:stricter_hom_set_iso_presheaf} a matching family \( \set{t_x \colon \molecin{\imel{U}{x}} \to X} \) which has an amalgamation \( t \colon \molecin{U} \to X \), which, by Lemma \ref{lem:stricter_hom_set_iso_presheaf} again, defines a strict functor \( \F \colon \molecin{U} \to \phi(X) \), which is the amalgamation of \( \set{\F_x}_{x \in U} \). 
    Thus \( X \mapsto \phi(X) \) defines a functor \( \omegaRegSet \to \somegaCat \).
    We construct an inverse up to natural isomorphism to \( \phi \).
    Let \( C \) be a stricter \( \omega \)\nbd category.
    Then the presheaf defined by \(\molecin{U} \mapsto \somegaCat(\molecin{U}, C) \) is \( S \)\nbd local by definition, and satisfy the rewrite condition by Corollary \ref{cor:pushout_principal_cell}.
    This defines a functor \( \psi \colon \somegaCat \to \omegaRegSet \).
    One sees directly that \( \phi \after \psi \) is isomorphic to the identity on \( \somegaCat \). 
    By Lemma \ref{lem:stricter_hom_set_iso_presheaf}, it is also the case of \( \psi \after \phi \).
    This concludes the proof.
\end{proof}

% \begin{cor} \label{cor:diagrams_are_dense} 
%     Let \( C \) be a stricter \( \omega \)\nbd category.
%     Then the canonical strict functor
%     \begin{equation*}
%         \phi \colon \colim_{u \in \Dgm(C)} \molecin{U} \to C,t
%     \end{equation*}
%     is an isomorphism.
%     That is, the category \( \omegaMol \) is dense in \( \somegaCat \).
% \end{cor}
% \begin{proof}
%     Follows from the density formula for presheaves and Proposition \ref{prop:stricter_cat_are_local_presheaves}.
% \end{proof}

\subsection{Gray product of stricter \texorpdfstring{$\omega$}{ω}-categories}

Recall that if \( P, Q \) are regular directed complexes, the basis \( \atomin{(P \gray Q)} \) of \( \molecin{P \gray Q} \) is given exactly by the morphisms \( u \gray v \) for \( u \in \atomin{P} \) and \( v \in \atomin{Q} \).  

\begin{dfn} [Gray product] \label{dfn:gray_product_of_stricter_regular_complexes}
    Let \( P, Q \) be regular directed complexes.
    The \emph{Gray product} of \( \molecin{P} \) and \( \molecin{Q} \) is the stricter \( \omega \)\nbd category
    \begin{equation*}
        \molecin{P} \gray \molecin{Q} \eqdef \molecin{(P \gray Q)}.
    \end{equation*}
    If \( \F \colon \molecin{P} \to \molecin{P'} \) and \( \G \colon \molecin{Q} \to \molecin{P'} \) are two strict functors, we let 
    \begin{equation*}
        \F \gray \G \colon \atomin{P} \times \atomin{Q} \to  \molecin{P'} \gray \molecin{Q'}
    \end{equation*}
    defined by sending a pair \( (u, v) \) to \( \F(u) \gray \G(v) \).
\end{dfn}

\begin{lem} \label{lem:gray_of_strict_functors}
    Let \( \F \colon \molecin{P} \to \molecin{P'} \) and \( \G \colon \molecin{Q} \to \molecin{P'} \) be strict functors. 
    Then \( \F \gray \G \) extends to a strict functor 
    \begin{equation*}
        \F \gray \G \colon \molecin{P} \gray \molecin{Q} \to \molecin{P'} \gray \molecin{Q'}.
    \end{equation*}
\end{lem}
\begin{proof}
    Suppose that \( \F \) and \( \G \) are given by the matching families \( \set{\F_x}_{x \in P} \) and \( \set{\G_y}_{y \in Q} \) respectively.
    We show by induction on \( n \) that the data of \( \set{\pcell{\F_x} \gray \pcell{\G_y}}_{\dim (x, y) \le n} \) determines, as per Remark \ref{rmk:data_matching family}, a \( (\gr{\le n}{(P \gray Q)}) \)\nbd matching family in \( \molecin{(P' \gray Q')} \).
    When \( n = 0 \), the statement is clear. 
    Inductively, let \( n > 0 \) and let \( (x, y) \in P \gray Q \) of dimension \( n \). 
    Then by inductive hypothesis, \( \set{\F_{x'} \gray G_{y'}}_{(x', y') < (x, y)} \) forms a \( \bd{}{} (x, y) \)\nbd matching family in \( \molecin{(P' \gray Q')} \).
    By Proposition \ref{prop:regular_directed_complex_stricter}, this matching family has an amalgamation \( \bd{}{}\fun{H} \colon \molecin{(\bd{}{} (x, y))} \to \molecin{(P' \gray Q')} \).
    We show that \( \bd{}{} \fun{H} \) extends to a strict functor \( \fun{H} \colon \molecin{\clset{(x, y)}} \to \molecin{(P' \gray Q')} \) by sending the principal cell to \( \pcell{\F_x} \gray \pcell{\G_y} \).
    Then for \( k \geq 0 \) and \( \a \in \set{-, +} \), we have
    \begin{align*}
        \bd{k}{\a} (\pcell{\F_x} \gray \pcell{\G_y}) &= \bigcup_{i = 1}^k \bd{i}{\a} \pcell{\F_x} \gray \bd{k - i}{(-)^i\a} \pcell{\G_y} \\
                                       &= \bigcup_{i = 1}^k \pcell{\bd{i}{\a} \F_x} \gray \pcell{\bd{k - i}{(-)^i\a} \G_y} \\
                                       &= \fun{H}(\bd{k}{\a} \idd{(x, y)}).
    \end{align*}
    Thus, \( \fun{H} \) is a morphism of the underlying globular graph, and since \( \idd{(x, y)} \) is the only globular cell of dimension \( \dim (x, y) \) in \( \molecin{(\clset{(x, y)})} \), this is enough to conclude that \( \fun{H} \) is a strict functor.
    This concludes the induction and shows that \( \set{\pcell{\F_x} \gray \pcell{\G_y}}_{(x, y) \in P \gray Q} \) determines a matching family in \( \molecin{(P' \gray Q')} \).
    By Proposition \ref{prop:regular_directed_complex_stricter}, this matching family has an amalgamation \( \F \gray \G \colon \molecin{P} \gray \molecin{Q} \to \molecin{P'} \gray \molecin{Q'} \).
    This concludes the proof.

\end{proof}

\begin{cor} \label{cor:gray_prodcut_omegareg_monoidal}
    The Gray product determines a monoidal structure on the category \( \omegaReg \), whose monoidal unit is the terminal stricter \( \omega \)\nbd category \( \globe{0} \).
\end{cor}
\begin{proof}
    Lemma \ref{lem:gray_of_strict_functors} shows that \( - \gray - \) is well defined on strict functors. 
    Since for all regular directed complexes \( P \), \( \pt \gray P = P = P \gray \pt \), we deduce that the monoidal unit is \( \molecin{\pt} = \globe{0} \).
    Functoriality is straightforward.
    This concludes the proof.
\end{proof}

\begin{rmk}
    If \( \F \colon \molecin{P} \to \molecin{P'} \) and \( \G \colon \molecin{Q} \to \molecin{P'} \) are strict functors, \( u \in \molecin{P} \), and \( v \in \molecin{Q} \), then 
    \begin{equation*}
       (\F \gray \G)(u \gray v) = \F(u) \gray \G(v). 
    \end{equation*}
\end{rmk}

% \begin{rmk}
%     Since the point is a molecule, and the Gray product of two molecules is a molecule, the category \( \omegaMol \) inherit from \( \omegaReg \) of the monoidal structure given by the Gray product. 
% \end{rmk}

\begin{lem} \label{lem:rewrite_condition_for_gray}
    Let \( P \) be a regular directed complex, and \( s \colon U \sd V \) be a subdivision between atoms.
    Then the square
    \begin{center}
        \begin{tikzcd}
            {\molecin{(P \gray \bd{}{}U)}} & {\molecin{(P \gray U)}} \\
            {\molecin{(P \gray \bd{}{}V)}} & {\molecin{(P \gray V)}}
            \arrow[from=1-1, to=1-2]
            \arrow["{\molecin{(\restr{P \gray s}{\bd{}{}U})}}"', from=1-1, to=2-1]
            \arrow["{{\molecin{P \gray s}}}", from=1-2, to=2-2]
            \arrow[from=2-1, to=2-2]
        \end{tikzcd}
    \end{center}
    is a pushout square in \( \somegaCat \).
\end{lem}
\begin{proof}
    By construction, \( P \gray V \) is the generalised substitution of \( P \gray (\bd{}{} U) \) for \( P \gray (\bd{}{} V) \) in \( P \gray U \).
    We conclude by Lemma \ref{lem:generalised_substitution}.
    % Recall from \cite[Lemma 7.2.8]{hadzihasanovic2024combinatorics} that the Gray product preserves pushout of inclusions of regular directed complexes.
    % Thus the square is a pushout if and only if for all \( x \in P \), the square
    % \begin{center}
    %     \begin{tikzcd}
    %         {\molecin{(\imel{P}{x} \gray \bd{}{}U)}} & {\molecin{(\imel{P}{x} \gray U)}} \\
    %         {\molecin{(\imel{P}{x} \gray \bd{}{}V)}} & {\molecin{(\imel{P}{x} \gray V)}}
    %         \arrow[from=1-1, to=1-2]
    %         \arrow["{\molecin{(\restr{\imel{P}{x} \gray s}{\bd{}{}U})}}"', from=1-1, to=2-1]
    %         \arrow["{{\molecin{\imel{P}{x} \gray s}}}", from=1-2, to=2-2]
    %         \arrow[from=2-1, to=2-2]
    %     \end{tikzcd}
    % \end{center}
    % is a pushout.
    % We prove the statement by induction on \( n \eqdef \dim P \).
    % If \( n = 0 \), this is the content of Corollary \ref{cor:pushout_principal_cell}.
    % Inductively, let \( n > 0 \) and \( x \in \gr{n}{P} \).
    % Let \( W \eqdef \imel{P}{x} \).
    % By Corollary \ref{cor:pushout_principal_cell} on \( W \gray s  \) and the pasting law for pushouts, it is enough to show that 
    % \begin{center}
    %     \begin{tikzcd}
    %         {\molecin{(W \gray \bd{}{} U)}} & {\molecin{(\bd{}{}(W \gray U))}} \\
    %         {\molecin{(W \gray \bd{}{} V)}} & {\molecin{(\bd{}{} (W \gray V))}}
    %         \arrow[from=1-1, to=1-2]
    %         \arrow[from=1-1, to=2-1]
    %         \arrow[from=1-2, to=2-2]
    %         \arrow[from=2-1, to=2-2]
    %     \end{tikzcd}
    % \end{center}
    % is a pushout.
    % Since 
    % \begin{equation*}
    %     \bd{}{}(W \gray U) = (W \gray \bd{}{} U) \cup (\bd{}{} W \gray U),\text{ and } \bd{}{}(W \gray V) = (W \gray \bd{}{} V) \cup (\bd{}{} W \gray V),
    % \end{equation*}
    % we conclude by Corollary \ref{cor:molecin_preserves_pushout_inclusions}, the inductive hypothesis on \( (\bd{}{} W) \gray s \), and a double application of the pasting law for pushouts. 
\end{proof}

\begin{dfn} 
    Let \( P \) be a regular directed complex and \( C \) be a stricter \( \omega \)\nbd category.
    By \cite[Lemma 7.2.8]{hadzihasanovic2024combinatorics} and Lemma \ref{lem:rewrite_condition_for_gray}, the presheaf on \( \omegaReg \) given by
    \begin{equation*}
        \molecin{U} \mapsto \somegaCat(\molecin{(P \gray U)}, C) 
    \end{equation*}
    satisfies the rewrite and pasting conditions.
    By Proposition \ref{prop:stricter_cat_are_local_presheaves}, this defines the stricter \( \omega \)\nbd category 
    \begin{equation*}
        \homlax(\molecin{P}, C),
    \end{equation*}
    whose globular \( n \)\nbd cells are strict functors \( \F \colon \molecin{(P \gray \dglobe{n})} \to C \).
    Dually, we also define the stricter \( \omega \)\nbd category
    \begin{equation*}
        \homcolax(\molecin{P}, C),
    \end{equation*}
    whose globular \( n \)\nbd cells are strict functors \( \F \colon \molecin{(\dglobe{n} \gray P)} \to C \).
\end{dfn}

\begin{lem}
    Let \( P, Q \) be finite regular directed complexes and \( C \) be a stricter \( \omega \)\nbd category.
    Then there are bijections
    \begin{align*}
        \somegaCat(\molecin{(P \gray Q)}, C) &\cong \somegaCat(\molecin{P}, \homcolax(\molecin{Q}, C)) \\
                                             &\cong \somegaCat(\molecin{Q}, \homlax(\molecin{P}, C)),
    \end{align*}
    natural in \( \molecin{P}, \molecin{Q} \) and \( C \).
\end{lem}
\begin{proof}
    Follows directly by Proposition \ref{prop:stricter_cat_are_local_presheaves}.
\end{proof}

\noindent Applying \cite[Th\'eor\`eme 5.3]{ara2020joint} together with the previous result, we get the following definition.
\begin{dfn} [Gray product of stricter \( \omega \)\nbd categories] \label{dfn:gray_product_stricter_categories}
    Let \( C, D \) be stricter \( \omega \)\nbd categories.
    The Gray product of \( C \) and \( D \) is the stricter \( \omega \)\nbd category
    \begin{equation*}
        \colim_{\substack{u \in \molecin{P} \to C \\ v \in\molecin{Q} \to D}} \molecin{(P \gray Q)},
    \end{equation*}
    where \( P \) and \( Q \) are finite regular directed complexes.
    This determines a biclosed monoidal structure on \( \somegaCat \) whose monoidal unit is the terminal stricter \( \omega \)\nbd category \( \globe{0} \) and such that the inclusion \( \omegaReg \incl \somegaCat \) is strong monoidal.
\end{dfn}

\begin{rmk}
    Let \( P, Q \) be regular directed complexes.
    Since \( - \gray - \) is biclosed and using Corollary \ref{cor:regular_directed_complex_colimit_of_itself}, we get
    \begin{align*}
        \molecin{P} \gray \molecin{Q} &\cong \colim_{x \in P, y \in Q} \molecin{\imel{P}{x}} \gray \molecin{\imel{Q}{y}} \\
                                      &\cong \colim_{(x, y) \in P \gray Q} \molecin{(\imel{P}{x} \gray \imel{Q}{y})} \\
                                      &= \molecin{(P \gray Q)}.
    \end{align*}
    Thus Definition \ref{dfn:gray_product_of_stricter_regular_complexes} and Definition \ref{dfn:gray_product_stricter_categories} agree for not necessarily finite regular directed complexes.
\end{rmk}

\subsection{Strict and stricter categories}

The goal of this section is to clarify the relationship between strict and stricter categories.

\begin{dfn} [Theta]
    Let \( U \) be a molecule.
    We say that \( U \) is a \emph{theta} if for all \( x \in U \), \( \clset{x} \) is a globe.
    We write \( \Theta \) for the full subcategory of \( \omegaReg \) on \( \molecin{U} \), for \( U \) a theta.
\end{dfn}

\begin{rmk} \label{rmk:theta_is_theta}
    By \cite[Corollary 9.1.29]{hadzihasanovic2024combinatorics}, a molecule is a theta if and only if it is a pasting of globes.
    Therefore, by \cite[Theorem 8.2.14, Lemma 9.1.16]{hadzihasanovic2024combinatorics}, the category \( \Theta \) is isomorphic to Joyal's \( \Theta \)\nbd category, the full subcategory of the category of strict \( \omega \)\nbd categories on pastings of globes.  
\end{rmk}

\begin{dfn} [Strict \( \omega \)\nbd category]
    We define \( S^\Theta \) to be the subset of \( S \) defined by 
    \begin{equation*}
        S^\Theta \eqdef \set{s_U \colon \colim_{x \in U} \imel{U}{x} \to \molecin{U} \mid U \text{ theta}}.
    \end{equation*}  
    A \emph{strict \( \omega \)\nbd category} is a composition structure \( C \) which is local with respect to \( S^\Theta \).
    We write \( \omegaCat \) for the category of strict \( \omega \)\nbd categories and strict functors.
\end{dfn}

\begin{rmk}
    By \cite[Theorem 1.12]{berger2002cellular}, this definition is equivalent to the standard definition.
\end{rmk}

\begin{prop} \label{prop:stricter_are_strict}
    Let \( C \) be a stricter \( \omega \)\nbd category.
    Then \( C \) is a strict \( \omega \)\nbd category.
\end{prop}
\begin{proof}
    Since \( C \) is local with respect to \( S \), it is in particular local with respect to \( S^\Theta \subseteq S \).
\end{proof} 

\noindent Akin to stricter \( \omega \)\nbd categories, strict \( \omega \)\nbd categories are a reflective subcategory of \( \Comp \).
By Proposition \ref{prop:stricter_are_strict}, we have a sequence of full subcategory inclusions
\begin{equation*}
     \somegaCat \incl \omegaCat \incl \Comp.
\end{equation*}
We define
\begin{equation*}
    \rcs \colon \omegaCat \to \somegaCat
\end{equation*}
to be the functor applying the reflector \( \rc \colon \Comp \to \somegaCat  \) to the underlying composition structure of a strict \( \omega \)\nbd category, which is left adjoint and exhibit \( \somegaCat \) as a reflective subcategory of \( \omegaCat \).

\begin{dfn} [Strict \( n \)\nbd categories]
    Let \( n \in \mathbb{N} \).
    A \emph{strict \( n \)\nbd category} is an strict \( \omega \)\nbd category which is also a \( n \)\nbd composition structure.
    We write \( \nCat{n} \) for the full subcategory of \( \omegaCat \) on strict \( n \)\nbd categories.
\end{dfn}

\begin{rmk}
    If \( C \) is a strict \( n \)\nbd category, then the stricter \( \omega \)\nbd category \( \rcs C \) is a stricter \( n \)\nbd category.
\end{rmk}

\noindent As for the case of stricter \( \omega \)\nbd categories, we have a \( n \)\nbd skeleton and \( n \)\nbd truncation functors \( \skel{n}, \trunc{n} \colon \omegaCat \to \nCat{n} \) fitting in a commutative diagram
\begin{center}
    \begin{tikzcd}
        {\nCat n} & \omegaCat \\
        {\snCat n} & \somegaCat.
        \arrow[""{name=0, anchor=center, inner sep=0}, hook, from=1-1, to=1-2]
        \arrow[""{name=1, anchor=center, inner sep=0}, "\rcs"', shift right=2, from=1-1, to=2-1]
        \arrow[""{name=2, anchor=center, inner sep=0}, "{\trunc n}"', shift right=3, from=1-2, to=1-1]
        \arrow[""{name=3, anchor=center, inner sep=0}, "{\skel n}", shift left=3, from=1-2, to=1-1]
        \arrow[""{name=4, anchor=center, inner sep=0}, "\rcs"', shift right=2, from=1-2, to=2-2]
        \arrow[""{name=5, anchor=center, inner sep=0}, shift right=2, hook, from=2-1, to=1-1]
        \arrow[""{name=6, anchor=center, inner sep=0}, hook, from=2-1, to=2-2]
        \arrow[""{name=7, anchor=center, inner sep=0}, shift right=2, hook, from=2-2, to=1-2]
        \arrow[""{name=8, anchor=center, inner sep=0}, "{\trunc n}"', shift right=3, from=2-2, to=2-1]
        \arrow[""{name=9, anchor=center, inner sep=0}, "{\skel n}", shift left=3, from=2-2, to=2-1]
        \arrow["\dashv"{anchor=center}, draw=none, from=1, to=5]
        \arrow["\dashv"{anchor=center, rotate=-90}, draw=none, from=0, to=3]
        \arrow["\dashv"{anchor=center}, draw=none, from=4, to=7]
        \arrow["\dashv"{anchor=center, rotate=-90}, draw=none, from=2, to=0]
        \arrow["\dashv"{anchor=center, rotate=-90}, draw=none, from=6, to=9]
        \arrow["\dashv"{anchor=center, rotate=-90}, draw=none, from=8, to=6]
    \end{tikzcd}
\end{center}

\begin{dfn} [Polygraph]
    A \emph{polygraph} is a strict \( \omega \)\nbd category \( C \), together with, for each \( n \geq 0 \), a pushout diagram
    \begin{center}
        \begin{tikzcd}[column sep=large]
            {\coprod_{e \in \cls{S}_n} \bd{}{}U_e} & {\coprod_{e \in \cls{S}_n} \molecin{U_e}} \\
            \skel{n - 1}C & {\skel{n} C}
            \arrow[""{name=0, anchor=center, inner sep=0}, "{\molecin{\bd{e}{}}}", from=1-1, to=1-2]
            \arrow["{(\bd{}{}e)_{e \in \cls{S}_n}}"', from=1-1, to=2-1]
            \arrow["{(e)_{e \in \cls{S}_n}}", from=1-2, to=2-2]
            \arrow[from=2-1, to=2-2]
            \arrow["\lrcorner"{anchor=center, pos=0.125, rotate=180}, draw=none, from=2-2, to=0]
        \end{tikzcd}
    \end{center}
    in \( \omegaCat \), exhibiting \( \skel{n}C \) as a cellular extension of \( \skel{n - 1}C \), and such that for each \( e \in \cls{S}_n \), \( \dim e = n \).
    The set \( \cls{S} = \coprod_{k \geq 0} \cls{S}_k \) is called the \emph{generating set} of \( C \).
    More generally, we say that a strict functor \( f \colon A \to C \) of composition structure is a \emph{relative polygraph} if \( f \) can be obtain as the transfinite composition of cellular extensions of \( A \).
\end{dfn}

\begin{rmk}
    Similar to Comment \ref{comm:def_cellular_extension}, this definition is equivalent to the usual definition of polygraphs.
\end{rmk}

\begin{lem} \label{lem:reflection_of_polygraphs_are_stricter_polygraphs}
    Let \( f \colon C \to D \) be a relative polygraph.
    Then \( \rcs f \) is a relative stricter polygraph.
\end{lem}
\begin{proof}
    Clear by Proposition \ref{prop:regular_directed_complex_stricter} and since \( \rcs \) is left adjoint, 
\end{proof}

Next, recall from \cite[Appendice A]{ara2020joint} that strict \( \omega \)\nbd categories also support a Gray product, defined along a similar method.

\begin{prop} \label{prop:reflection_to_stricter_monoidal}
    The functor \( \rcs \colon \omegaCat \to \somegaCat \) is strong monoidal with respect to the Gray product on stricter and strict \( \omega \)\nbd categories.
\end{prop}
\begin{proof}
    Since \( \rcs \) is left adjoint, and since \( \Theta \) are dense in \( \omegaCat \) by \cite[Proposition 4.6]{ara2020joint}, it is enough to show that for \( U, V \) two thetas, \( \rcs (\molecin{U} \gray \molecin{V}) \) is naturally isomorphic to \( \molecin{U \gray V} \).
    By \cite[Lemma 9.1.16, Proposition 11.2.36]{hadzihasanovic2024combinatorics}, \( \molecin{U} \gray \molecin{V} \cong \molecin{(U \gray V)} \) in \( \omegaCat \).
    Since \( \molecin{U \gray V} \) is stricter by Proposition \ref{prop:regular_directed_complex_stricter}, we conclude.
\end{proof}

\noindent We conclude this section by showing a partial converse to Proposition \ref{prop:stricter_are_strict}, and exhibiting a point of divergence between strict and stricter \( \omega \)\nbd categories.

\begin{thm}\label{thm:strict_le_3_are_stricter}
    Let \( n \le 3 \), and \( C \) be a strict \( n \)\nbd category.
    Then \( C \) is a stricter \( n \)\nbd category.
\end{thm}
\begin{proof}
    Let \( P \) be a regular directed complex with \( \dim P \le 3 \).
    By \cite[Corollary 8.4.12]{hadzihasanovic2024combinatorics}, \( \molecin{P} \) is a polygraph.
    Thus the map
    \begin{equation*}
        s_P \colon \colim_{x \in P} \imel{P}{x} \to P 
    \end{equation*}
    is an isomorphism in \( \omegaCat \).
    In particular, any strict \( n \)\nbd category \( C \) is local with respect to \( s_P \).
    By Lemma \ref{lem:stricter_n_iff_local_with_dim_le_n}, this concludes the proof.
\end{proof}

\begin{comm} \label{comm:strict_are_not_stricter}
    Thus, the category \( \nCat{n} \) and \( \snCat{n} \) are equivalent for \( n \le 3 \).
    However, they differ starting from \( n = 4 \).
    For instance, consider the \( 4 \)\nbd dimensional regular directed complex \( P \) of \cite[Example 8.2.20]{hadzihasanovic2024combinatorics}.
    Then \( \molecin{P} \) is \emph{not} a polygraph.
    Let \( Q \eqdef \colim_{x \in P} \molecin{\imel{P}{x}} \), where the colimit is computed in \( \omegaCat \).
    By construction, \( Q \) is a polygraph, but it cannot be a stricter \( \omega \)\nbd category.
    Indeed, if it were the case, we would have \( \molecin{P} \cong \rcs Q \cong Q \), contradicting the fact that \( \molecin{P} \) is not a polygraph.
\end{comm}

\subsection{Suspension of stricter \texorpdfstring{$\omega$}{ω}-categories}

\begin{dfn} 
    Let \( C \) be a composition structure, and \( a, b \) be \( 0 \)\nbd dimensional globular cells.
    We define the composition structure
    \begin{equation*}
        C(a, b) \eqdef \set{u \in C \mid \bd{0}{-} u = a, \bd{0}{+} u = b},    
    \end{equation*}
    whose boundary operators and \( k \)\nbd composition are induced by the one of \( C \) shifted by \( 1 \).
\end{dfn}

\begin{lem} \label{lem:hom_of_stricter_is_stricter}
    Let \( C \) be a stricter \( \omega \)\nbd category, and \( a, b \) be two \( 0 \)\nbd dimensional globular cells of \( C \).
    Then \( C(a, b) \) is a stricter \( \omega \)\nbd category.
\end{lem}
\begin{proof}
    Let \( U \) be a molecule.
    Any \( U \)\nbd matching family 
    \begin{equation*}
        \set{\F_x \colon \molecin{\imel{U}{x}} \to \C(a, b)}_{x \in U}
    \end{equation*}
    determines a \( \sus{U} \)\nbd matching family in \( C \) by letting \( \F_{\sus{x}} \eqdef \F_x \) for \( x \in U \), and \( \F_{\bot^-}, \F_{\bot^+} \) be the strict functors from \( \molecin{\pt} \) classifying \( a \) and \( b \) respectively. 
    This matching family has an amalgamation \( \sus{\F} \colon \molecin{\sus{U}} \to C \), showing that the amalgamation of \( \set{\F_x}_{x \in U} \) is well defined.
\end{proof}

\begin{comm}
    Thus, any stricter \( \omega \)\nbd category can canonically be seen as a category enriched in stricter \( \omega \)\nbd categories.
    The converse is not true. 
    Indeed if it were the case, \( \snCat{4} \) would be equivalent to the categories of categories enriched in \( \snCat{3} \), the latter being equivalent to \( \nCat{4} \) by Theorem \ref{thm:strict_le_3_are_stricter}.
    We know from Comment \ref{comm:strict_are_not_stricter} that this is not the case.
    We do not know any general condition for showing that a category enriched in stricter \( \omega \)\nbd category is stricter.
\end{comm}

\begin{dfn} [Suspension]
    Let \( C \) be a composition structure.
    The \emph{suspension of \( C \)} is the composition structure \( \sus{C} \) with two objects \( \bot^+, \bot^- \) and such that
    \begin{equation*}
        \sus{C}(\bot^\a, \bot^\beta) \eqdef 
        \begin{cases}
            C       & \text{if } (\a, \beta) = (-, +),\\
            \set{*} & \text{if } \a = \beta,\\
            \varnothing & \text{else.}
        \end{cases}
    \end{equation*}
    The boundary operator and \( k \)\nbd composition operation are induced by the one of \( C \). 
\end{dfn}

\noindent If \( C \) is a stricter \( \omega \)\nbd category, then \( \sus{C} \) is a strict \( \omega \)\nbd category, since it is the suspension of its underlying strict \( \omega \)\nbd category.
The goal of this section is to show that it is in fact a stricter \( \omega \)\nbd category. 

\begin{dfn} [Collapsible closed subset]
    Let \( U \) be a molecule, \( K \subseteq U \) be a closed subset, and \( \beta \in \set{-, +} \).
    We say that \( K \) is \emph{\( \beta \)\nbd collapsible} if \( K \) is non-empty and for all \( x \in U \), if \( \bd{0}{-\beta} x \in K \) then \( x \in K \).
    In that case, we let \( \coll K U \) be oriented graded poset whose underlying set is
    \begin{equation*}
        \set{\zcoll} \coprod \set{\icoll u \mid u \in U \setminus K},
    \end{equation*}
    and oriented covering diagram given by
    \begin{equation*}
        \cofaces{}{\a} x \eqdef
        \begin{cases}
            \set{ \icoll v \mid v \in \cofaces{}{\a} u} & \text{if } x = \icoll u,\\
            \set{ \icoll u \mid \dim u = 1, \faces{}{\a} u \in K} &\text{if } x = \bullet, \a = \beta\\
            \varnothing &\text{else.}
        \end{cases}
    \end{equation*}
    The underlying poset of \( \coll{K}{U} \) fits into the commutative square
    \begin{equation} \label{tik:square_collapse}
        \begin{tikzcd} 
            K & \pt \\
            U & {\coll K U}
            \arrow[""{name=0, anchor=center, inner sep=0}, two heads, from=1-1, to=1-2]
            \arrow[hook, from=1-1, to=2-1]
            \arrow[hook, from=1-2, to=2-2]
            \arrow[two heads, from=2-1, to=2-2]
            \arrow["\lrcorner"{anchor=center, pos=0.125, rotate=180}, draw=none, from=2-2, to=0]
        \end{tikzcd}
    \end{equation}
    where \( \pt \to \coll{K}{U} \) picks out the element \( \zcoll \), and \( \mapcoll \colon U \to \coll{K}{U} \) is the order preserving map defined by
    \begin{equation*}
        x \mapsto
        \begin{cases}
            \icoll x & \text{if } x \in U \setminus K\\
            \zcoll & \text{otherwise.}
        \end{cases}
    \end{equation*}
\end{dfn}

\begin{lem} \label{lem:collapsible_is_puhsout}
    Let \( U \) be a molecule, \( \beta \in \set{-, +} \) and \( K \subseteq U \) be a \( \beta \)\nbd collapsible subset.
    Then the commutative square (\ref{tik:square_collapse}) is a pushout in the category of posets.
\end{lem}
\begin{proof}
    Since the underlying square is a pushout of set, for the statement to be true, we claim that it is enough that for every element \( u \in U \setminus K \) covering an element \( k \) of \( K \), then either \( \dim u = 1 \), or there exists \( v \in U \setminus K \) such that \( v < u \) and \( v \) covers an element of \( K \).
    Indeed, in that case, the elements covering \( \zcoll \) in \( \coll{K}{U} \) with the universal partial order will be exactly of the form \( \icoll{u} \) for \( u \in U \setminus K \) with \( \dim u = 1 \). 
    Therefore, suppose that \( u \in U \setminus K \) of dimension \( n \geq 2 \) covers an element \( k \) of \( K \).
    Recall that in the augmentation of \( U \), we have a chain of covering \( u_{-1} \eqdef \bot \to u_0 \to u_1 \to \ldots \to u_n \eqdef u \) such that \( u_i \in \faces{}{\beta} u_{i + 1} \) for all \( - 1 \le i < n \) and \( \set{u_0} = \bd{0}{\beta} u \).
    Then none of the \( u_i \) belong to \( K \), since otherwise, \( u_0 \) would be in \( K \), hence so would be \( u \).
    Then pick any chain \( k_{- 1} = \bot \to k_1 \to \ldots \to k_{n - 1} \eqdef k \) from \( \bot \) to \( k \), which necessarily belong to \( K \), since it is closed.
    Then we have in the covering diagram of \( \augm{U} \) 
    \begin{center}
        \begin{tikzcd}
            & u \\
            {k_{n - 1}} && {u_{n - 1}} \\
            {k_1} && {u_0} \\
            & \bot.
            \arrow[no head, from=1-2, to=2-1]
            \arrow[no head, from=1-2, to=2-3]
            \arrow[dashed, no head, from=2-1, to=3-1]
            \arrow[dashed, no head, from=2-3, to=3-3]
            \arrow[no head, from=3-1, to=4-2]
            \arrow[from=3-3, to=4-2]
        \end{tikzcd}
    \end{center}
    By Lemma \ref{lem:diamond_transitive}, there exists a sequence of diamonds
    \begin{center}
        \begin{tikzcd}
            & {x_i} \\
            {y_i} && {y'_i} & {0 \le i \le r } \\
            & {z_i}
            \arrow[no head, from=1-2, to=2-1]
            \arrow[no head, from=1-2, to=2-3]
            \arrow[no head, from=2-1, to=3-2]
            \arrow[no head, from=2-3, to=3-2]
        \end{tikzcd}
    \end{center}
    rewriting the left chain to the right chain.
    Let \( i_0 \eqdef \max \set{i \mid y_i \in K} \), which exists since \( y_1 \in K \).
    If \( x_{i_0} < u \), then we are done since \( x_{i_0} \) covers \( y_{i_0} \in K \).
    Else \( x_{i_0} = u \).
    Then, either \( i_0 = r \), in which case \( y'_{i_0} = u_j \) for some \( j \), or \( i_0 < r \), in which case \( y'_{i_0} = y_{i + 1} \notin K \).
    In any case, \( y'_{i_0} \) is not in \( K \).
    Since \( y_{i_0} \in K \) is of dimension \( \geq 1 \) and \( K \) is closed, \( z_{i_0} \in K \).
    Thus we found a \( y'_{i_0} < u \) which is not in \( K \) and covers \( z_{i_0} \in K \).
    This concludes the proof.
\end{proof}


\begin{lem} \label{lem:path_from_zero_bd_to_all_points}
    Let \( U \) be a molecule, \( \a \in \set{-, +} \) and \( x \in \gr{0}{U} \).
    Then there exists \( k \geq 0 \) and a inclusion \( \iota \colon k\dglobe{1} \incl U \) such that \( \bd{0}{\a} \iota = \bd{0}{\a} U \) and \( \bd{0}{-\a} \iota = x \).
\end{lem}
\begin{proof}
    We proceed by induction on the submolecules of \( V \) of \( U \).
    The base case where \( V \) is the point is trivial.
    Suppose inductively that the statement is true of all proper submolecules \( V \) of \( U \).
    First suppose that \( U \) is an atom.
    If \( \dim U = 0 \), then the statement is trivial, otherwise \( \dim U > 0 \) and \( x \in \bd{}{\beta} U \) for some \( \beta \in \set{-, +} \).
    We conclude by inductive hypothesis and globularity.
    Now suppose split into \( U = V \cp{k} W \), and suppose that \( x \in V \), the case \( x \in W \) is symmetrical.
    Then either \( k > 0 \), hence \( \bd{0}{} V = \bd{0}{} U \) and we may conclude by inductive hypothesis on \( V \), or \( k = 0 \).
    If \( \a = - \), then \( \bd{0}{-} V = \bd{0}{-} U \), and we conclude by inductive hypothesis on \( V \) again.
    Else, \( \a = + \) and \( \bd{0}{+} U = \bd{0}{+} W \).
    Consider by inductive hypothesis an inclusion \( \iota \colon k\dglobe{1} \incl V \) from \( x \) to \( \bd{0}{+} V \).
    Then \( \bd{-}{0} W \) is isomorphic to \( k'\dglobe{1} \) for some \( k' > 0 \), hence \( \iota \cp{0} \bd{1}{-} \idd{W} \colon (k + k') \dglobe{1} \incl U \) is the desired path.
\end{proof}

\begin{lem} \label{lem:collapsible_connected_with_all_closure_element}
    Let \( U \) be a molecule, \( \beta \in \set{-, +} \), \( K \subseteq U \) be a \( \beta \)\nbd collapsible subset, and \( V \submol U \).
    Then \( V \cap K \) is either either empty or connected.
    In the latter case, \( V \cap K \subseteq V \) is \( \beta \)\nbd collapsible and contains \( \bd{0}{-\beta} V \).
\end{lem}
\begin{proof}
    Assume that \( \beta = - \), the case \( \beta = + \) is dual.
    Suppose that \( V \cap K \) is non-empty.
    We show that \( x \eqdef \bd{0}{-} V \) is in \( V \cap K \).
    Let \( z \in V \cap K \), then \( y \eqdef \bd{0}{-} z \in V \cap K \).
    By Lemma \ref{lem:path_from_zero_bd_to_all_points}, there exist \( k \geq 0 \) and \( \iota \colon k\arr \to V \) such that \( \bd{0}{} \iota = (x, y) \).
    Since \( y = \bd{0}{+} \iota \in K \) and \( K \) is collapsible, \( \cofaces{}{+} y \in K \).
    Since \( K \) is closed, \( \faces{}{-} \cofaces{}{+} y \in K \).
    Iterating this process, we find indeed that \( x \in K \).
    Then, suppose that \( V \cap K = F \cup G \) for two closed subsets \( F, G \) with \( F \cap G = \emptyset \), and suppose without loss of generality that \( x \in F \).
    Let \( y \in \gr{0}{G} \). 
    As previously, we have an inclusion \( \iota \colon k\arr \incl F \cup G \) such that \( \bd{0}{}\iota = (x, y) \).
    Then \( \iota(k\arr) = \invrs{\iota}(F) \cup \invrs{\iota}(G) \) and \( \invrs{\iota}F \cap \invrs{G} = \emptyset \).
    By \cite[Lemma 3.3.13]{hadzihasanovic2024combinatorics}, \( \invrs{\iota}F = \emptyset \) or \( \invrs{\iota}G = \emptyset \), a contradiction.
    This proves that \(  \gr{0}{G} = \emptyset \), thus that \( G = \emptyset \).
    This shows that \( V \cap K \) is connected.
    Finally, let \( x \in V \) such that \( \bd{0}{+} x \in K \).
    Since \( K \) is collapsible, \( x \in V \cap K \).
    Hence \( V \cap K \subseteq V \) is collapsible.
    This concludes the proof.
\end{proof}

\begin{lem} \label{lem:collapsible_negbeta_boundary_collapse_all}
    Let \( U \) be a molecule, \( \beta \in \set{-, +} \), and \( K \subseteq U \) be a \( \beta \)\nbd collapsible subset such that \( \bd{0}{-\beta} U \subseteq K \).
    Then \( K = U \).
\end{lem}
\begin{proof}
    We proceed induction on the layering dimension \( \ell \) of \( U \).
    If \( \ell = -1 \), then \( U \) is an atom, thus \( U = \clset{\top_U} \subseteq K \) by definition.
    Inductively, let \( \ell \geq 0 \). Then \( U \) admits a \( \ell \)\nbd layering
    \begin{equation*}
        U \eqdef \order{1}{U} \cp{\ell} \ldots \cp{\ell} \order{r}{U}
    \end{equation*}
    with \( r \geq 2 \) such that \( \order{i}{U} \) has layering dimension \( < \ell \) for all \( 1 \le i \le r \).
    Since \( \bd{0}{+} U = \bd{0}{+} \order{i}{U} \) and, by Lemma \ref{lem:collapsible_connected_with_all_closure_element}, \( \order{i}{U} \cap K \) is collapsible, we deduce by inductive hypothesis that \( \order{i}{U} = \order{i}{U} \cap K \).
    Therefore \( U = U \cap K \).
    This concludes the proof.
\end{proof}

\begin{lem} \label{lem:collapsible_mapcoll_preserve_boundaries}
    Let \( U \) be a molecule, \( \beta \in \set{-, +} \), \( K \subseteq U \) be a \( \beta \)\nbd collapsible subset, \( \a \in \set{-, +} \) and \( k \geq 0 \).
    Then 
    \begin{equation*}
        \mapcoll (\bd{k}{\a} U) = \bd{k}{\a} (\coll{K}{U}),
    \end{equation*}
    in particular \( \bd{0}{\beta} (\coll{K}{U}) = \set{\zcoll} \).
    Furthermore, \( \coll{K}{U} \) is globular, and if \( U \) is round, so is \( \coll{K}{U} \).
\end{lem}
\begin{proof}
    Without loss of generality, we assume that \( \beta = - \) and that \( K \) is a proper subset of \( U \), otherwise the statement is trivial.
    By Lemma \ref{lem:collapsible_connected_with_all_closure_element}, \( \bd{0}{-} U \subseteq K \).
    Suppose that \( k > 0 \), then
    \begin{equation*}
        \bd{k}{\a} (\coll{K}{U}) = \clos \faces{k}{\a} (\coll{K}{U}) \cup \gr{< k}{\clset{\maxel (\coll{K}{U})}} = \mapcoll(\bd{k}{\a} U) \cup \set{\zcoll}.
    \end{equation*}
    Since \( k > 0 \), Lemma \ref{lem:collapsible_connected_with_all_closure_element} implies that \( \set{\zcoll} = p(\bd{0}{-} U) \subseteq p(\bd{k}{\a} U) \).
    Thus \(  \mapcoll (\bd{k}{\a} U) = \bd{k}{\a} (\coll{K}{U}) \).
    Next suppose \( k = 0 \) and \( \a = + \).
    Then since \( K \) is a proper subset, \( \set{x} = \bd{0}{+} U \) is not a subset of \( K \) by Lemma \ref{lem:collapsible_negbeta_boundary_collapse_all}. 
    Now by definition of \( \coll{K}{U} \), one sees that \( \mapcoll (\bd{0}{+} U) = \bd{0}{+} (\coll{K}{U}) \).
    The next thing to see is therefore that \( \set{\zcoll} = \bd{0}{-} \coll{K}{U} \).
    By construction, \( \cofaces{}{+} \zcoll = \emptyset \), thus we have one inclusion.
    Let \( \icoll{u} \in \gr{0}{(U \setminus K)} \).
    If \( u \in \bd{0}{-} U \) then \( u \in K \) by Lemma \ref{lem:collapsible_connected_with_all_closure_element}, a contradiction.
    Thus there exists \( v \in \cofaces{}{+} u \), and since \( K \) is closed, \( v \notin K \).
    We find that \( \icoll{v} \in \cofaces{}{+} \icoll{u} \), meaning that \( u \notin \bd{0}{-} (\coll{K}{U}) \).
    This proves that \( \set{\zcoll} = \bd{0}{-} \coll{K}{U} \).
    By Lemma \ref{lem:collapsible_is_puhsout} and formal properties, for all closed subsets \( V \subseteq U \) such that \( V \) is a molecule, we have 
    \begin{equation*}
        p(V) \cong
        \begin{cases}
            V & \text{if } V \cap K = \emptyset, \\
            \coll{(V \cap K)}{V} & \text{else}.
        \end{cases}
    \end{equation*}
    With this together with the first part of the proof, one sees that \( \coll{K}{U} \) is globular.
    Last, suppose that \( U \) is round of dimension \( n \geq 0 \).
    Let \( k < n \), and let \( x \in (\bd{k}{-} (\coll{K}{U})) \cap (\bd{k}{+} (\coll{K}{U})) \).
    We must show that \( x \in \bd{k - 1}{}  (\coll{K}{U}) \)
    Suppose first that \( x = \zcoll \), then \( \set{x} = \bd{0}{-} (\coll{K}{U}) \).
    If \( k > 0 \) we are done by globularity, else \( k = 0 \).
    In that case, \( \zcoll \in \bd{0}{+} (\coll{K}{U}) \), so \( \bd{0}{+} U \subseteq K \), hence \( \coll{K}{U} = \pt \) by Lemma \ref{lem:collapsible_negbeta_boundary_collapse_all}.
    Thus \( n = 0 = k \), contradiction.
    Suppose now that \( x = \icoll{u} \) for some \( u \in U \setminus K \).
    By the first part of the proof, \( u \in \bd{k}{-} U \cap \bd{k}{+} U \), and since \( U \) is round, \( u \in \bd{k - 1}{} U \), so that by the first part of the proof again, \( \icoll{u} \in \bd{k - 1}{} (\coll{K}{U}) \).
    This concludes the proof.
\end{proof}

\begin{prop} \label{prop:collapsible_collapse_to_molecules}
    Let \( U \) be a molecule, \( \beta \in \set{-, +} \) and \( K \subseteq U \) be a \( \beta \)\nbd collapsible subset.
    Then \( \coll{K}{U} \) is a molecule and the canonical map \( \mapcoll \colon U \surj \coll{K}{U} \) is a final map of molecules.
\end{prop}
\begin{proof}
    Recall that a map of poset \( L \to \pt \) is final just when \( L \) is connected. 
    Let \( \mapcoll \colon U \surj \coll{K}{U} \) be the canonical map.
    Let \( V \submol U \) be a submolecule of \( U \). 
    If \( V \cap K = \emptyset \), then \( \restr{\mapcoll}{V} \colon V \surj \mapcoll(V) \) is the identity.
    Otherwise, by Lemma \ref{lem:collapsible_is_puhsout}, the diagram
    \begin{center}
        \begin{tikzcd}
            {V \cap K} & \pt \\
            V & {p(V)}
            \arrow[two heads, from=1-1, to=1-2]
            \arrow[from=1-1, to=2-1]
            \arrow[from=1-2, to=2-2]
            \arrow[two heads, from=2-1, to=2-2]
        \end{tikzcd}
    \end{center}
    is a pushout.
    By Lemma \ref{lem:collapsible_connected_with_all_closure_element}, \( V \cap K \) is connected, thus \( V \cap K \surj \pt \) is final.
    Since final maps of posets are stable under pushouts, we conclude that \( \restr{\mapcoll}{V} \colon V \surj \mapcoll(V) \) is final.
    This proves that for all submolecules \( V \submol U \), \( \restr{\mapcoll}{V} \colon V \surj \mapcoll(V) \) is final.
    Now let \( x \in U \), \( k \geq 0 \) and \( \a \in \set{-, +} \).
    Using either the fact that \( \restr{p}{\bd{k}{\a} x} \) is the identity if \( \clset{x} \cap K = \emptyset \) and Lemma \ref{lem:collapsible_mapcoll_preserve_boundaries} otherwise, we find that \( \mapcoll(\bd{k}{\a} x) = \bd{k}{\a} \mapcoll(x) \).
    This proves that, in case \( \coll{K}{U} \) is a regular directed complex, \( \mapcoll \) is a final map of regular directed complexes.

    We prove by induction on the submolecules \( V \) of \( U \) that \( \mapcoll(V) \) is a molecule.
    The base case where \( V \) is a point is clear.
    Suppose inductively that \( \mapcoll(V) \) is a molecule for all proper submolecules of \( U \).
    Suppose first that \( U \) is an atom.
    By inductive hypothesis, for \( \a \in \set{-, +} \), \( p(\bd{}{\a} U) \) is a molecule, and is round by Lemma \ref{lem:collapsible_mapcoll_preserve_boundaries}.
    Thus \( p(\bd{}{-} U) \celto p(\bd{}{+} U) \) is a well defined atom, and one sees that, unless \( K = U \), it is by construction isomorphic to \( \coll{K}{U} \).
    If \( K = U \), then \( U = \pt \) is also an atom. 
    Now suppose that \( U \) is a molecule.
    Then by inductive hypothesis, \( \coll{K}{U} \) is a regular directed complex.
    We conclude by \cite[Proposition 6.2.33]{hadzihasanovic2024combinatorics} and the first part of the proof.
\end{proof}

\begin{lem} \label{lem:has_two_point_is_susp}
    Let \( U \) be a molecule such that \( \gr{0}{U} = \bd{0}{-} U \coprod \bd{0}{+} U \).
    Then \( U = \sus{U'} \) for some molecule \( U' \).
\end{lem}
\begin{proof}
    We proceed by induction on the submolecules \( V \) of \( U \).
    The base case where \( V \) is a point is vacuously true, since \( \gr{0}{V} \) have exactly one element.
    Suppose inductively that the statement is true of all submolecules of \( U \)
    Suppose first that \( U \) is an atom.
    We distinguish two cases.
    Either \( U = \dglobe{1} \), in which case it is \( \sus{\pt} \), or \( U = V \celto W \) with \( \dim V = \dim W \geq 1 \).
    Then, 
    \begin{align} 
         \bd{0}{} V \subseteq \gr{0}{V} \subseteq \gr{0}{U} &= \bd{0}{} U = \bd{0}{} V, \text{ and} \nonumber\\
         \bd{0}{-} V \cap \bd{0}{+} V &= \emptyset \label{aln:grade_boundary_zero}
    \end{align}
    % and \(  \). 
    Thus \( \gr{0}{V} = \bd{0}{-} V \coprod \bd{0}{+} V\).
    Similarly, \( \gr{0}{W} = \bd{0}{-} W \coprod \bd{0}{+} W \).
    By inductive hypothesis on the submolecules, \( V = \sus{V'} \) and \( W = \sus{W'} \) for some molecules \( V' \) and \( W' \).
    By \cite[Proposition 7.3.16]{hadzihasanovic2024combinatorics}, \( U = \sus{(V' \celto W')} \).
    Suppose next that \( U \) splits into \( V \cp{k} W \) with \( \dim V, \dim W > k \).
    If \( k = 0 \), then \( \bd{0}{+} V \subseteq \gr{0}{U} \), but \( \bd{0}{+} V \cap (\bd{-}{0} U \coprod \bd{+}{0} U) \) is empty, contradicting the assumption.
    Thus \( k > 0 \) and (\ref{aln:grade_boundary_zero}) also holds of \( V \) and \( W \).
    We deduce by inductive hypothesis that \( V = \sus{V'} \) and \( W = \sus{W'} \).
    By \cite[Proposition 7.3.16]{hadzihasanovic2024combinatorics} again, \( U = \sus{(V' \cp{k - 1} W')} \).
    This concludes the proof.
\end{proof}

\begin{prop} \label{prop:desuspension}
    Let \( U \) be a molecule, and \( K^- \coprod K^+ \) be a partition of \( \gr{0}{U} \) such that for all \( \a \in \set{-, +} \), \( K^\beta \subseteq U \) is \( \beta \)\nbd collapsible.
    Then there exists a molecule \( V \), together with a commutative diagram
    \begin{center}
        \begin{tikzcd}
            {K^- \coprod K^+} & {\pt \coprod \pt} \\
            U & {\sus{V}}
            \arrow[two heads, from=1-1, to=1-2]
            \arrow[from=1-1, to=2-1]
            \arrow["{(\bot^-, \bot^+)}", from=1-2, to=2-2]
            \arrow["q"', two heads, from=2-1, to=2-2]
        \end{tikzcd}
    \end{center}
    whose underlying diagram in \( \Pos \) is a pushout, and such that \( q \) is a final map of molecules.
\end{prop}
\begin{proof}
    By Proposition \ref{prop:collapsible_collapse_to_molecules}, we have a final map of molecules \( p \colon U \surj \coll{K^-}{U} \) is a molecule, and it is straightforwards to see that \( p(K^+) \subseteq \coll{K^-}{U} \) is collapsible.
    Applying Proposition \ref{prop:collapsible_collapse_to_molecules} again, we have a final map of molecules \( q \colon U \surj W \), which is the pushout of the final map \( K^- \coprod K^- \to \pt \coprod \pt \) along the inclusion \( K^+ \coprod K^- \incl U \).
    By Lemma \ref{lem:collapsible_mapcoll_preserve_boundaries}, the inclusion \( \pt \coprod \pt \incl W \) has image \( \bd{0}{} W \).
    To conclude, we need to show that \( W = \sus{V} \) for some molecule \( V \).
    Since \( q \) is a surjective map of molecules, \( \gr{0}{W} = q(\gr{0}{U}) = q(K^-) \coprod q(K^+) = \pt \coprod \pt \).
    We conclude by Lemma \ref{lem:has_two_point_is_susp}.
\end{proof}

\begin{thm} \label{thm:suspension_of_stricter}
    Let \( C \) be a stricter \( \omega \)\nbd category.
    Then \( \sus{C} \) is a stricter \( \omega \)\nbd category.
\end{thm}
\begin{proof}
    Let \( U \) be a molecule and \( \set{\F_x \colon \molecin{\imel{U}{x}} \to \sus{C}}_{x \in U} \) be a \( U \)\nbd matching family in \( \sus{C} \).
    For \( \beta \in \set{-, +} \), let \( K^\beta \eqdef \set{x \in U \mid \pcell{\F_x} = \bot^\a} \).
    We claim that \( K^\beta \subseteq U \) is either empty or \( \beta \)\nbd contractible.
    Let \( x \in U \), and suppose that \( \bd{0}{-\beta} x \in K^\beta \).
    Then \( \bd{0}{-\beta} \pcell{F_x} = \bot^{\beta} \).
    If \( \pcell{\F_x} \neq \bot^\beta \), then by construction of \( \sus{C} \), \( \bd{0}{-\beta} \pcell{\F_x} = \bot^{-\beta} \).
    Thus \( \pcell{\F_x} = \bot^\beta \), hence \( x \in K^\beta \).
    This shows that \( K^\beta \) is either empty or \( \beta \)\nbd contractible.
    Now \( U = K^- \coprod K^+ \).
    Suppose that there exists \( \beta \in \set{-, +} \) such that \( K^\beta \) is empty and let \( x \in U \).
    If \( \pcell{\F_x} \in C \), we would have \( x \in K^\beta \).
    Thus \( \pcell{\F_x} = \bot^{-\beta} \), and \( K^{-\beta} = U \).
    Therefore, the candidate amalgamation \( \F \colon \molecin{U} \to \sus{C} \) is well defined since it factors as
    \begin{equation*}
        \molecin{U} \to \molecin{\pt} \to \sus{C},
    \end{equation*}
    where \( \molecin{\pt} \to \sus{C} \) classifies \( \bot^{-\beta} \).
    Now suppose that \( K^- \coprod K^+ \) forms a partition of \( U \).
    Then Proposition \ref{prop:desuspension} applies, and we have in particular a molecule \( V \) and a final map of molecules \( q \colon U \surj \sus{V} \). 
    By \cite[Theorem 6.2.35]{hadzihasanovic2024combinatorics}, \( \molecin{q} \) is a strict functor.
    Then, one checks that the family of globular cells of \( C \)
    \begin{equation*}
        \set{\pcell{\F_x} \mid q(x) = \sus{y}}_{y \in V}
    \end{equation*} 
    defines, as per Remark \ref{rmk:data_matching family}, a \( V \)\nbd matching family \( \set{\G_y \colon \molecin{\imel{V}{y}} \to C}_{y \in V} \) in \( C \).
    Since \( C \) is stricter, it admits a well defined amalgamation \( \G \colon \molecin{V} \to C \).
    Finally, the candidate amalgamation \( \F \colon \molecin{U} \to \sus{C} \) is well defined, since it factors as \( \F = \sus{\G} \after \molecin{q} \).
    This concludes the proof.
\end{proof}

\begin{dfn} [Bipointed stricter \( \omega \)\nbd category]
    A \emph{bipointed stricter \( \omega \)\nbd category} is given by a stricter \( \omega \)\nbd category \( C \) together with a pair of objects \( (a, b) \).
    We let \( \bpt\somegaCat \) be the category of bipointed stricter \( \omega \)\nbd category and strict functors that respect the bipointing.
\end{dfn}

\begin{cor} \label{cor:adjunction_hom_suspension}
    There is an adjunction
    \begin{equation*}
        \sus{} \colon \somegaCat \leftrightarrows \bpt\somegaCat \cocolon \hom,
    \end{equation*}
    where \( \sus{C} \) is the suspension of \( C \) bipointed by \( (\bot^-, \bot^+) \), and \( \hom(C, a, b) \) is given by \( C(a, b) \).
    Furthermore, the functor
    \begin{equation*}
        \sus{} \colon \somegaCat \to \somegaCat
    \end{equation*}
    preserves connected colimits.
\end{cor}
\begin{proof}
    That the pair of functors is well defined follows from Lemma \ref{lem:hom_of_stricter_is_stricter} and Theorem \ref{thm:suspension_of_stricter}.
    That this form an adjunction follows from standard arguments.
    Last, since \( \bpt\somegaCat \) is a coslice construction, \( \sus{} \colon \somegaCat \to \somegaCat \) preserves connected colimits by \cite[Proposition 3.3.8]{riehl2019context}.
\end{proof}

