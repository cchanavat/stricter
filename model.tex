\section{Homotopy theory of stricter \texorpdfstring{$\omega$}{ω}-categories}

\subsection{Folk model structure on stricter \texorpdfstring{$\omega$}{ω}-categories}

% \begin{thm} \label{thm:folk_model_structure}
%     There exists a model structure, called the \emph{folk model structure}, on \( \omegaCat \) such that:
%     \begin{enumerate}
%         \item every strict \( \omega \)\nbd categories is fibrant;
%         \item cofibrations are the retract of relative polygraphs. 
%     \end{enumerate} 
% \end{thm}
% \begin{proof} 
%     See \cite{lafont2010folk}.
% \end{proof}

\begin{dfn} [Stricter \( \omega \)\nbd category of cylinders]
    Let \( C \) be a stricter \( \omega \)\nbd category.
    The \emph{stricter \( \omega \)\nbd category of cylinder} is the stricter \( \omega \)\nbd category 
    \begin{equation*}
       \Gamma(C) \eqdef \homlax(\globe{1}, C). 
    \end{equation*}
\end{dfn}

\begin{rmk} \label{rmk:strict_stricter_same_cylinders}
    By Proposition \ref{prop:reflection_to_stricter_monoidal}, the stricter categories \( \Gamma(C) \) coincides with the strict \( \omega \)\nbd categories of cylinders \cite[Remark 20.2.9]{ara2025polygraphs}, which happen to be stricter, since \( C \) is.
\end{rmk}

% \begin{thm} \label{thm:folk_model_structure_on_stricter}
%     There is a cofibrantely generated model structure on the category \( \somegaCat \), called the \emph{folk model structure}, where:
%     \begin{enumerate}
%         \item every \( \omega \)\nbd categories is fibrant;
%         \item cofibration are the retracts of relative stricter polygraphs;
%     \end{enumerate}
%     Furthermore, this model structure is right transferred from the folk model structure on \( \omegaCat \) along the adjunction 
%     \begin{center}
%         \begin{tikzcd}
%             \somegaCat & \omegaCat,
%             \arrow[""{name=0, anchor=center, inner sep=0}, "\iota"', curve={height=12pt}, hook, from=1-1, to=1-2]
%             \arrow[""{name=1, anchor=center, inner sep=0}, "\rcs"', curve={height=12pt}, from=1-2, to=1-1]
%             \arrow["\dashv"{anchor=center, rotate=-90}, draw=none, from=1, to=0]
%         \end{tikzcd}
%     \end{center}
%     which is in particular a Quillen pair.
% \end{thm}
% \begin{proof}
%     This is a direct application of \cite[Proposition 21.3.2]{ara2025polygraphs} with Remark \ref{rmk:strict_stricter_same_cylinders} and Lemma \ref{lem:reflection_of_polygraphs_are_stricter_polygraphs}.
% \end{proof}

% \begin{rmk} 
%     By Corollary \ref{cor:pushout_principal_cell}, the set
%     \begin{equation*}
%         \Icof \eqdef \set{\molecin{\bd{}{}U} \incl \molecin{U} \mid U \text{ atom}}
%     \end{equation*}
%     is a generating set of cofibrations for the folk model structure on \( \somegaCat \).
% \end{rmk}

\begin{dfn}
    We write \( \Icof \) for the the set
    \begin{equation*}
        \Icof \eqdef \set{\molecin{\bd{}{}U} \incl \molecin{U} \mid U \text{ atom}}
    \end{equation*}
\end{dfn}

\begin{thm} \label{thm:folk_model_structure_on_stricter_n}
    Let \( n \in \mathbb{N} \).
    There is a cofibrantely generated model structure on the category \( \snCat{n} \), called the \emph{folk model structure}, where:
    \begin{enumerate}
        \item every \( n \)\nbd categories is fibrant.
        \item \( \trunc{n}\Icof \) is a set of generating cofibrations;
    \end{enumerate}
    Furthermore, this model structure is right transferred from the folk model structure on \( \nCat{n} \) along the adjunction 
    \begin{center}
        \begin{tikzcd}
            \snCat{n} & \nCat{n},
            \arrow[""{name=0, anchor=center, inner sep=0}, "\iota"', curve={height=12pt}, hook, from=1-1, to=1-2]
            \arrow[""{name=1, anchor=center, inner sep=0}, "\rcs"', curve={height=12pt}, from=1-2, to=1-1]
            \arrow["\dashv"{anchor=center, rotate=-90}, draw=none, from=1, to=0]
        \end{tikzcd}
    \end{center}
    which is in particular a Quillen pair.
\end{thm}
\begin{proof}
    This is a direct application of \cite[Proposition 21.3.2]{ara2025polygraphs} with Remark \ref{rmk:strict_stricter_same_cylinders} and Lemma \ref{lem:reflection_of_polygraphs_are_stricter_polygraphs}.
    By Corollary \ref{cor:pushout_principal_cell}, the set \( \trunc{n}\Icof \) is a generating set of cofibrations.
\end{proof}

\noindent Therefore, we have the following commutative square of left Quillen functors
\begin{center}
    \begin{tikzcd}
        \omegaCat & \somegaCat \\
        {\nCat{n}} & {\snCat{n}.}
        \arrow["\rcs", from=1-1, to=1-2]
        \arrow["{\trunc{n}}"', from=1-1, to=2-1]
        \arrow["{\trunc{n}}", from=1-2, to=2-2]
        \arrow["\rcs"', from=2-1, to=2-2]
    \end{tikzcd}
\end{center}

\subsection{Coherent walking equivalence}

We let \( \molecin{-} \colon \dgmSet \to \somegaCat \) be the left Kan extension along the Yoneda embedding of the functor \( \atom \to \somegaCat \) defined by \( U \mapsto \molecin{U} \).
Notice that by Corollary \ref{cor:regular_directed_complex_colimit_of_itself}, there is no ambiguity when one writes \( \molecin{P} \) for a regular directed complex \( P \).

\begin{lem} \label{lem:molecin_preserves_cofibration}
    Let \( f \colon X \incl Y \) be a monomorphism of diagrammatic sets.
    Then \( \molecin{f} \) is a relative stricter polygraph.
\end{lem}
\begin{proof}
    Let \( I \eqdef \set{\bd{}{} U \incl U \mid U \text{ atom}} \).
    Then \( f \) is a relative \( I \)\nbd cell complex by \cite[Remark 2.9]{chanavat2024htpy}.
    Since \( \molecin{-} \) is left adjoint, \( \molecin{f} \) is a relative \( \molecin{I} \)\nbd cell complex, that is, a relative stricter polygraph.
\end{proof}

\begin{cor} \label{cor:molecin_polygraph_with_basis}
    Let \( X \) be a diagrammatic set. 
    Then \( \molecin{X} \) is a stricter polygraph with generating set \( \cls{S} = \coprod_{k \geq 0} \cls{S}_k \), where
    \begin{equation*}
        \cls{S}_k \eqdef \set{\molecin{u} \colon \molecin{U} \to \molecin{X} \mid u \colon U \to X \in \gr{k}{\nd X}}.
    \end{equation*}
\end{cor}

\begin{lem} \label{lem:molecin_monoidal}
    The functor \( \molecin{-} \colon \dgmSet \to \somegaCat \) is monoidal with respect to the Gray product of diagrammatic sets and stricter \( \omega \)\nbd categories.
\end{lem}
\begin{proof}
    The result is true on atoms by definition of the Gray product of stricter \( \omega \)\nbd categories.
    We conclude by universal property of Day convolution.
\end{proof}

\begin{dfn} \label{dfn:strict_walking_equivalence}
    We recall from \cite{hadzihasanovic2024model} the construction of the \emph{coherent walking equivalence \( \wE \)}, which is a strict \( \omega \)\nbd category equipped with an inclusion \( \globe{0} \incl \wE \).
    We define \( \wE \) as a transfinite composition of inclusions \( \order{n}{\wE} \incl \order{n + 1}{\wE} \) such that
    \begin{itemize}
        \item \( \order{n}{\wE} \) is a strict \( n \)\nbd category, and
        \item for \( n \geq 1 \), \( \order{n}{\wE} \) is equipped with strict functors \( \iota_n \colon \order{n - 1}{\wE} \to \order{n}{\wE} \),
    as well as \( \fun{L}_k, \fun{R}_k \colon \sus{\order{n - 1}{\wE}} \to \wE \).
    \end{itemize}
    Then, \( \order{0}{\wE} \) is the set with two elements \( \set{x, y} \) and \( \order{1}{\wE} \) is the free category on three generators \( \set{a \colon x \to y, a^L \colon y \to x, a^R \colon y \to x} \), equipped with the evident inclusion \( \iota_1 \colon \set{x, y} \incl \order{1}{\wE} \), and
    \begin{equation*}
        \fun{L}_1 \colon \sus{x} \mapsto a \comp{0} a^L, \sus{y} \mapsto x, \quad\text{ and }\quad \fun{R}_1 \colon \sus{x} \mapsto a^R \comp{0} a, \sus{y} \mapsto y.
    \end{equation*}

    Let \( k > 1 \), and suppose that \( (\order{k - 1}{\wE}, \iota_{k - 1}, \fun{L}_{k - 1}, \fun{R}_{k - 1}) \) have been defined.
    Then \( (\order{k}{\wE}, \iota_{k}, \fun{L}_{k}, \fun{R}_{k}) \) is defined by the pushout
    \begin{center}
        \begin{tikzcd}[column sep=huge]
            {\sus{\order{k - 2}{\wE}} \coprod \sus{\order{k - 2}{\wE}}} & {\order{k - 1}{\wE}} \\
            {\sus{\order{k - 1}{\wE}} \coprod \sus{\order{k - 1}{\wE}}} & {\order{k}{\wE}}
            \arrow[""{name=0, anchor=center, inner sep=0}, "{(\fun{L}_{k-1},\fun{R}_{k-1})}", from=1-1, to=1-2]
            \arrow["{\sus{\iota_{k-1}} \coprod \sus{\iota_{k-1}}}"', from=1-1, to=2-1]
            \arrow["{\iota_k}", from=1-2, to=2-2]
            \arrow["{(\fun{L}_k,\fun{R}_k)}"', from=2-1, to=2-2]
            \arrow["\lrcorner"{anchor=center, pos=0.125, rotate=180}, draw=none, from=2-2, to=0]
        \end{tikzcd}
    \end{center}
    in \( \omegaCat \).
    The inclusion \( \globe{0} \incl \wE \) then classifies \( x \in \order{0}{\wE} \).
    We denote by \( \fun{L}_\infty, \fun{R}_\infty \colon \sus{\wE} \to \wE \) the transfinite composition of \( (\fun{L}_k)_k \) and \( (\fun{R}_k)_k \) respectively.
\end{dfn}


\begin{dfn} [Stricter coherent walking equivalence]
    The \emph{stricter coherent walking equivalence} is the stricter \( \omega \)\nbd category \( \swE \eqdef \rcs \wE \), which comes equipped with the inclusion \( j \colon \globe{0} \incl \swE \).
\end{dfn}

\begin{lem} \label{lem:inclusion_into_stricter_is_equivalence}
    The inclusion \( j \colon \globe{0} \incl \swE \) is an acyclic cofibration in the folk model structure on stricter \( \omega \)\nbd categories.
\end{lem}
\begin{proof}
    Let \( i \colon \dglobe{0} \incl \wE \) be the inclusion of strict \( \omega \)\nbd categories such that \( j = \rcs i \).
    By \cite[Remark 1.29, Theorem 1.33]{hadzihasanovic2024model} and the two-out-of-three, \( i \) a cofibration and a weak equivalence.
    By Theorem \ref{thm:folk_model_structure_on_stricter_n}, \( j \) is an acyclic cofibration. 
\end{proof}

\begin{lem} \label{lem:swE_is_iso_to_molecin_loc_globe}
    The stricter \( \omega \)\nbd categories \( \swE \) and \( \molecin{\rglobe{1}} \) are isomorphic.
\end{lem}
\begin{proof}
    By \cite[Remark 1.29]{hadzihasanovic2024model}, \( \swE \) is the stricter polygraph whose \( k \)\nbd globular cells are
    generated by the set \( \cls{S}_k \), which can be describe inductively as
    \begin{equation*}
        \cls{S}_0 = \set{x, y}, \cls{S}_1 = \set{a, a^L, a^R},
    \end{equation*} 
    and for \( k > 1 \),
    \begin{equation*}
        \cls{S}_k = \set{\fun{L}_\infty(\sus{u}), \fun{R}_\infty(\sus{u}) \mid u \in \cls{S}_{k - 1}}.
    \end{equation*}
    Letting \( \fun{L} \eqdef \fun{L}_\infty(\sus{}) \) and \( \fun{R} \eqdef \fun{R}_\infty(\sus{}) \), we canonically identify elements of \( \cls{S}_k \)
    with
    \begin{equation*}
        \cls{S}_k \cong \set{(a, \s), (a^L, \s), (a^R, \s) \mid \s \in \set{L, R}^*},
    \end{equation*}
    by letting \( (a, \langle\rangle) \eqdef a, (a^L, \langle\rangle) \eqdef a^L \), and \( (a^R, \langle\rangle) \eqdef a^R \). 
    Then inductively on \( \s \in \set{L, R}^* \), we let 
    \begin{align*}
        (a, L\s)   & \eqdef \fun{L}_\infty(\sus{(a, \s)},  \\
        (a^L, L\s) & \eqdef \fun{L}_\infty(\sus{(a^L, \s)}), \\
        (a^R, L\s) & \eqdef \fun{L}_\infty(\sus{(a^R, \s)})),
    \end{align*}
    and similarly with \( \fun{R}_\infty \) for the case \( \s R \).    
    By Corollary \ref{cor:molecin_polygraph_with_basis}, \( \molecin{\rglobe{1}} \) is the stricter polygraph whose generating set is given, in the notation of (\ref{dfn:localisation}) by \( \coprod_{k \geq 0} \cls{T}_k \), where, letting \( b \colon \dglobe{1} \to \dglobe{1} \) be the identity, 
    \begin{equation*}
        \cls{T}_0 = \set{0^-, 0^+}, \cls{T}_1 = \set{b, b^L, b^R},
    \end{equation*}
    and for \( k > 1 \),
    \begin{equation*}
        \cls{T}_k = \set{\hinv{\s}b, (\hinv{\s}b)^L, (\hinv{\s}b)^R \mid \s \in \set{L, R}^{k - 1}}.
    \end{equation*}
    Then, we define inductively on \( k \geq 0 \) a family of bijections \( \phi_k \colon \cls{T}_k \cong \cls{S}_k \). 
    We et \( (0^-, 0^+) \mapsto (x, y) \) for the base case. 
    Inductively for \( k > 0 \), we define
    \begin{equation*}
        \hinv{\s}b \mapsto (b, \s),\quad (\hinv{\s}b)^L \mapsto (b^L, \s),\quad (\hinv{\s}b)^R \mapsto (b^R, \s).
    \end{equation*}
    Then, we have that \( \phi \colon \molecin{\rglobe{1}} \to \swE \) induces a strict functor, which is therefore an isomorphism.
    This concludes the proof.
\end{proof}

\begin{lem} \label{lem:suspension_commute_selfloc}
    Let \( U \) be an atom of dimension \( \geq 1 \).
    Then \( \sus{\selfloc{U}} \) and \( \selfloc{\sus{U}} \) are isomorphic. 
\end{lem}
\begin{proof}
    By \cite[Lemma 3.3.13]{hadzihasanovic2024combinatorics}, \( U \) is connected.
    Then by induction on the construction of the localisation, it is clear that \( \selfloc{U} \) is connected.
    Therefore, by Proposition \ref{prop:suspension_of_dgmSet}, 
    \begin{equation*}
        \sus{\left(\colim_{v \colon V \to \selfloc{U}} V\right)} \cong \colim_{v \colon V \to \selfloc{U}} \sus{V}
    \end{equation*}
    Then, for all \( \s \in \set{L, R}^* \), \( \sus{\hcyl{\s}U} \) and \( \hcyl{\s} \sus{U} \) are isomorphic.
    By induction on the construction of the localisation, \( \selfloc{\sus{U}} \) and \( \colim\limits_{v \colon V \to \selfloc{U}} \sus{V} \) are isomorphic.    
    This concludes the proof.
\end{proof}

\begin{lem} \label{lem:sus_commute_molec_connected}
    Let \( X \) be a connected diagrammatic set.
    Then \( \molecin{\sus{X}} \) and \( \sus{\molecin{X}} \) are isomorphic.
\end{lem}
\begin{proof}
    Follows by Corollary \ref{cor:adjunction_hom_suspension} and Proposition \ref{prop:suspension_of_dgmSet}. 
\end{proof}

\begin{prop} \label{prop:walking_eq_of_dim_n}
    Let \( n \geq 0 \), and call \( i_n \colon \dglobe{n + 1} \to \rglobe{n + 1} \) the canonical inclusion.
    Then 
    \begin{equation*}
        \molecin{(\bd{}{-} i_n)} \colon \molecin{\dglobe{n}} \to \molecin{\rglobe{n + 1}} 
    \end{equation*}
    is an acyclic cofibration in the folk model structure for stricter \( \omega \)\nbd categories.
\end{prop}
\begin{proof}
    By Lemma \ref{lem:inclusion_into_stricter_is_equivalence} and Lemma \ref{lem:swE_is_iso_to_molecin_loc_globe}, \( \molecin{(\bd{}{-} i_0)} \) is an acyclic cofibration.
    By a straightforward variation of \cite[Proposition 2.8]{hadzihasanovic2024model}, the functor \( \sus{} \colon \somegaCat \to \somegaCat \) preserves acyclic cofibrations, thus \( \sus{}\cdots\sus{\molecin{(\bd{}{-} i_0)}} \), where \( \sus{} \) is applied \( n \)\nbd times, is a acyclic cofibration.
    Using Lemma \ref{lem:sus_commute_molec_connected} and Lemma \ref{lem:suspension_commute_selfloc}, one sees that the latter is in fact isomorphic to \( \molecin{(\bd{}{-} i_n)} \).
    This concludes the proof.
\end{proof}

\subsection{Right transfer from the diagrammatic model structures}

\begin{lem} \label{lem:pushout_with_localisation}
    Let \( s \colon U \sd V \) be a subdivision between atoms of dimension \( \geq 1 \)
    Then there is a strict functor \( \td{s} \colon \molecin{\selfloc{U}} \to \molecin{\selfloc V} \) fitting in a pushout diagram
    \begin{center}
        \begin{tikzcd}
            {\molecin{U}} & {\molecin{\selfloc{U}}} \\
            {\molecin{V}} & {\molecin{\selfloc V}}
            \arrow[""{name=0, anchor=center, inner sep=0}, hook, from=1-1, to=1-2]
            \arrow["{\molecin{s}}"', from=1-1, to=2-1]
            \arrow["{\tilde s}", from=1-2, to=2-2]
            \arrow[hook, from=2-1, to=2-2]
            \arrow["\lrcorner"{anchor=center, pos=0.125, rotate=180}, draw=none, from=2-2, to=0]
        \end{tikzcd}
    \end{center}
\end{lem}
\begin{proof}
    Let \( n \geq 1 \) be the dimension of \( U \)
    Let \( u \colon U \to U \) and \( v \colon V \to V \) be the identity.
    For each \( k \geq 0 \), let \( \order{k}{U} \) and \( \order{k}{V} \) be the marked diagrammatic sets corresponding respectively to the \( n \)\nbd step of the localisation as in (\ref{dfn:localisation}).
    Note that we have
    \begin{align*}
        \nd \gr{n + k}{\order{k}{U}} &= \set{\hinv{\s}u \colon \hcyl{\s} U \to \order{k - 1}{U}, \hinv{\s}^Lu, \hinv{\s}^Ru \colon \dual{n + k}{\hcyl{\s} U} \to \order{k - 1}{U}}_{\s \in \set{L, R}^k},\\
        \nd \gr{n + k}{\order{k}{V}} &= \set{\hinv{\s}v \colon \hcyl{\s} V \to \order{k - 1}{V}, \hinv{\s}^Lv, \hinv{\s}^Rv \colon \dual{n + k}{\hcyl{\s} V} \to \order{k - 1}{U}}_{\s \in \set{L, R}^k}.
    \end{align*}
    We will then show by induction on \( k \geq 0 \) that letting \( c_0 \eqdef \molecin{c} \), and for all \( k > 0 \), that \( s_k \colon \molecin{\order{k}{U}} \to \molecin{\order{k}{V}} \) by letting, for all \( \s \in \set{L, R}^k \),
    \begin{equation*}
        \hinv{\s}u \mapsto \hinv{\s}v,\quad \hinv{\s}^Lu, \mapsto \hinv{\s}^Lv,\quad \hinv{\s}^Ru, \mapsto \hinv{\s}^Rv,
    \end{equation*}
    is a well defined strict functor fitting in a pushout square
    \begin{center} \label{tik:inductive_square_subdivision_localisation}
        \begin{tikzcd}
            {\molecin U} & {\molecin{\order k U}} \\
            {\molecin V} & {\molecin {\order k V}.}
            \arrow[""{name=0, anchor=center, inner sep=0}, from=1-1, to=1-2]
            \arrow["{s_0}"', from=1-1, to=2-1]
            \arrow["{s_k}", from=1-2, to=2-2]
            \arrow[from=2-1, to=2-2]
            \arrow["\lrcorner"{anchor=center, pos=0.125, rotate=180}, draw=none, from=2-2, to=0]
        \end{tikzcd}
    \end{center}
    The base case is evident.
    Inductively, let \( k > 0 \).
    To make the following diagrams more readable, we omit writing \( \molecin{(-)} \). 
    Then \( {\molecin{\order k U}} \) and \( {\molecin{\order k V}} \) are produced from \( {\molecin{\order {k - 1} U}} \) and \( {\molecin{\order {k - 1} V}} \) by first attaching left and right inverses of the elements of the form \( \hinv{\s}u \) and \( \hinv{\s}v \) for \( \s \in \set{L, R}^{k - 1} \), giving the following pushouts in \( \somegaCat \)
    \begin{center}
        \begin{tikzcd}
            {\coprod\limits_{\s \in \set{L, R}^{k - 1}} \bd{}{}\dual{n + k - 1}{\hcyl{\s}U}} & {\coprod\limits_{\s \in \set{L, R}^{k - 1}} \dual{n + k - 1}{\hcyl{\s}U}} \\
            {\order{k - 1} U} & {U'}
            \arrow[""{name=0, anchor=center, inner sep=0}, from=1-1, to=1-2]
            \arrow[from=1-1, to=2-1]
            \arrow[from=1-2, to=2-2]
            \arrow[from=2-1, to=2-2]
            \arrow["\lrcorner"{anchor=center, pos=0.125, rotate=180}, draw=none, from=2-2, to=0]
        \end{tikzcd}
    \end{center}
    and 
    \begin{center}
        \begin{tikzcd}
            {\coprod\limits_{\s \in \set{L, R}^{k - 1}} \bd{}{}\dual{n + k - 1}{\hcyl{\s}V}} & {\coprod\limits_{\s \in \set{L, R}^{k - 1}} \dual{n + k - 1}{\hcyl{\s}V}} \\
            {\order{k - 1} V} & {V'}
            \arrow[""{name=0, anchor=center, inner sep=0}, from=1-1, to=1-2]
            \arrow[from=1-1, to=2-1]
            \arrow[from=1-2, to=2-2]
            \arrow[from=2-1, to=2-2]
            \arrow["\lrcorner"{anchor=center, pos=0.125, rotate=180}, draw=none, from=2-2, to=0]
        \end{tikzcd}
    \end{center}
    By Lemma \ref{lem:subdivision_of_invertors}, Corollary \ref{cor:pushout_principal_cell}, we have the pushout square
    \begin{center}
        \begin{tikzcd}
            {\coprod\limits_{\s \in \set{L, R}^{k - 1}} \bd{}{}\dual{n + k - 1}{\hcyl{\s}U}} & {\coprod\limits_{\s \in \set{L, R}^{k - 1}} \dual{n + k - 1}{\hcyl{\s}U}} \\
            {\coprod\limits_{\s \in \set{L, R}^{k - 1}} \bd{}{}\dual{n + k - 1}{\hcyl{\s}V}} & {\coprod\limits_{\s \in \set{L, R}^{k - 1}} \dual{n + k - 1}{\hcyl{\s}V}}
            \arrow[""{name=0, anchor=center, inner sep=0}, from=1-1, to=1-2]
            \arrow[from=1-1, to=2-1]
            \arrow["{\coprod \dual{n + k - 1}{\hcyl{\s}c}}", from=1-2, to=2-2]
            \arrow[from=2-1, to=2-2]
            \arrow["\lrcorner"{anchor=center, pos=0.125, rotate=180}, draw=none, from=2-2, to=0]
        \end{tikzcd}
    \end{center}
    Combined with the inductive pushout square (\ref{tik:inductive_square_subdivision_localisation}), pasting law for pushouts therefore gives a strict functor \( s' \colon \molecin{U'} \to \molecin{V'} \) being the pushout of \( s_k \) along \( \molecin{\order{k - 1}{U}} \to \molecin{U'} \).
    Finally, \( \molecin{\order k U} \) and \(  {\molecin{\order k V}} \) are given by the following cellular extensions of \( U' \) and \( V' \):
    \begin{center}
        \begin{tikzcd}[column sep=small]
            {\coprod\limits_{\s \in \set{L, R}^k} \bd{}{}\hcyl{\s}U} & {\coprod\limits_{\s \in \set{L, R}^k} \hcyl{\s}U} & {\coprod\limits_{\s \in \set{L, R}^k} \bd{}{}\hcyl{\s}V} & {\coprod\limits_{\s \in \set{L, R}^k} \hcyl{\s}V} \\
            {U'} & {\order k U} & {V'} & {\order k V}.
            \arrow[""{name=0, anchor=center, inner sep=0}, from=1-1, to=1-2]
            \arrow[from=1-1, to=2-1]
            \arrow[from=1-2, to=2-2]
            \arrow[""{name=1, anchor=center, inner sep=0}, from=1-3, to=1-4]
            \arrow[from=1-3, to=2-3]
            \arrow[from=1-4, to=2-4]
            \arrow[from=2-1, to=2-2]
            \arrow[from=2-3, to=2-4]
            \arrow["\lrcorner"{anchor=center, pos=0.125, rotate=180}, draw=none, from=2-2, to=0]
            \arrow["\lrcorner"{anchor=center, pos=0.125, rotate=180}, draw=none, from=2-4, to=1]
        \end{tikzcd}
    \end{center}
    By a similar argument, the pushout of \( s' \colon  \molecin{U'} \to \molecin{V'} \) along the strict functor \( \molecin{U'} \to \molecin{\order{k}{U}} \) is \( s_k \colon \molecin{\order{k}{U}} \to \molecin{\order{k}{V}} \).
    The pasting law for pushout concludes the inductive step.
    By transfinite composition \( \tilde{s} \colon \molecin{\selfloc{U}} \to \molecin{\selfloc{V}} \) is the pushout of \( \molecin{s} \) along \( \molecin{U} \to \molecin{\selfloc U} \).
\end{proof}

\begin{prop} \label{prop:molecin_send_Jcomp_to_acof}
    Let \( U \) be a round molecule.
    Then
    \begin{equation*}
        \molecin{c_U} \colon \molecin{U} \to \molecin{(U \simeq \compos{U})}
    \end{equation*}
    is an acyclic cofibration in the folk model structure for stricter \( \omega \)\nbd categories.
\end{prop}
\begin{proof}
    Let \( n \eqdef \dim U \), and \( s \colon \dglobe{n} \sd U \) be the unique subdivision.
    By Lemma \ref{lem:pushout_with_localisation}, we have the two commutative squares
    \begin{center}
        \begin{tikzcd}
            {\molecin{\dglobe n}} & {\molecin{\dglobe {n + 1}}} & {\molecin{\rglobe {n + 1}}} \\
            {\molecin U} & {\molecin{(U \celto \compos U)}} & {\molecin{(U \simeq \compos U)}}
            \arrow[""{name=0, anchor=center, inner sep=0}, from=1-1, to=1-2]
            \arrow["{{\molecin{s}}}"', from=1-1, to=2-1]
            \arrow[""{name=1, anchor=center, inner sep=0}, from=1-2, to=1-3]
            \arrow[from=1-2, to=2-2]
            \arrow[from=1-3, to=2-3]
            \arrow[from=2-1, to=2-2]
            \arrow[from=2-2, to=2-3]
            \arrow["\lrcorner"{anchor=center, pos=0.125, rotate=180}, draw=none, from=2-2, to=0]
            \arrow["\lrcorner"{anchor=center, pos=0.125, rotate=180}, draw=none, from=2-3, to=1]
        \end{tikzcd}
    \end{center}
    where the right square is a pushout.
    By Lemma \ref{lem:generalised_substitution}, the left square is also a pushout.
    By the pasting law for pushouts, the outer square is a pushout.
    Since composite of the top horizontal strict functors is an acyclic cofibration by Proposition \ref{prop:walking_eq_of_dim_n}, we conclude that \( \molecin{c_u} \) is an acyclic cofibration.
\end{proof}

% \begin{dfn} [Diagrammatic nerve]
%     The \emph{diagrammatic nerve} is the right adjoint \( \N{} \) to the functor \( \molecin{-} \).
% \end{dfn}

\begin{dfn} [\( n \)\nbd truncated diagrammatic nerve]
    For \( n \in \mathbb{N} \cup \set{\omega} \), the \( n \)\nbd truncated diagrammatic nerve is the right adjoint \( \N{n} \) to the functor \( \trunc{n} \after \molecin{-} \).
    If \( n = \omega \), \( \trunc{\omega} \) is the identify, and we call \( \N{} \eqdef \N{\omega} \) the \emph{diagrammatic nerve}.
\end{dfn}

\begin{rmk}
    For all stricter \( n \)\nbd categories, \( \N{n} C \) is equal to \( \N{} C \), where in the latter expression, the stricter \( n \)\nbd category \( C \) is seen as a stricter \( \omega \)\nbd category.
\end{rmk}


\begin{prop} \label{prop:quillen_folk_dgm_infty}
    The folk model structure on stricter \( \omega \)\nbd categories is right transferred from the \( (\infty, \infty) \)\nbd model structure on \( \dgmSet \) along the diagrammatic nerve \( \N{} \colon \somegaCat \to \dgmSet \).
    In particular, the adjunction 
    \begin{equation*}
        \molecin{-} \colon \dgmSet \leftrightarrows \somegaCat \cocolon \N{}
    \end{equation*}
    is a Quillen adjunction.
\end{prop}
\begin{proof}
    Recall from Theorem \ref{thm:n_model_structure_on_dgm_set} that \( I \eqdef \set{\bd{}{} U \incl U \mid U \text{ atom}} \) is a generating set of cofibrations for the \( (\infty, \infty) \)\nbd model structure on diagrammatic sets, while \( \Jcomp \) is a pseudo-generating set of acyclic cofibration. 
    By Lemma \ref{lem:molecin_preserves_cofibration}, Proposition \ref{prop:molecin_send_Jcomp_to_acof}, \( \molecin{-} \) preserves cofibrations and a pseudo-generating set of acyclic cofibrations.
    By \cite[E.2.14]{joyal2008theory}, \( \molecin{-} \) is left Quillen. 
    Since all objects are fibrant in the folk model structure, Ken Brown's Lemma implies that \( \N{} \) preserves all weak equivalences, and in particular acyclic cofibrations. 
    Thus, \( \N{} \) takes relative \( \molecin{\Jcomp} \)\nbd complexes to weak equivalences.
    By \cite[Theorem 11.3.2]{hirschhorn2003model}, there exists a model structure on \( \somegaCat \), let us call it the transferred model structure, cofibrantely generated by \( \molecin{I} \) and \( \molecin{\Jcomp} \).
    Since \( \molecin{I} \) contains the generating cofibration of the folk model structure, the transferred and the folk model structure have the same cofibrations.
    Since all stricter \( \omega \)\nbd categories are fibrant in the folk model structure, they have in particular the right lifting property against \( \molecin{\Jcomp} \).
    Thus all stricter \( \omega \)\nbd categories are fibrant in the transferred model structure.
    By \cite[Proposition E.1.10]{joyal2008theory}, the transferred and the folk model structure coincide.
    This concludes the proof.
\end{proof}

\begin{thm}\label{thm:quillen_folk_dgm_n}
    Let \( n \in \mathbb{N} \cup \set{\omega} \).
    Then the folk model structure on stricter \( n \)\nbd categories is right transferred from the \( (\infty, n) \)\nbd model structure on \( \dgmSet \) along the \( n \)\nbd truncated diagrammatic nerve \( \N{n} \colon \snCat{n} \to \dgmSet \).
    In particular, the adjunction 
    \begin{equation*}
        \molecin{-} \colon \dgmSet \leftrightarrows \snCat{n} \cocolon \N{n}
    \end{equation*}
    is a Quillen adjunction.
\end{thm}
\begin{proof}
    The case \( n = \omega \) is Proposition \ref{prop:quillen_folk_dgm_infty}
    A look at the generating cofibrations and acyclic cofibration of Theorem \ref{thm:folk_model_structure_on_stricter_n} indicates that the folk model structure on stricter \( n \)\nbd categories is right transferred along the right adjoint inclusion \( \snCat{n} \incl \somegaCat \).  
    In particular, the functor \( \trunc{n} \colon \somegaCat \to \snCat{n} \) is left Quillen.
    Let \( U \) be an atom of dimension \( k > n \), and \( i \colon U \incl \selfloc{U} \) be the canonical inclusion.
    A direct inspection and Corollary \ref{cor:molecin_polygraph_with_basis} show that \( \trunc{n}\molecin{i} \) is an isomorphism.
    Thus \( \trunc{n} \after \molecin{(\Jcomp \cup \Jn{n})} \) is a set of acyclic cofibrations.
    By \cite[E.2.14]{joyal2008theory}, \( \molecin{-} \) is left Quillen. 
    Since ``being right transferred'' is a compatible with composition of right Quillen functors, we conclude by Proposition \ref{prop:quillen_folk_dgm_infty}.
\end{proof}

\noindent By Day convolution, diagrammatic sets support a biclosed monoidal structure \( - \gray - \) which restricts to the Gray product on atoms.

\begin{prop} \label{prop:Gray_monoidal}
    Let \( n \in \mathbb{N} \cup \set{\omega} \).
    Then the folk model structure is monoidal with respect to the Gray product.
    Furthermore, the functors \( \molecin{-} \colon \dgmSet \to \snCat{n} \) and \( \rcs \colon \nCat{n} \to \snCat{n} \) are strong monoidal left Quillen functors for the Gray products, when \( \dgmSet \) and \( \nCat{n} \) are endowed with the \( (\infty, n) \) and the folk model structures respectively.
\end{prop}
\begin{proof}
    The functor \( \rcs \) is strong monoidal by Proposition \ref{prop:reflection_to_stricter_monoidal}, and \( \molecin{-} \) is monoidal by definition and universal property of the Day convolution.
    Since the folk model structure is right transferred along the right adjoint of \( \rcs \) by Theorem \ref{thm:folk_model_structure_on_stricter_n} and also of \( \molecin{-} \) by Theorem \ref{thm:quillen_folk_dgm_n}, we conclude either by \cite[Theorem 5.6]{ara2020monoidal} or by \cite[Theorem 5.10]{chanavat2025gray}.
\end{proof}